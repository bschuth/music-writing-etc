\chapter{3 December 2012}

If you were not at Sunday’s performance by the Discovery Ensemble at Sanders Theater, you should have been. If you value classical music, if you want challenging but accessible programs, if you want to see the astonishing level of expertise of Boston’s young professional musicians, you need to see this orchestra. Engagement is written on the faces of the players, and audible in each piece played. I am going to quibble a bit with some interpretive decisions, but don’t let this give you any excuse to see them. It was wonderful to see so many young, committed players producing such impressive music.

We heard Béla Bartók’s three-movement Divertimento; Esa-Pekka Salonen’s Five Images after Sappho; and Beethoven’s Second Symphony. Music Director Courtney Lewis has fashioned a recognizable style: all the performances this afternoon were brisk, with a bright (but rich) string sound. In the outer movements of the Bartók the orchestra combined propulsive drive with feats of ensemble playing – perfect tutti entrances on unisons, asymmetric blocks of chords artfully voiced. Bartók calls for harmonies that grind against each other; here, they ground together with vigor, but still produced a transparent texture. The first movement is, among other things, an exercise in contrasts — loud and soft, agitato and tranquillo, tutti and solo quartet, thick chord clusters and sixteenth-note filigree — and contrast is something over which the Discovery Ensemble has consummate control. The second movement begins with a slow rising and falling motive in the low strings which was haunting and impeccably shaped, played at a bare whisper yet filling the room with sound. The third movement has a wonderful moment where it leaves behind its breakneck pace and most of the strings play pizzicato while the cellos play glissandos below them – this was so convincingly played as to elicit a murmur of laughter from the audience at the sudden change in mood.

Salonen’s Five Images After Sappho is a tour-de-force of orchestration. The mixture of players is unusual — a soprano soloist singing over a bare string quartet plus bass, with winds (many of whom were doubling on bass versions of their instruments), celeste, harp and much percussion. Salonen uses this palette masterfully to create soundscapes of vibrant and widely ranging color. Soprano Karin Wolverton has an impressive voice, combined with sensitive flexible musicianship. The soprano in this piece is both soloist as well as a featured instrument in the orchestra. She was able to blend or stand out as the music demanded. She effortlessly negotiated the often tricky part, which is fond of sudden and surprisingly high leaps. The texts are not set so as to maximize their comprehensibility; when the soprano sings “I shall sing beautifully”, “beautifully” is stretched out into a sinuous melisma which is itself quite beautiful, but the word becomes unrecognizable. According to the Salonen, the piece is meant to describe the “first part” of a woman’s life, using representative fragments from Sappho. The music is not particularly successful at conveying that particular goal. But if you accept that it is music of fascinating and gorgeous surfaces rather than depth, you may find the surfaces sufficiently ravishing to stand on their own. To pick just two examples, the second movement’s glistening descending lines were like falling leaves in bright sunshine, while the long, long line of the soprano in the fourth movement evoked the space inhabited by the evening star about which she was singing.

The afternoon ended with the Beethoven, where I found the briskness and brightness of the orchestra becoming a little overbearing, as if the high energy Mr. Lewis and his orchestra were creating together was just a little out of control. When Beethoven moved rapidly from loud to soft, something that happens with regularity, the room echo from the loud passages occasionally obscured the quiet that followed. There were a couple of moments where the first movement felt just a little strained, the only time during the afternoon that there was even a hint of stress in the orchestra. But even with these caveats, the performance was excellent — and the very end of the symphony, where Beethoven verges into something like good-natured violence, we received an enjoyable battering.

Mr. Lewis has stated that his goal with the Discovery Ensemble is to emulate the great chamber orchestras of Europe while showcasing the young talent this city has in overplus. They appear to have succeeded at this goal, which makes it all the more frustrating that there were not more of us in Sanders Theatre as witnesses. I hope this can be remedied when they play Rossini, John Adams, Stravinsky, and Haydn at Sanders on February 1st.
