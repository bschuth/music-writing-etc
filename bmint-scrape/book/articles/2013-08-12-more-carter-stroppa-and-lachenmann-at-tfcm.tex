\chapter{12 August 2013}

\textsc{Boxxx}

The third concert at the Tanglewood Festival of Contemporary Music on Saturday was presented as a “Prelude Concert” before the evening’s program of Beethoven, Carter and Brahms at the Shed—anyone with tickets to the later concert could attend. Attendance at the Festival has been excellent, but this coupling with the “big event” led to an even larger audience. They were given an overview of the featured composers of the Festival, roughly in order of ascending difficulty, and although there were more walkouts and there was a restiveness not sensed in other concerts, a gratifying large number made it all the way through.

The pieces here were all small scale solo or chamber works. More Carter, Stroppa and Lachenmann at TFCM presented three pieces of Elliott Carter for solo piano, Retrouvailles, Tri-Tribute and 90+. Carter’s output for solo piano is not large, and only the massive and infamously difficult Night Fantasies qualifies as a major work. Listening to these smaller pieces, I was struck by how much of what I expect from Carter’s music comes not only from the characteristic use of pitch and rhythm, but also by his exploitation of voice and timbre among players. This poses challenges when writing for the solo keyboard. In the three tiny movements of Tri-Tribute, broadly different styles in the three movements provided the necessary variation to keep the music interesting. In 90+, the music is structured around a kind of erratic ostinato, of 90 notes around which the music swirls and resolves itself. Retrouvailles, the third of a series of birthday tributes to Pierre Boulez, is a more characteristic and extended piece, depending more on the skill and musicality of the performer to make the composition’s argument. Aimard’s playing was absolutely excellent, the different voices in the music clearly delineated and characterized. Several times in Retrouvailles one would hear two voices entangled in the middle of the keyboard before scuttling off to the extremes of the instrument, and the individual lines were just as audible when they were jostling with each other as they were when they inhabited different registers. His playing was authoritative and rock-solid, with a clear attack and a presence that never hardened in loud dynamic, nor did it relent in quieter passages. These were all later pieces of Carter’s, relatively brief, leaving behind a sense of rigor and austerity as they were completed.

Marco Stroppa’s first two works in the Festival relied heavily on electronics for their realization, a reliance I have found problematic. His piano trio Ossia: Seven Strophes for a Literary Drone demonstrates the same interest in the complexity of sound and in the harmonic components of pitch, but in this piece electronics have no place. The work is in seven short movements, all of which demand extensive use of natural and artificial harmonics on the violin and cello. In addition, the string players move around on stage, taking positions behind and in front of the piano, as well as at the extreme sides of the stage. This spatial arrangement turns out to be very effective, drawing attention to the different realizations of overtones and resonance that in an electronic context might simply be manipulated and amplified. The opening movement places both string instruments behind the piano, which remains mostly silent. A long, questioning melody is played in harmonics on the cello, eventually joined by the violin. The placement of the strings and the fragility of their sound created an unusual sense of depth on stage. Their placement behind the piano seemed to dampen them somewhat, and the effect was of archaic horns being played in an unusual temperament at a great distance. Each movement explored different spatial, expressive and stylistic qualities of the sound, the strings emphasizing elements provoked by the material in the piano, which variously used thick clusters, scales that sprinted up and down the length of the keyboard, ostinati, and a catalog of attacks from the softly resonant to the violently struck. The title of the piece, which refers to the nickname of the poet Joseph Brodsky as well as to a particular poem, did not seem to have any immediate relevance to the music. This was the most satisfying work of Stroppa’s I have heard. Stroppa’s music doesn’t yet have the rigorous internal logic that we have heard in the music of Helmut Lachenmann, and so has moments where the attention wanders, where the listener must forcefully refocus attention. But the ideas are fresh and their realization engaging. Even when the strings appear invoke a cliché of electronic music (as in the occasional appearance of rapid arpeggiating clouds of harmonics), it transcends the banal by being performed on live instruments in a vibrant space.

The ushers handed out texts for Helmut Lachenmann’s GOT LOST to audience members as they came in the door. This may have raised unrealistic expectations in their minds, as this piece for voice and piano is just as relentless in its disassembly of musical sound. A good minute or two went past before the soprano uttered a clear vowel sound, much less a recognizable word. I have written extensively about my experience of Lachenmann in my other Festival reviews -- how what sounds like an entirely unpleasant experience turns into something unexpectedly gripping. The same process happened here, although the initial sense of discomfort was greater than either of the previous pieces. Although there was also a prominent part for voice in …zwei Gefühle…, but it was one element among the players of a chamber orchestra. In GOT LOST, the soprano is the primary focus, and the connection to (or disconnection from) to the text is much more profound. The bizarre, twisted fragments of language threatened to become comical or embarrassing, and for the first few minutes of the piece I found myself unable to look at either performer—Lachenmann does not let the pianist off the hook either, as he must frequently produce many of the same vocal sounds. However, the dedication, precision and, yes, musicality that Elizabeth Keusch and Stephen Drury brought to this piece made it possible to leave behind that discomfort and raise my eyes to the players. The piece retained a sense of instability; as flickers of meaning would occasionally appear, a stab of… humor? confusion? discomfiture? would shoot through me. It seems likely that Lachenmann expects the audience to find the piece funny:  those texts handed out to the audience were ludicrously diverse—a rather dark paragraph from Nietzsche’s Gay Science (“Ringing abyss and the stillness of death…you’re lost if you believe in danger”); a poem by the Portuguese author Fernando Pessoa that rings changes on “love is ridiculous”; and finally a sign posted in English in an elevator lamenting the loss of a laundry basket (hence giving the piece its name). The audience listened respectfully—perhaps too respectfully, as the one obvious joke that was audible in English (the sudden appearance of the phrase “pretty difficult” in the midst of this insanely complex music), had only a small number laughing out loud. Keusch’s performance was spellbinding, her technique impressive whether she was parceling out Germanic phonemes or singing clear ringing tones into the belly of the piano—where the resonating strings would echo back her own voice to her with uncanny verisimilitude. Drury sang and uttered and manipulated the piano (he carried a ball-peen hammer on stage for the piece) without ever drawing focus from Keusch. The piece drew a quite divided reaction from the audience, a significant portion of which fled Ozawa Hall like a shot off a shovel, while those who remained applauded vigorously and called Keusch and Drury back to the stage repeatedly.
