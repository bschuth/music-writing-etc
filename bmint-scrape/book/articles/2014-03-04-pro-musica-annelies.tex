\chapter{4 March 2014}

I approached Sunday’s Chorus pro Musica performance of James Whitbourn’s Annelies with trepidation. It is a large work for chorus, soprano,  and chamber ensemble to texts from Anne Frank’s diary. The Diary of a Young Girl was required reading for me in high school, and I have revisited it a couple of times since then; it remains a heartbreaking and moving document. I was afraid of being lectured to, or that Anne’s words would be used as emotional weapons, or to give the music a depth it hadn’t earned.

It turns out that I do take issue with the piece, but for different reasons. It is clear that the piece has a broad appeal. It has become very popular recently; Jamie Kirsch, Chorus pro Musica’s music director, said that there were three performances scheduled in Boston this season alone. He also stated that he felt the piece would enter the canon, in large part due to its accessibility. The portion of the audience that stood at its conclusion certainly felt it worth commemorating. But I left Old South Church unmoved, feeling that Annelies lacked the one thing that you might have thought it could not possibly be without: a voice.

The piece is about 80 minutes, in 14 (!) movements. With the exception of the opening “Introit” and a brief foreshadowing in the second movement, the underlying structure of the piece is neither musical nor thematic—it is more or less strictly chronological. The movements have titles: “The last night at home and arrival in the Annex,” “Devastation of the outside world,” “The hope of liberation and a spring awakening.” With a few exceptions, the texts are all from Anne; Melanie Challenger is credited with the libretto. The music is written in a pretty but anonymous style that one encounters frequently in contemporary “accessible” choral music. There are touches of Vaughan Williams (Dona Nobis Pacem), falling gestures that made me think of Gerald Finzi, and harmonic movements that recalled Michael Nyman; and the ghost of Randall Thompson never seems far away. The mood is predominantly dark and melancholy. Whitbourn takes every opportunity to sound-paint pictures when the words permit, perhaps intending to give the piece a visual, cinematic quality. (Kirsch, in his pre-concert remarks, noted that some of the piece was “movie-like,” and endorsed that as one of the elements that contributed to the work’s accessibility.) There are episodes where the writing adopts an identifiable style: the fifth movement, “Life in hiding,” includes a waltz and an echo of 1930s popular song.

But that same movement, the longest in the piece, demonstrates in the small the conundrum of the piece as a whole. It begins very quietly, with the line “The days here are very quiet.” The piano chimes in a high register, which is explained by the text “the chiming of the Westertoren clock/reassures me at night.” When the piano drops out entirely, we hear “the silence makes me so nervous.” Upon being told, “you no doubt want to hear what I think of life in hiding?,” there is a sudden transition to a kind of swooning waltz music for description of nature The movement then shudders to a halt for the only text that is spoken in the piece—oddly, the one place where any humor appears, the “Prospectus and Guide to the Secret Annex.” This passage is perhaps a parody of an advertisement for a rest home: &#8220;A Unique Facility for the Temporary Accommodation of Jews and Other Dispossessed Persons'' It is Open All Year Round, Located in Beautiful, Quiet, Wooded Surroundings'' Singing is Permissible, only Softly and After Six pm!&#8221; The appearance of bare speech is odd, the more so as the passage states ``Singing &#8220; Permissible!&#8221; Then the 1930s song appears, for description of washing in tub; and then sudden pathos, for a description of ragged children running in the cold. The movement is a collection of effects, each moment engaging and unthreatening to the ear, but they do not cohere. This sacrificing of point-of-view for scene-setting keeps Anne’s voice at arm’s length. The texts are sung at times by a soprano, and at times by the chorus, and it is not always clear why the texts are apportioned the way they are. Placing Anne’s intimate words in the mouths of a large chorus changes their impact and meaning, and switching them between soloist and chorus weakens the sense the reader has of distinctive, youthful, intelligent human being.

Kirsch’s pre-concert talk contained several anecdotes where he wrote to the composer to ask for clarification about the piece, and this points to the other major issue with the work: There are choices that seem just inexplicable and which require outside justification. The use of speech in the fifth movement mentioned above a choice that I still cannot figure out. In the eighth movement, Whitbourn uses only three small fragments of text, and the first is “Kyrie eleison.” Leaving aside any issues of theological or religious appropriateness, the text appears without preparation or explanation, it is not further elaborated on, and the music doesn’t make any case for its appearance. Kirsch said the composer explained this choice by saying the use of the text from the Catholic mass was intended to show that the catastrophe that enveloped Anne was “a universal, not exclusively Jewish, tragedy.” Again, leaving aside the sloppy use of the word “universal,” this simply makes no sense. The mere inclusion of a text cannot make such an argument, as shown by the fact that the composer had to explain it to the conductor, and the conductor to us. There are other, similar confusing choices that I will not catalog here.

Finally, as much beloved as Anne is, as preternaturally talented and thoughtful as she was, she was still a fourteen year old girl, and the texts struggle to support the weight of a piece as long as the Beethoven 9th. The quotation chosen to close the work is a lovely if confusing thought, sentimental but not profound, especially taken out of the narrative of what became of Anne: “Whenever you feel lonely or sad, try going to the loft on a beautiful day and looking at the sky. As long as you can look fearlessly at the sky, you’ll know you are pure within.”

Kirsch told us that the work of rehearsing the piece was emotionally difficult for some participants, and the performance was affecting for the commitment of the performers. The music’s broad appeal depends heavily on that accessible harmonic language which is lush and rich, and the chorus never stinted on color and texture. A few more low voices might have been wanted in the loudest parts of the piece. Soprano Lynn Eustis, who sang at the American premiere of the piece, conveyed the texts intelligently and displayed a warm voice with dark shadings. Her pleasing sound though was at odds with the girlish qualities of the text. The chorus was supported by an ensemble of piano, violin, cello and clarinet (there also exists a version for large orchestra). Despite Kirsch’s suggestion that clarinet might evoke Klezmer music, no hint of anything that exotic ever appeared; neither did the music seem to acknowledge the fact that the ensemble was identical to that of Messiaen’s Quartet for the End of Time. The Old South Church did not make the musicians’ job easier. It was quite cold, which may explain some pitch issues in the ensemble, and since the space seems to swallow up some plosive consonants, having the texts in the program was a necessity. In the quieter passages, the combination of quartet and chorus was effective; but as the music increased in volume, the chorus tended to swamp the instruments.

The piece was bookended by iterations of Stephen Paulus’ Hymn to the Eternal Flame, a short, pretty setting of a text by Michael Dennis Browne. Browne’s poem is an unaccompanied incantation that repeats the word “every:” “Every face is in you/Every voice/Every sorrow in you/Every pity…” Paulus gently shifts the text accent so that the music has a gentle pulsation that does not fall into boredom, and which subtly emphasizes with a brief melisma one single-syllable word in each of the three stanzas: “love,” “hope” and “soul.” Playing it both before and after Annelies, was an interesting conceit that didn’t quite take flight. At the end, the chorus left their risers to surround the audience in the church, and sang the Hymn from the floor. This surprised some audience members who had not realized the piece would be performed again, and who had to sit down once they understood they could not escape. It was impressive to hear the piece sung with perfect ensemble and solid pitch despite their dispersal. It was a welcome gesture of consolation.
See related interview here.
Brian Schuth graduated from Harvard with a Philosophy degree, so in lieu of a normal career he has been a clarinetist, theater director and software engineer. He currently resides in Boston after spending the last 15 years in Eastport, Maine.
Share this:EmailTweet