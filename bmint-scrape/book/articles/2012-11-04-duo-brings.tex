
\chapter{4 November 2012}

\index{Long Duo}The Long Duo, sisters Beatrice and Christina Long, played a duo-piano concert at Jordan Hall on Saturday night, under the auspices of the Foundation for Chinese Performing Arts. The sisters come from a musical Taiwanese family and have achieved a measure of renown both as a duo and as individual solo performers, appearing around the world, and with such American orchestras as the Baltimore and Dallas Symphonies.

The Duo has a well-formed aesthetic and a highly accomplished technique that nevertheless does not draw attention to itself. They play as one player; their attacks simultaneous and their long lines pass back and forth without seam. But their interpretive choices are restrained to the point of reticence. Now, if that is all you demand of music making, you cannot quibble with their playing. I found them consistently faithful to the text before them, but could not find the animating principle behind their playing, or their programming. Their considerable skills were brought to bear against a program that was puzzlingly varied, and which I found only erratically illuminating.

\index{Mozart, Wolfgang!Overture, Magic Flute}The first half was a bizarre grab bag of pieces arranged without any discernible rationale beyond that of variety. It began with Mozart's Magic Flute overture, in an unfussy arrangement by Busoni; all of the Mozart was there, but there was little sense of drama. It was curious that the adagios, which in the original rely so much on sustained strings and winds, were more effective than the skittery allegros, which got swallowed up in an unusually resonant acoustic this evening. It was accurate but un-playful.

\index{Messiaen, Olivier!Visions de l'Amen}This was followed by a single movement wrenched from Olivier Messiaen's Visions de l'Amen, turning it from a study of a particular variety of religious ecstasy into a virtuoso showpiece performed with great skill and taste. The rather frantic birdsong Messiaen writes into the piece was crystalline over the rich modal intonations of the melody. It was a pretty thing.

\index{Asai, Takeshi!Spring Thunder}Exactly why this level of playing was brought to bear on the next piece, Takeshi Asai's Spring Thunder, is unclear. The New York based composer wrote this work in 2010. The program described it as “mood music [with] jazz-toned piano meanderings.” That strikes me as exactly right.

\index{Arensky, Anton!Silhouettes, op. 24}The most successful piece in the first half was a relative unknown Silhouettes, Op. 24 of Anton Arensky. We were presented with 80\% of it; of the five movements, four were presented; no reason was given as to why the fourth was omitted. These are “characteristic” pieces, each with its own title: “The Scholar,” “The Coquette,” “The Buffoon,” “The Dancer.” They were not sophisticated renderings, but they were spot on. Our Scholar is depicted by a grave and dense melody deep in the bass end of the keyboards which then gives way to — just wait — a fugue. Arensky managed a mediocre fugue, which burned itself out quickly, but only after it had successfully painted its picture. Similarly the flighty melody in triple time that depicts Coquette nearly made me laugh out loud. All the movements successfully evoked entertaining, broad-stroked caricatures. This is not music you want to think about too much, and I doubt it could carry much interpretive burden. The Duo’s conservative approach served the movements well, executing each with aplomb and then dispatching it to move on to the next. I will be certain to seek out this piece again, if only to hear the fourth movement, “The Dreamer”, to figure out why the sisters chose to exclude it from its siblings.

\index{Bach, Johann Sebastian!Concerti for Two Keyboards,\\ BWV 1061a and 1062}\index{St. Botolph Strings}The strengths and weaknesses of the Duo were shown in relief in the second half of the program. It consisted of two Bach concerti for two keyboards: BWV 1061a in C Major and 1062 in C Minor. For these they were joined by the St. Botolph Strings, described in the program as an ensemble” of 18 top level players from the New England Conservatory of Music, coached by Lynn Chang.” Both pieces were performed without conductor. The strings were wasted on the C Major concerto, whose orchestration was perfunctory at best. The first two movements of the piece consist of the keyboards tossing music back and forth, with much imitation and a constant sixteenth-note pulse.  The last movement is a five-voice fugue which moved along quickly, if rather densely. I don't know if a more opinionated performance would have made this music more engaging. What we were given did not make a persuasive case for it.

What a surprise, then, to find the C Minor concerto thought provoking. Given musical material to work with, the St. Botolph Strings provided a new palette of colors and a youthful urgency that made every statement of that material propulsive. It is as though the normal hierarchy of the concerto had been inverted, that the solo keyboards were in fact playing a complex intellectual accompaniment to the impassioned fragmentary statements of the strings. It was not typical Bach, to be sure, and a companion referred to the performance as “relentless,” which I suppose it was. The middle Adagio Ovvero Largo moved along especially quickly, but to me, even that uncovered something interesting, giving the pizzicatos in the strings a melodic connectedness, and making the sudden arco two-thirds of the way through a real event.

Before playing the Messiaen, the duo warned the audience of the “dissonance” that was about to ensue. This was the only address they made to the audience all night. Why would they program a piece about which they felt the need to warn their audience? Why would they put out a warning on a piece whose dissonance is relatively innocuous? Visions de l'Amen is hardly Boulez or Babbitt. The Long Duo is deeply committed to its craft, but I am left feeling that they are uncertain about what they are crafting, and for whom they wish to offer it.