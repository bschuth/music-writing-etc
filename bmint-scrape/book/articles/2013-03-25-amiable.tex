\chapter{25 March 2013}

\textsc{Overbox}

“Role-Play” was the title of Sunday afternoon’s Boston Musica Viva program at the Tsai Performing Arts Center, led by Richard Pittman. While not all of the music on the program really fit that theme, the playfulness implied by the name did capture the light, capricious and often humorous quality of the music.

MIT’s Peter Child Duo for Flute and Percussion opened the concert. This is the second work of Child’s that I have encountered this month, the first being his orchestral work Jubal which was performed by the New England Philharmonic, also led by Pittman. I found that work almost overwhelming in its density; this much earlier work from 1979 was simpler and slighter. Duo is a four-movement suite that pitting the lone flute against vibraphone, temple blocks, suspended cymbal and an assortment of drums—but is mostly an agreeable get-together. The range of expression was fairly limited, the flute favoring independent lines with a great deal of leaping up and down through its range, the percussion preferring to spend most of its time on the vibraphone, with occasional outburst from the drum. The enjoyment of the piece came from the ease with which the composer got this odd couple to accommodate themselves to one another. The movements (“Prelude”, “Two Part Invention”, “Caprice” and “Fantasy”) distinguished themselves primarily through tempo and texture, “Prelude” and “Fantasy” being smoother and reflective while the “Two-Part Invention” was a spiky and the “Caprice” rapid and rhythmic. Child’s note mentions moments of “argument” in the piece—such there were, but they never threatened to disrupt the conversation. Ann Bobo’s flute was alternately glowing and incisive; Robert Schulz played the array of instruments before him with strength and confidence, but never obscuring or overwhelming the flute.

Judith Weir has amassed a large catalog of music, and has become one of the most significant figures in contemporary music in Britain. Musica Viva is familiar with her work; familiar enough to have commissioned Blue-Green Hill, the second work on the program. Written for the “Pierrot” ensemble of clarinet, flute, violin, cello and piano, it received its world premiere at this concert, and was performed twice, a practice I hope continues to gain in popularity. Weir was on hand for the event and described how Scottish folk tunes, which she had always considered “without interest”, nevertheless have intruded as she composed. Each of the three brief movements in Blue-Green Hill has what she called “gestures” from this folk music as part of its raw material. She said she was avoiding the typical compositional uses of folk song, instead using the gestures as material that could generate modern concert music. I thought I could hear the gestures in the leap and turn in the cello in the first movement, the heavy swinging rhythm in the second, and the flourishes in the third; but the music certainly did not insist on them. The music was a bit dense texturally—this worked to advantage in the first movement, which had a lush and overtly romantic feeling, with the alto flute deployed to cunning effect to secure the bass. The subsequent movements felt congested. Often the instruments were playing simultaneously—the players managed to keep their lines clear and independent, but there was a similar color to all of the movements. I found myself interested mostly in Weir’s use of rhythm, which in the second and third movements was always present without necessarily insisting on a sense of constant pulse. The third movement came to a sudden halt and then provided us with an brief enigmatic coda that still puzzled even on its second hearing. Bobo was joined by William Kirkley on clarinet, Geoffrey Burleson on piano, Gabriela Diaz on violin and Jan Mueller-Szeraws on cello.

Elliott Carter’s Double Trio was sandwiched between the two performances of the Weir, and was receiving its Boston premiere. Count me among those who have followed Carter’s late-life compositional evolution with great interest. While not losing his distinctive personality, the works of the last fifteen years or so of his life display a clarity that the more challenging works of the 1960s and 1970s rarely have. The Double Trio pits two unexpected groups of instruments against one another—piano, trumpet and cello on the one hand; violin, trombone and percussion on the other. The pleasure of this piece comes from hearing how the instruments alternately mesh and clash, the textures sometimes sweet and sometimes knotty, but always transparent. It is also source of humor, not something one always associates with Carter. In fact, the piece begins with the violin-trio intoning long lines that are interrupted with a great splutter from the trumpet group and I was instantly reminded of P.D.Q. Bach’s “Echo Sonata for Two Unfriendly Groups of Instruments.” Of course, Carter moves on from this joke, but the music is full of pleasure at what juxtapositions these instruments offer. A later moment has them passing around a single note so we can take in the various timbres of the instruments, arranged for maximum contrast. Eric Berlin on trumpet and John Faieta on trombone made the presence of brass instruments in a chamber ensemble seem absolutely ordinary and proper (except where Carter asked them to startle), their playing idiomatic and sensitive.

The Carter did not inspire outright laughter, but perhaps we in the audience have to take the blame for that. The final work on the program, Sebastian Currier’s Vocalissimus, certainly had moments that were meant to humor us and yet it was still rather silent in the room. Perhaps we’re not used to being allowed to laugh at contemporary music. Vocalissimus is suite of 18 very short movements for soprano and “Pierrot” ensemble plus percussion. It sets the very brief Wallace Stevens poem “To The Roaring Wind” 18 times, each one from a different perspective. Each movement has an evocative title: “Satirist”, “Lunatic”, “Child with Dying Mother”, “Somnabulist”. The poem, though brief, offers a free field for experimentation, being both simple, distinctive and mysterious: “What syllable are you speaking,/Vocalissimus,/In the distance of sleep?/Speak it.” The “Optimist” and “Pessimist” can twist the opening question as they wish; the “Interrogator” insistently repeats “Speak it!”; “vocalissimus” is subjected to several transformations. Each of the movements is smartly written, the flavor of each movement conveyed immediately, but without any sense of effort or straining for effect. Soprano Zorana Sadiq presented the pieces with just a slight hint of theatricality—a knowing glance, a supple gesture, enough to let us know she was in on the joke without drawing focus from the music. Currier’s music here is tonal moment-to-moment, but which threatens to fracture and recombine at any moment. This is my first experience of his works, and it is difficult to know exactly what to make of such a widely varied sampler; but I was smiling throughout, and even laughing under my breath at times. There are worse ways to spend your Sunday than being in the company of such amiable music.
