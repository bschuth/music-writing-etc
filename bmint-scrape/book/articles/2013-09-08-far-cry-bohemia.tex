\chapter{8 September 2013}

\textsc{Boxx}

The chamber orchestra collective “A Far Cry” opened its seventh season at Jordan Hall Saturday night. The “Criers” title, “Once Upon a Time,” was mystifying—none of the pieces by Gideon Klein, Joseph Suk, and Mozart had any sense of overt storytelling. From the stage a second, more apropos theme was suggested: “Bohemia,” from which both Klein and Suk came, and which was fond of Mozart. A weak premise for including the “Prague” symphony, perhaps, but workable with a stretch. The works didn’t seem to have much to say to one another, but the orchestra’s playing fully justified the evening.

The players did so brilliantly in the first half. The conductorless Criers rotate positions of leadership from piece to piece. Currently made up of 14, the collective has evolved a distinctive aural personality. The tone is athletic, rich and glossy; the execution rhythmically emphatic, and there is intense energy in everything. This is manifest in driving phrases; in body language (all the players save the cellists stand to play, including bassoons and horns when present), and also in the tendency to land on the fast side of tempi. All of these virtues were on display in the outer movements of Gideon Klein’s Partita. A Czech Jew born in 1919, Klein’s entire, brief compositional career was lived in the shadow of Nazi anti-Semitism. At 21 he had to leave the Prague Conservatory, and in late 1941 he was imprisoned at the Terezin show camp. The Partita is an arrangement for orchestra by Vojtěch Saudek of Klein’s last completed work, a trio for strings, which he completed it in 1945, at the age of 25, a week before being transported to Auschwitz. The work is in three movements: two fairly brief and driven with a wild dance character, and a longer middle one of slow variations. Klein’s language is conservative for the time. Its melodies have an Eastern European folk character, aggressive and agreeably dissonant. The sound is personal, but has obvious connections to Bartók (with the sharpest edges polished down) or Janacek (but with rhythms distinctly less obsessive). The fast movements are vivid and exciting, constantly moving, with motivic fragments coalescing and dissolving below the primary melodies. The slow movement is less successful; Klein shows some skill in handling string textures (thought we wonder how much of that was in the transcription), but the variations never quite flower. After the thrill of the last movement ends, one is left with a melancholy sense of having heard a distinctive voice, only beginning its development, but now and forever silenced.

Joseph Suk’s four-movement Serenade for Strings also draws on regional melodies, but these are stereotypically beautiful such as one finds in Dvorak, Suk’s teacher and anxious influence. If one doubted that a conductorless ensemble could play with flexibility and expression, this performance should have dispelled that concern. The Criers’ tone is brilliant and incisive, almost at odds with the broadly singing opening melody of the initial Andante, but the phrase was shaped impeccably, with a gentle swelling of dynamic and a just perceptible rubato. The ensemble was rock solid and yet yielding, with no wandering in pitch or articulation. After the initial downbeat it was impossible to detect a single leader; coordination seemed to be created moment to moment, conveyed by quick glances and body language. This is physically ravishing music. The second movement starts with an almost absurdly ingratiating triple-time tune with syncopations that swoons into a cadence; the third movement’s slow melody sounded like an operatic outtake from “Showboat” (and I mean that as praise). Suk doesn’t do much with the melodies, which are so successful they really only need to be repeated verbatim to have an effect; any but the most subtle development would risk disrupting them. Suk’s inventiveness fails in the final movement, an “Allegro giocoso” that has the right mood but banal material. The “collective leadership” of the group was vividly displayed between the Klein and Suk, as the players moved around the stage between pieces. The first and second violins swapped position, the players rearranged themselves within the sections, visually reinforcing the Criers’ stated egalitarianism.

Before the Mozart began, we were informed that we were about to see the largest ensemble that “A Far Cry” had ever assembled. In fact, according to the program there were more “Guest Criers” than actual Criers on stage, resulting in an orchestra of thirty. This sound from this relatively small ensemble filled Jordan Hall effortlessly, but unfortunately the exquisite calibration in the Suk and Klein was not on display. The first movement suffered from balance problems, the first violins struggling to be heard clearly (at one point they was nearly covered by the two bassoons). In addition, the tempi chosen for the outer movements were too fast, and the playing lacked expression. This may be a conscious choice by the group when playing Mozart: their performance of the Mozart 40th this past spring suffered from the same problem. While the rapid speeds conveyed a headlong excitement, especially in the finale, it did so at the expense of blurring subtle details. The more rapid passages in the final movement were reduced to passagework exercises, the slight adjustments Mozart makes in his figuration almost inaudible. The challenges of coordinating the ensemble were also on display, as the eyes of players were more constantly fixed on the first violins. Though on balance, this was Mozart incompletely realized, there were still pleasures to be found here, as Mozart played fast, loud and nearly perfectly is a thrill, as was watching these passionate young players bobbing and swaying as they passed the thread of the music around.
