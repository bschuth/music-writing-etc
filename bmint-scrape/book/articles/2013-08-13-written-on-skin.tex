\chapter{13 August 2013}

\textsc{Boxxx}

After all of the attention and critical praise George Benjamin’s Written on Skin has garnered since its triumphant debut in Aix-en-Provence last year, it may be time for someone to take it down a notch now that it has had its U.S. premiere. However, you are not going to get that here. After attending that premiere at the Festival of Contemporary Music at Tanglewood Monday night, I can report that it is a vivid vision of human disaster, coruscating and scouring in equal parts, based on a brilliant text set to music both brooding and iridescent.

The outline of the plot is taken from a 13th-century Provençal razo, a razo being a brief prose poem providing justification for the lyrics of a troubadour song. This genesis was a consequence of the one condition the Aix festival required for its commission that the opera be in some way connected to Provence. The bare outline of the story is simple, and typically operatic: a powerful lord, the Protector, commissions a fantastically expensive illuminated manuscript, “written on skin,” from “the Boy,” a talented illustrator. The lord’s wife and the Boy become intimate, and tragedy ensues. This skeleton may sound prosaic, but almost everything about the realization of this story is unexpected and contributes to the curious spell the piece casts. The opera begins with three Angels, who exist outside of time, calling us back to our archaic history. The English text, by playwright Martin Crimp, aspires to the condition of poetry while retaining a plain-spoken direct voice that allows the characters to speak starkly of the human condition. These opening lines are characteristic of the force of his speech:

\begin{verbatim}
Strip the cities of brick, dismantle them
Strip out the wires and cover the land with grass
Force chrome and aluminum back into the earth…
\end{verbatim}

This is no Edenic vision, however; humanity still exists, and this time before technology is further described as making way “for the wild primrose, and the slow torture of criminals.” Anachronisms are scattered throughout the piece, giving uncomfortable modern inflections to the cruelties the Protector inflicts on his subjects in order to preserve his family. The entire opera occurs within the intimate space of a powerful marriage, and the price that power exacts is never forgotten. It happens quickly, lasting slightly longer than an hour and a half; it falls into three parts, but no break was taken in this concert performance, emphasizing the headlong fall into catastrophe sketched by the action.

The Boy is played by a counter-tenor, on this occasion, Augustine Mercante, with a ringing voice whose narrow vibrato contributed color without damaging his tone’s purity. Although initially this makes the Boy sound a little unsuited to the frank erotic attention of the wife, it succeeds in bringing to the foreground her physical needs and devastating desires. She is without a doubt the driving force of the destruction to come. The consummation of their attraction is sketched in two economical but extremely effective scenes: in the first, the wife entices the boy to illuminate a page with a “real woman”, not the idealized Eve he has been creating—they sing in echoing, aching major thirds, each picking up the other’s pitches. When in a later scene the boy brings the picture to her, they intertwine with one another, until at the end of the scene (and the first Part), the woman insists on being called by her name, Agnès, and precipitates the crisis of their intimacy, telling him “Love’s not a picture; love is an act.” Soprano Lauren Snouffer brought a voice of immense dramatic flexibility to the part, a pure, bright sound that nevertheless had immense power whenever Agnès insists on the use of her name. Benjamin gives Agnès something of a mad scene at the end of part two, where she allows herself to be overwhelmed by contempt for her husband, an incendiary aria of hatred that Snouffer attacked with brio, bringing the second part to a hair-raising close.

Bass-baritone Evan Hughes played the Protector; at first I thought his appearance perhaps a bit too suave for this man of raw power, who is described as “addicted to purity and violence.” However, vocally he was commanding, with a dark-colored instrument that compelled respect. Given the restrictions of a concert performance, he was the most dynamic actor of the ensemble, able to credibly portray anger and rage. In addition, he brought a strangled poignancy to what may be the most painful “letter scene” in opera, where the words “pleasure” and “pornography” are both used to break the Protector’s heart.

Mezzo-soprano Tammy Coil and tenor Isaiah Bell had the supporting parts of the two remaining Angels; the Angels specialize in a kind of brutal deadpan, telling the story of Genesis (as if drafted by Beckett) in two dramatic duets. In addition, once the Boy has drawn the picture of the “real” woman and the Protector has seen it, he tells a lie, saying that the woman in his picture was actually Agnès sister, Marie. As he tells this lie, the Angels act it out, in a performance that manages to precisely convey insincerity without resorting to humor or parody.

The orchestra of Tanglewood Music Center Fellows was led by the composer himself. A student of Messiaen, Benjamin has as deft a hand for orchestration as his teacher and an innovative ear. The orchestra includes a handful of minor exotica – a glass harmonica, violinists doubling on mandolin, a prominent and dramatically crucial solo part of bass viola da gamba. The mood of the music is brooding and hints at the destruction to come throughout, but this is not the unrelenting expressionism of, say, Berg’s Lulu. While the vocal lines wander freely through harmonic space, they often return to familiar intervals, and characters often imitate one another when singing to one another. Benjamin sets the text idiomatically, only rarely repeating a phrase, and fitting the words to rhythms that creatively mimic everyday speech. The orchestra is large, but only rarely does any but a small subset play together. When they do assemble en masse, the effect was brilliant and overwhelming. Benjamin’s direction was crisp and clear and presumably definitive.

Whatever tendency towards excess Benjamin may have in his music is kept in check by Crimp’s text, which manages to be densely allusive and dramatically economical. Words that reappear become signposts of meaning; words like “body,” “mouth,” “protect,” and of course “skin” become touchstones of meaning.

We will have to await a proper staging, though. Here the singers gathered together in larger or smaller groups, and the lighting of each side was used to underline the temperature of the scene; on the left side of the stage the lighting was warm in reds and yellows, while on the right side, it was cool blues. They were dressed without overt attempts to characterize them, except for Mercante, who wore a sleeveless vest and a white shirt with rolled-up sleeves, perhaps attempting to emphasize the “boyness” of his part. Unfortunately, as he was much taller than Snouffer, the effect failed and instead he looked underdressed compared to his colleagues. There were moments where the lack of staging made nonsense of the lines (the “insincere” lie episode suffered especially from this), and the depth of intimacy implied by the text was simply impossible to realize in a concert performance. However, the work did not fail to make its impact. The audience rose to its feet immediately upon its completion, and the ovation went on for some time. A ringing success closed a remarkable Festival of Contemporary Music.