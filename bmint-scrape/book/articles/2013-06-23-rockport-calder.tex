\chapter{23 June 2013}

\textsc{Boxxx}

The Calder Quartet appeared on the stage of the Shalin Liu Performance Center in Rockport Thursday attired in black suits with narrow black ties, the black plastic frames on the glasses of two of the players putting me in mind of Stan Freberg for the first time in many years. This gave an air of 1950s American intellectual seriousness that suited both their thoughtful, probing performances and the acute selection of repertoire: one recent quartet by Esa-Pekka Salonen, one modernist juggernaut of Bartok’s, and what might be the last great “classical” quartet, Ravel’s.

Salonen famously left the Los Angeles Philharmonic to devote time to composing—he and Leila Josefowicz brought his dynamic and incredibly busy Violin Concerto to the BSO this past season. The Calders opened with his Homunculus, named in reference to the theory of reproduction which posited the existence of a fully-formed tiny being in each individual sperm. Salonen uses this as a metaphor, compressing an entire large-scale four-movement quartet into 15 minutes. Materials representing a scherzo, slow movement, “main” movement and chorale are introduced and then intermixed. Boulez once famously put down Messiaen by saying “he doesn’t compose, he juxtaposes.” However, as art music struggles to find a 21st-century voice in out of the ruins of serialism, juxtaposition seems as practical as any other strategy. To be able to listen attentively to demanding sounds, some method is necessary to allow the piece to teach the listener how to receive it. In Homunculus, Salonen works with four groups of material—a violent and spiky scherzo, a rustic-sounding slow melody, minimalist accompaniments in the “main” movement, and glassy unresolved chords in the chorale. The result is bracing and intriguing, and dense with incident: the scherzo’s violence acts as a sign post, reorienting us to the next round of presentation as well as providing as one extreme of a continuum of expression. At the other end are the waving moments of almost Philip Glass-like minimalism, but unlike Glass, these undulations are way stations of relative peace, the music looking ahead to decide its next direction. At this first listening, I arrived at the end not entirely sure I had followed the argument all the way through. One hopes for another opportunity to hear the piece to assess its depth and durability.

The intelligence behind the programming became evident as echoes of the next work, Bartok’s Fifth String Quartet, could be heard retrospectively, as it were, in Homunculus. Salonen’s use of repeated violent gestures to re-establish the listener has a direct equivalent in the repeated notes of Bartok’s first movement; Bartok’s radically transformed folk sensibility has its echo in Salonen’s “slow movement” melody. In both pieces the sheer volume of musical ideas threaten to overwhelm the structures into which they were placed. Of course, in Bartok’s case the scale is much greater, with a richer variety of musical material, and a wider range of developmental tactics. The Calders allowed the music slowly to build in intensity. Although certain savagery was lacking in the opening movement, there was a supreme sense of clarity, allowing the listener to track the appearance and the logic underlying the appearance of each new idea. They took advantage of Bartok’s extended demands to create a subtle palette of colors—glissandi with different resonances, differing shades of pizzicati. The overall effect was to make the quartet feel larger on the inside than on the outside, a sense that much was left to be discovered once the piece was over, without the performance being at all unclear or difficult to follow. The appreciative and attentive Rockport audience was rapt, and the odd hurdy-gurdy tune that appears near the very end garnered a surprisingly loud laugh from the audience.

The members play with a firm sense of ensemble and project a strong, shared interpretive vision. The magic of the ensemble is that they possess unique and individual voices, but have found a way to create a blended sound that loses none of that individuality. At one point in the Bartok they played a long descending scale, handing it from instrument to instrument as it went lower, and as each player took it in hand, their particular tone took over, the bright sound of first violinist Benjamin Jacobsen giving way to the darker and woodier sound of second violinist Andrew Bulbrook, then to the centered and resonant viola of Jonathan Moerschel, and finally into the passionate cello of Eric Byers. Each of the players was able to project his personality, and yet the scale itself was a single thing, an integral gesture.

After Salonen and Bartok, Ravel’s early String Quartet felt a little nostalgic. The astonishing sunset visible through the glass back wall of the Shalin Liu center during the first half had highlighted the more modern qualities of this beautiful hall; by the time Ravel came along, the sunset had given way to darkness, giving the hall a burnished-wood warmth.. The panoply of tone colors found in the Ravel was a different scheme than those of the Bartok, but they were equally wide ranging. The first movement sonata allegro was crystalline, every turn and change visible, and yet still “tres doux”; the pizzicato movement full of both energy and precision; the slow third movement another etude in subtle color.

As further proof of this group’s quality of intellect and thoughfulness, when they returned for an encore they played no chestnut or popular arrangement. Instead, they gave us a sampler of some extremely new music, the second movement of a quartet they had premiered in London at the Barbican Center in May. The brief slow movement Joby Talbot with a long aching melody and accompanying roulades was both lush and inscrutable. The piece was not an ideal standalone movement—it felt like a quiet but uncertain bridge between unseen masses—but made one want to seek out something new, rather than taking refuge in something familiar.
