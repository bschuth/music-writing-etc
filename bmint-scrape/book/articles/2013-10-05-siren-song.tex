\chapter{5 October 2013}

\textsc{Boxx}

The 17th Boston University College of Fine Arts Fringe Festival opened Friday night at the Lane-Comley Studio 210 Theater with a sold-out first performance of Jonathan Dove’s chamber opera Siren Song. A collaboration between the CFA’s Schools of Music and Theater and the BU Opera Institute, Siren Song has many virtues to recommend it: a brilliantly colorful tonal score excellently realized by a 10 player ensemble; a story that is both ludicrous and true; an intimate setting that is well suited to depict the twisted intimacies that are the subject of the piece. Siren Song will be repeated three more times—twice on Saturday and once on Sunday.Dove has an extensive body of theatrical work. His earliest stage works were “community operas,” works intended to involve large numbers of participants, especially youth. His more recent pieces include the operas The Enchanted Pig, The Adventures of Pinocchio, Mansfield Park and incidental music to a theatrical realization of Philip Pullman’s His Dark Materials. His breakthrough work, 1998’s Flight, used material also used by Steven Spielberg in The Terminal. In addition to his original compositions, he has produced a version of Wagner’s Ring cycle for 18 players to be played over two evenings, and has produced similar chamber reorchestrations for Rossini’s La Cenerentola and Janáček’s Cunning Little Vixen.

Siren Song is a relatively early work, written in 1994 when the composer was 35. It was inspired by the novel of the same name by Gordon Honeycombe, a newscaster in Great Britain whose literary corpus includes such works as Royal Wedding and  More Murders at the Black Museum. Siren Song is a fictionalization of a true story from the 1980s: a British sailor falls in love with a woman he only knows from letters and, later, through the proxy of her brother. Only after his communications with the brother are misinterpreted to suggest he is having a homosexual affair, illegal in the British Navy at that time, does he learn that she does not exist.  The deception was quite elaborate.  The sailor sent money, ultimately £18,000, to his “beloved” to build a nest egg with which they might start their lives together. Instead, the money went to pay for hotels and travel for the brother, who would promise the sailor that his “girlfriend” would meet him in distant ports.  Instead, the brother would appear with an excuse for why his sister was not able to make the trip.

Siren Song is being performed with two different casts—on Friday night, the sailor Davey was played by Christopher Hutchinson, and Jonathan, the brother, by Nickoli Strommer. In addition, Dove gives the imaginary woman a concrete existence on stage—Katrina Galka played Diana, the idealized beloved.

While the truth of the deception is not stated overtly until the opera is nearly over, the falseness of the relationship is made clear through the objectification of Diana—Galka is costumed in short dresses, lingerie and at one point, a bikini. She sings the words of her letters to Davey as she moves around him, a dance of seduction and insincerity that Davey is ill equipped to resist. Hutchinson’s performance painfully evokes the vulnerability, naïveté and neediness that allows Davey to accept this fantasy as reality. Strommer’s Jonathan feels every inch the con man from the moment he arrives on stage, his baritone voice filled with false bonhomie as he delivers bare-faced lies.  Dove knows what he is doing with voices.  Diana’s arias have a florid quality that are made more erotic by hers being the only female voice heard. Davey is frequently asked to sing in falsetto when he is most exposed; and Jonathan’s forceful lines simultaneously reassure Davey while cowing him into accepting this incredible deception.

The extent to which you will enjoy Siren Song may depend on how interesting you find the vicissitudes of the deceptive, imaginary love affair. Beyond the fascination of watching a man devote himself to someone he has never met, there is the mystery of why Jonathan chooses to enact this particular cruelty on Davey. We witness the corrosive effects of loneliness. There is a hint of homoeroticism, perhaps more than a hint in the final tableau. But much of what happens on stage is just a form of extended dramatic irony, so extended that when Davey finally learns what has been going on, the emotion felt is not one of betrayal or injury, but of relief that the truth has finally come out. Subtlety is not Dove’s strength as a dramatist; this is not a work where you will lose track of the story if you let your mind wander. This bluntness extends to the emotional content of the opera as well: the eroticism underlying Davey’s attraction is clear from the moment the tall, thin, blonde Diana appears in her red dress – but in case you missed it, there is a later scene of imaginary love—making, staged matter-of-factly on the floor, complete with beautifully sung moans and groans (and watched by another sailor, a spoken role played with a creepy intensity by Griffin Griggs). The libretto by Nick Dear is mostly unremarkable, except when it has an unfortunate tendency to fall into awkward rhyme. The opening scenes are filled with longing and erotic ache; the middle of the piece is quasi—comedic, with a surprising amount of talk about food.  Comedy in opera is a tricky thing, and the effect was uneven, with the biggest laughs coming from the use words one might not expect —“vindaloo” does sound funny when sung with an operatic voice. The final scenes strive for confrontation and devastation but haven’t been adequately prepared for the moments to land. The ridiculousness of this story would be unbearable if we didn’t know it was true, and the fact that last year a similar story played out around football player Manti Te’o gives Siren Song a certain contemporary frisson.  But ultimately not enough is at stake – it’s just a long con with an easy mark. While it’s a shame it happened, one walks away merely intrigued, at best reflecting on the emptiness in Davey and on his eagerness to fill it.

The music is “post—romantic minimalist”, to use a phrase quoted by William Lumpkin, the music director, in his program note. It recalls John Adams’ vocal writing, but without the surprises and complexities of Adams’ later work. Sondheim is another influence, but with a bit more “process” and greater vocal technique. Its overtly dramatic qualities call Britten to mind—the opening and closing “ocean” music seem to have Peter Grimes as a distant ancestor.  The score is tuneful without having any really great tunes, but is uniformly attractive and easy on the ears. Only near the end, when Davey realizes what has been done to him, does dissonance enter into the sound world, but it is merely a symbol which disappears as soon as Davey’s anger dissipates.

The performances and casting on this evening were excellent; as with last season’s Owen Wingrave, the Opera Institute offers us performers with strong acting chops as well as polished voices. Galka was fearless and courageous as Diana, Hutchinson a touching and affecting Everyman. Strommer’s had a convincing confidence man’s voice, a bit at odds with his boyish appearance. John Slack and Erik Van Heyningen played the Regulator and Captain, who discover the relationship and mistake it for a gay liaison. Slack was appealingly unpleasant as the “bad cop”, Van Heyningen appropriately bluff and inflexible as the Captain who has no objection to what people do in their private lives but who must enforce the rules.
