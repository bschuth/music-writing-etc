\chapter{13 July 2013}

\textsc{Foobar}

Pianist Gilles Vonsattel came to the Rockport Chamber Music Festival on Thursday as part of its final week of performances. The first half of his program was made up of 19th century works: Saint-Saens’ Africa, the “Moonlight Sonata” of Beethoven, and two pieces from Liszt’s L’Années de Pélerinage: “Les Jeux d’Eau à la Villa d’Este” from Book III and “Funerailles” from Book I. The second half contained two quite different 20th-century works: Heinz Holliger’s Partita and Winnsboro Cotton Mill Blues by Frederick Rzewski. Vonsattel is in his 30s, a winner of the Naumberg International Piano Competition in 2002 and an Avery Fisher Career Grant in 2008. A former student of David Deveau, the artistic director of the Rockport festival, he studied at Julliard and now teaches at UMass-Amherst.

The first half of the concert gave Vonsattel a familiar platform from which to demonstrate a strong interpretive viewpoint, which is married to immense technique and considerable power. His tone is solid without being leaden; it does not conjure up visions of light, but rather of wood and stone. It has a particular richness and impact on the lower third of the keyboard, from which he was able to produce torrents of sound that threatened to overwhelm the Shalin Liu Performing Arts center. His performance emphasized power, clarity and stability rather than overt emotion. At its best, this playing serves to induce emotion in listeners rather than displaying it to them: the last movement of the Beethoven and “Funerailles” were both devastating but delivered with only minimal expressive gestures. The famous first movement of the Beethoven was a model of Apollonian clarity. As it affords little opportunity for technical display, and as it is burdened with an immense history of previous interpretation, Vonsattel channeled his power into a sense of supple, but absolute, control over the music, not straining for any interpretive effect. There were moments where one might want to be surprised a little more—the middle movement of the Beethoven offers ingratiating possibilities that were not indulged. I have heard this movement played more humorously, or with greater warmth, but I don’t think I have ever heard it sound so impeccably constructed. Africa, which can sound quite splashy and mock-exotic in its original piano and orchestra version, was here rather threatening, Vonsattel refusing to linger to over any of its various episodes (the flatted-note “Arab” scales, the scherzo-like double-note dancing), but proceeding with a sense of inevitability. “Funerailles” started out with a sense of doom, and refused to stray far from that mood; if you wanted to spend a little more time with the Chopin-esque slow melody in the middle, for respite or for solace, you were out of luck. The funereal triumphed over all and the effect at its conclusion was crushing. Vonsattel’s technique is so prodigious I found myself wanting to hear a little more grit in it at times—the water fountains of “Jeux d’Eau” occasionally shot by so quickly and so frictionlessly that gravity seemed suspended—these were jets of water rather than arcs—beautiful nevertheless, but somehow unnatural.

Perhaps the most surprising thing about the concert was Vonsattel’s thoughtful and engaging commentary that started the second half, preparing the audience for the considerable challenge that Heinz Holliger’s Partita was to present. Having already shown in the first half that he could hold the audience rapt playing Saint-Saens, Beethoven and Liszt, he could certainly have played a second half of similar repertoire without enduring much complaint. It is encouraging that he is willing to turn his considerably gifts to this craggy and difficult tribute to Bach. He talked of his passion for this music, which he described as densely self-referential and meticulously crafted, with the realization that it poses insuperable problems for the first time listener. For example, he told us that the final movement re-develops much of the material from the previous movements, while acknowledging that you might not actually be able to hear any of that material. It was disarming and may have allowed the Partita a more indulgent reception than had it been presented merely with the program notes. Heinz Holliger is, of course, the pre-eminent oboist of his generation, but he is also a composer and conductor of some stature. His well-known interest in the Baroque certainly informs the Partita, which is roughly structured like any number of Bach’s suites or partitas, although for the most part the musical language couldn’t be more unlike Bach. The piece is in 7 movements, all of them brief excepting the last “Ciacona monoritmica,” which consumes nearly half the total duration. The movements all have strong individual personalities and clear compositional structures, allowing the listener to comprehend their mechanisms if not necessarily understand the moment to moment process of each piece. The opening “Prelude” contrasted crashing expressionist gestures with ghostly resonances of a Bach chorale, produced by the player holding down the keys of the chords, allowing the strings to vibrate sympathetically. It sounded like a vivid argument for Leonard Bernstein’s thesis presented in his Norton lectures that the spirit of tonality hovers over all music, no matter how atonal on it may appear to be. The following “Fuga” was built on a figure whose first intervals had enough character that it was not difficult to track the entrances of the voices, even when they were transformed or inverted. A very brief toccata-like “Czárdás obstine” was flanked by two “Sphinxen für Sch.”, inspired by the three short motives Schumann published in the score of Carnaval, but which are not played. The “Sphynxen” are short explorations of extended playing techniques—playing inside the piano, knocking on the body, etc.—which present, in a hidden and enigmatic way, some of the basic material of the piece. The Ciacona’s repeating ground-line was clearly perceptible during its nearly quarter-hour evolution, but as more musical material was added to the piece and it approached a critical density, it became difficult for the first time listener to hold on to the argument. Vonsattel’s performance was polished, confident, powerful and clear. While the music was often quite difficult to absorb, if you turned your ear to it, you could clearly hear Vonsattel enunciate inner voices.

The evening ended with a very different 20th-century piece, Frederick Rzewski’s Winnsboro Cotton Mill Blues. Rzewski is an American treasure, a composer of intense social consciousness who has created a body of work that is absolutely of the last century in its challenge and approach, while remaining broadly accessible—literally so, in that the composer has made many of his scores available on imslp.org. Winnsboro is a musical juggernaut emerging from the extreme depths of the piano, an increasingly loud and frightening depiction of the relentlessness of mechanized cotton mills. Eventually this initial wave breaks, leaving behind a repeated chromatic figure that is less overwhelming but no less anxious. The title of the piece refers to a work song that Pete Seeger made known in the 1930’s. The performance was preceded by Seeger’s recording, where the music serves to provide a salve for the grimness implied by the lyric. In Rzewski’s realization, when the tune finally appears on its own, it is sad, melancholy, and is shortly eaten up by a return of the relentless machines, but now played on the extreme high end of the piano, as if spinning off into space. This was perhaps an ideal work for Vorsattel, whose playing often suggests a restless, searching anxiety along with its technical astonishments and astonishing strength. All of these combined in Winnsboro for an exhausting and fulfilling end to the evening.
