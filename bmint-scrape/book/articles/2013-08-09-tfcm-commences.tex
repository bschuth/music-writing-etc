\chapter{9 August 2013}

\textsc{FOo}

The Tanglewood Festival of Contemporary Music is directed this year by Pierre-Laurent Aimard, who has said that his aim is to “bring music that is not much played in the US,” a sentiment not terribly well-defined, since most “contemporary” classical music is not much played here. In practice this summer, it means focusing on two composers little-known in America, Helmut Lachenmann and Marco Stroppa. In addition, the Festival is something of a memorial to Elliott Carter, who died last year at 103: six of Carter’s pieces are being performed, and the Saturday night concert in the Shed will include his Sound Fields.

The Festival opened Saturday night at Ozawa Hall with a Prelude concert featuring a “classic” from 1951: Carter’s First String Quartet. This is both a transitional piece and a masterpiece, one that confirmed the direction the composer’s writing was moving in after earlier forays into neoclassicism, but whose language was not fully mature. Carter himself said that it was taken up with “linear, melodic” material as opposed to his later work, which he says had “other concerns.” The melodies are strongly characterized material, often dissonant, which clash and support one another in turn; this is especially clear in the first movement, where four melodies contend throughout. A skilled and eventful piecing together of divergent musical material, the quartet is often dense with independent polyphony, producing an effect reminiscent of Ives but with a cool, reasoned surface. The piece is also filled with Carter’s “metric modulations,” changes in tempo precisely notated. Four young members of the New Fromm Players (Sarah Silver and Matthew Leslie Santana, violins; Jocelin Pan, viola; Michael Dahlberg, cello) were assembled to play the quartet, the first piece they worked on. Silver spoke briefly beforehand, her remarks suggesting a certain intimidation that had grown into respect and even love. The energetic and assured performance had a clarity that was critical for following the many simultaneous arguments. The New Fromm Players mined the piece for every nugget of expression, especially in the aggressive viola and cello outbursts in the third movement. They navigated the rhythmic landscape with confidence, the changes subtle and organic.

An hour later the full concert began, with pieces by the two featured Europeans, plus a work by Christian Mason and another Carter. Carter completed Instances in his final year, and it is yet another of the finely wrought miniatures that characterize his late work. The music moves very quickly from texture to texture, style to style – the pitch material is not conventionally melodic, but careful listening as the piece progresses readily finds connections, echoes and transformations of the melodies. One wants to avoid hunting for valedictory content in any composer’s final works, but might be forgiven for finding something haunting in the episodes that sound like chorales, in the lonely lines of the trumpet, in the figure that combines falling fourths or thirds with a strong rising fifth, in the hushed slow ending, punctuated by short, sharp isolated notes.

Born in 1984, Mason is by far the youngest composer in the Festival. We heard the world premiere of his The Years of Light, a setting of a short fragment of the poem “Lachrymae” by David Gascoyne. This brief (10-minute) piece showcases sound color. Mason arranges a chamber ensemble around the hall – in the balcony above the orchestra he places a vocalist in each corner, soprano and mezzo, paired with a trumpet. Below each pair on the floor of the stage is an E-flat clarinet. In front of them is a semicircular arc of 12 harmonica players (!),who surround an ensemble of keyboard, strings, winds and percussion, including a set of Chinese gongs, which were put to impressive use. Mason’s attraction to harmonicas is fascinating if less exciting in practice than it might seem in theory. The program notes suggest his attachment to the instrument is related to being a big fan of Bob Dylan, a connection at best tenuous. Each harmonica player has two instruments: all have a C harmonica, and each has their own chromatic one as well. Only the bottom and top notes are played, allowing the parts to be performed by any musician who can read music and do that task. The harmonicas thus deployed provide a couple of distinct flavors: en masse they have a distinctive buzz that makes them otherworldly and electric, and they are not wholly in tune, creating a fuzzy cloud of interference with other instruments. The voices and trumpets often play in unison or near-unison, giving the sung texts an oracular and hieratic quality. The musical material is relatively simple, often falling back on a pedal point on E. The piece has four parts: an atmospheric, almost pulseless introduction on the word “slow”; a more dynamic and very bright section on the word “light,” with glissandi and exotic string sounds; a last section that ends focusing on the word “unity”; and a ritual exit of the harmonica choir. This last was done while the choir was playing a clutch of loosely interlocking motives, whose effect was of a crowd of mechanical birds singing the same simple songs over and over again, something Messaien would have heard in his nightmares. The players walked out of the hall and onto the lawn, a charming if protracted effect.

Of the two featured Europeans, Lachenmann is the better-known; born in 1935, he has had a longer career and has amassed a reputation and small discography. Stroppa, born almost 25 years later, has also been very successful in Europe as measured by appointments: he served as director at France’s IRCAM, founded and directed the International Bartók Festival in Hungary, and even studied at MIT in the late 1980s on a Fulbright. Stroppa’s work this evening was Let Me Sing In Your Ear, a concerto for amplified basset horn and orchestra. In it the soloist stands on a platform behind the orchestra; the instrument has been outfitted with an exoskeleton that holds five microphones, with a sixth behind him. At the front of the orchestra next to the conductor is a stack of five loudspeakers, roughly in the shape of a basset horn, with the bottom and top units turned up toward the ceiling in something of an S shape. The soloist is amplified in various ways—during one section there is no amplification; during another the amplification is very high, so the sound of keys and pads hitting the instrument can be clearly audible. There are clear shifts in the presence of the instrument, the sound appearing to move from the back to the front of the hall as amplification is applied. The music itself is a tour de force of extended techniques for the soloist, the astonishing Michele Marelli. Much of the part is played in multiphonics, a technique whereby the player forces the column of air in the instrument to produce two tones simultaneously. In practice, this is done by using fingerings and pressure on the reed to destabilize the pure vibration of the column of air. Beginning players produce transient multiphonics by making mistakes; in the hand of a player like Marelli, these “degenerate” sounds can run the gamut from ghostly singing reminscent of Tuvan throat singers to violent shrieks and squawks. In addition to multiphonics, Marelli is called upon to do rapid tonguing, “slap” tonguing, as well as playing music of high technical difficulty. The music develops through texture and gesture, at times grandly expressionistic. Amplification is necessary for the balance between instrument and orchestra to work at all, but the presence of the huge stack of speakers next to the conductor was offputting. When the sound disappears from floating around the player and instead is thrust forward through equipment, the effect is of the machine taking over. The ambiance of the room disappears and one is subjected to the flat impact of amplified performance. This effect does not seem to be intended; the program notes do not suggest that Stroppa is interested in exploring questions of distance or separation between audience and player. For this listener, the technology intrusion was not justified by the increased range of sound possibilities, but the composer clearly feels differently. Indeed, in the ovation afterward, Stroppa, after acknowledging the players, embraced the stack of speakers.

Even those who revel in the challenge of much contemporary music might be forgiven for finding Helmut Lachenmann a hard case. I encountered his music years ago, seeking it out as something of lark—his recording of his Accanto was reviewed in Fanfare as a defacement of Mozart’s Clarinet Concerto. Being a somewhat obnoxious young clarinetist who both loved and struggled with that piece, I sought out Accanto, only to learn it wasn’t even funny or mean. It was a collection of key clicks and string scratches, with mere seconds of a recording of the Mozart interjected from time to time, to make an effect I couldn’t comprehend. Listening to Lachenmann’s “…zwei Gefühle…”, I was struck by how clearly I remembered that recording of Accanto, how complex a response it called up in me. The piece is constructed around a text of Leonardo’s, and from it the “two feelings” of the title would appear to be fear and desire. While fear is clearly present, one struggles to locate desire. The text is spoken but hugely distorted—it has been translated into German, a language filled with dense clusters of consonants, and those sounds are singled out, exposed, emphasized, separated from one another, stretched and compressed (Brian Church was the rock-solid speaker, and how one obtains the skills required to put across such a text is a mystery). In addition, two texts are often being “read” at the same time, but interleaved. I do not know if a German speaker can make heads or tails out of them as presented; the program doesn’t even bother to print the German, although an English translation was provided so one could understand the inspiration of the text if not its delivery. In large measure the sounds produced by the instrumentalists are not those they were trained to produce – breathings and scratching and grindings and percussive sounds from nonpercussive instruments. The program note contained many paragraphs of explanation; a representative sentence: “Whatever resonates is understood as twofold: a material deduced and transformed from the phonetic components and, at the same time, as sparse fragments of a traditional reservoir of affective gestures, arranged in a new way through the sonic relationship of acoustic fields, articulated variously from within, like different volcanoes which come to life or cool off.”

Such tendentiousness primed me to dismiss “…zwei Gefühle…”, and chuckles from listeners around me indicated I was not alone. However, the music works its way into the ear and brain and even soul. The sounds, seemingly arbitrary when they begin, have an internal logic that defies easy explanation but that is clearly audible. The spoken text at first seems comical, but its insistent delivery and internal integrity break down that reaction and allow the sounds to exist individually while still signifying as language fragments. The orchestra gave the impression of listening as intently and as searchingly as the audience. The music pauses at several times for long moments of silence or near-silence, and I found myself hoping it was not yet over. This sense of incomprehensible mastery can be found in many musics of the 20th century; the first time I heard Boulez’s Second Piano Sonata, I remember being struck by the force of intellect and personality behind the music, although I was easily fatigued and lost interest as I lost the argument of the piece. Lachenmann’s work resides near the same intersection of sonic freedom and control that Boulez and Cage briefly occupied together. Boulez moved off into ever more complex organizational schemes, giving each sound its own character by allowing it to assert its personality in an egalitarian theoretical construct; Cage gave up agency. Lachenmann retains as much control as Boulez, but the logic has an impulsive affective quality; while the sounds are as exotic as anything Cage might request, Lachenmann does not relinquish his role as composer.

Make no mistake, this is all the kind of thing you play to your friends who dislike contemporary music to give them a laugh; and on a recording it perhaps would still have that effect. But live performance in a hall full of dedicated musicians gives the piece interest, even dignity. The movement of sound around the orchestra was much more striking than Mason’s more theatrical spatial arrangements; and there was a sense of ritual to the proceedings that made the Mason’s harmonica procession feel a bit tawdry in retrospect. Two more pieces of Lachenmann are to come, for much-reduced forces (string quartet and voice and piano), and it will be fascinating to see how durable and bearable this aesthetic proves.

The music was performed by the New Fromm Players supplemented by Tanglewood faculty and guest artists. The level of performance and commitment was uniformly excellent, as was the direction: Stilian Kirov conducted the Mason and Carter, Ciarán McAuley the Stroppa and Stefan Asbury the Lachenmann.

U.S. premiere of Marco Stroppa's ``Let Me Sing Into Your Ear''; on Thursday night with amplified basset horn player Michele Marielli. (photo by Gabriel Scott)
