\chapter{12 March 2013}

\textsc{overfull}

I walked into Jordan Hall thinking the program for Monday night’s NEC Philharmonia concert was a little odd.  As I exited the hall, I still thought so, but no longer cared. What was on offer last night was not a themed program, but rather an evening of high-energy from a remarkable young ensemble.

It began with Haydn’s Symphony No. 82 in C Major, The Bear.  The opening movement is an example of how Haydn was capable of making great entertainment out of the slightest materials. An even earlier C major symphony (No. 38) has a first subject that is nothing more than the simplest arpeggio, and yet he spins out a good eight minutes of music from it. This symphony came along 20 years later and its opening material only is slightly more elaborate; but the uses to which it is put is even more charming. It alternates three small motives: – a rising arpeggio, the slightest fragment of a melody, and a three-beat martial figure. Beethoven may be renowned for extracting the maximum music from the sparest stuff, but with him one hears how hard it is to pull the music together.  When Haydn is doing it, one hardly notices the effort. Conductor Hugh Wolff and the Philharmonia had an excellent time making this music hit its marks—that opening movement flew forward headlong with tremendous energy. As the second movement’s theme and variations made its move into the minor, Wolff let the lower strings cut loose with drama and non-period-instrument richness; it was really quite gorgeous, if just a bit out of place. What slight lack of balance may have ensued was more than made up for when Haydn suddenly snapped out of his minor key forte seriousness back into a suddenly sweet piano in the major, a moment that caused me to laugh—not quite out loud. The symphony has its nickname from the heavy repeated notes at the beginning of the finale, based on a good-natured obsessive-repetitive fiddle like tune, which the Philharmonia attacked lustily and which they brought to a modest apotheosis at the end.

William Walton’s Concerto for Viola followed. I have not been able to warm much to Walton’s music, though I find little objectionable about it. While Haydn’s symphony made more out of modest material than expected, I find the Walton full of ideas but few that linger with me afterwards. I feel bad disparaging one of the few major pieces in the viola repertoire, especially given the deeply felt and expert performance violist Steven O. Laraia gave in it. His program biography seems too short for his level of accomplishment, noting a scholarship, a few festivals, and an award.  He has more than enough technique, but more importantly he brings personality to his viola’s distinctive deep, warm voice; Laraia’s viola playing has groundedness. I was drawn to his sound, not because of fireworks, but rather because even in the most rapid passagework, he was saying something. The solo part felt more collegial than one expects in a concerto. This was not a heroic opposition of individual against the collective—instead, a dual striving for a common goal, with the soloist merely leading the way. The orchestra seemed to lean in to hear Mr. Laraia’s playing, with dynamics that were restrained, never covering the soloist.

The Shostakovich Sixth Symphony closed the concert. It is an oddly balanced piece—its first movement is unrelievedly slow and sad and lonely, with morose solo lines floating in the air, occasionally resolving themselves into great heaving gestures, then melting away again. This is followed by a scherzo filled with startling brilliant writing for winds, and then by a galloping reckless finale. Perhaps because the Philharmonia is a student orchestra, with a makeup that changes by the year, it wasn’t quite able to convince me in the first movement; it needed more architecture to make its dark episodes coalesce. However, these people can play, and the final two movements showed off astonishing technique and interpretive unanimity.  The finale took my breath away with that bitter, almost joyless exhilaration that Shostakovich made his own: a dizzying carnival nightmare that brought the audience to their feet with whoops and hollers.