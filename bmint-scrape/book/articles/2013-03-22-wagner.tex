\chapter{22 March 2013}

\textsc{Box}

To celebrate the 200th anniversary of the composer’s birth, Daniele Gatti led the Boston Symphony in an all-Wagner program on Thursday night at Symphony Hall. Gatti's impressive Wagnerian résumé includes Parsifals at Bayreuth every year from 2008 to 2011, and he conducted the Metropolitan Opera’s production of that work just last month. Michelle DeYoung joined the orchestra for two pieces, Kundry’s Narrative from Act II of Parsifal and the latter part of the Prelude and Liebestod from Tristan und Isolde. From her long and varied resume we remember her mostly for her striking Brangäne in the Met Opera Tristan that was simulcast to theaters in 2008.

Performing excerpts of Wagner in concert has a long history, and the program notes remind us that the practice started with Wagner himself. Such concerts were public relations exercises. Since it could take months or years for Wagner to get a production mounted, presenting excerpts enabled him to get his music before audiences much sooner. There is no suggestion, though, that this was ever considered a good idea on purely artistic grounds. The results on Thursday night couldn’t make the artistic case either, being rather uneven in musical quality, albeit excellent of execution. Gatti’s strong presence and a clear aesthetic vision for this music deeply impressed when it was working at its best.

The first half of the evening was occupied with the famous orchestral excerpts from Götterdämmerung, “Dawn and Rhine Journey” and “Siegfried’s Death and Funeral March.” Gatti’s approach to all the pieces on this evening was lean, lithe, controlled, and dramatic. Every note sounded placed on purpose, the unfolding of the music inevitable. Gatti exploited pianos and pianissimos to focus concentration and to heighten the tension of the music. The very first sound the orchestra produced was hushed tones from the trombones, a sound both firm and distant, drawing attention by the quality of its quiet. Gatti’s movements were fluid and confident; a slightly bearish man, he nevertheless exuded grace and control—his physicality during his more demonstrative podium gestures reminded me of the sight of John Travolta walking in Pulp Fiction. At many of the more active moments of the music, by contrast, his gestures ceased, conveying what felt like a gesture of trust that the orchestra knew how to carry on. He conducted the entire second half from memory; I believe he may have done so in the first half as well, but I failed to note that. I do know he conducted his Met Parsifal from memory—an astonishing achievement. The high point of the evening may well have been “Siegfried’s Death.” This music moves by fits and starts, fragments separated by silences, with an obsessive repeated rhythmic motif. The silences carried the drama of the piece. Every halt was an anticipation of dread, and one waited raptly for each new entrance. At the very quiet dynamic Gatti asked for, is it difficult for the instrumentalists to enter cleanly, especially winds and brass, yet the BSO responded superbly, their precise and crisp attacks surgically piercing the looming silences they had created. At the other end of the spectrum, his fortes glowed and were never overbearing. When the music finished, I distinctly heard a “wow” behind me.

After intermission, Mr. Gatti attempted to work similar magic on the Overture to Tannhauser, but in this youthful piece, Wagner lets him down. Applying same sense of attention and care allowed him to elicit beautiful sounds, especially from the winds. The Pilgrim’s Chorus melody, which is just slightly longer than one might expect, was always sustained and its long line anchored. But the piece exhausts its material too quickly, and ends conventionally. In a way, I felt that Gatti was too good for this music. In general, his Wagner is Apollonian: precise, measured, and confident. It is no less powerful for this approach, but perhaps a bit of messier Dionysian interpretation would have helped invigorate this very early work (I think of a Bernstein recording which is thrilling, if not exactly in the best taste). But even here, he made interesting points;  I remember from this performance not the big statement of the theme in the brass early on, but rather the surprising twist in the harmony a few bars back (bar 31, if you want to hunt it down).

The Prelude to Lohengrin, which came after “Kundry’s Narrative” from Parsifal was, though, a complete success, positively shimmering and flowing in waves over the audience. By calling it “Hypnotism in music,” Nietzsche meant to disparage it, but this performance redeemed it.

This evening was about orchestral command, and alas, DeYoung was not an equal participant. “Kundry’s Narrative” sat uncomfortably among the orchestral excerpts. In this excerpt, Kundry is singing to the holy fool Parsifal, intending to “seduce” him. I add the quotes because, as with much in this work, there is something a bit odd going on. Kundry’s seduction takes the form of--;. well, in the words of the Metropolitan Opera’s summary: “Kundry, transformed into a siren, enters to woo [Parsifal] with tender memories of his childhood and mother.” Make of that wooing what you will. Like many Wagnerian arias it isn’t as self-contained as a Mozartean or Verdian number, and suffers more when taken from its context. It ends clumsily on the word “sterb” and a terminal pizzicato. DeYoung sang well, with perhaps a little steeliness in the upper register, but could not make more of the excerpt beyond a display of vocal luxury. She wore a black dress, which she exchanged for a white one to sing Isolde in the Liebestod, the evening’s finale—but her interpretation did not change so dramatically.

Fair warning: I am very attached to Tristan, but I am not fond of the evisceration that is the Prelude and Liebestod. Nietschze, in another moment of abuse which uncovers a truth, called Wagner “our greatest miniaturist in music. His wealth of colors, of half shadows, of the secrecies of dying light--;” Tristan is a five-hour domestic scene, an entire evening that explores the depths of passion and abandon. The magic of Tristan inheres not in the famous chord or in the portmanteau love-death, but in the way that a domestic entanglement succeeds in generating one of the crowning achievements of Western culture. To cut out the middle merely robs the opening of its mystery and longing and the conclusion of its sense of completion. Yet Gatti’s one overtly Dionysian impulse occurred in the Liebestod, where the erotic pulses that briefly appear made their urgent presence known, carrying away the orchestra, even to the point of covering DeYoung. Though I found the performance inevitably disappointing, I was in a clear minority, as the audience sprang to its feet at the end to voice their approval.

That audience was disappointingly sparse—people were seated all over the house, leaving numerous empty seat. The evocative silences of Gatti’s Götterdämmerung excerpts were broken not only with the expected upper-respiratory interruptions, but also with a general furniture-moving rumble. But I’m also fairly sure that the “wow” after the Götterdämmerung excerpts came from one of numerous young concert goers who surrounded me.

Gatti’s interpretations are muscular, disciplined, restrained and well considered. I am anxious to hear more from him. In the next BSO concert he conducts the Third Symphony of Gustav Mahler; it will be interesting to hear how his intelligence and force of personality is applied to that unwieldy work, and how his interpretive principles will adapt to that composer’s sound world.
