
\chapter{1 April 2013}

\textsc{Outof}

Violinist Miriam Fried came to Calderwood Hall at the Gardner Museum as a performer and as the program director of the Stearns Music Institute, the conservatory department of the Ravinia summer festival. She had with her four young musicians from the Institute with whom she gave works by Beethoven, Earl Kim, Puccini, and Faure.

Beethoven’s early String Trio Op. 9, No.1 opened the program. Composed in 1797 or 1798, it is full of turbulence that turns to sweetness and back again. While ably and beautifully accompanied by Ayane Kozasa on viola and Karen Ouzounian on cello, Fried was clearly the authority, conjuring a Beethovenian irascible joy.  Fried had her way with Beethoven, dominating the performance right from the introduction, where her dark tone threatened to out-viola the viola. In the finale, she went for broke, the music note-perfect but sounding on the edge of control, with a biting bow that put an edge on every attack, and Kosaza and  Ouzounian were right there with her, if not quite reaching her level of abandon—though  Ouzounian brought a distinct edge of wildness to the scherzo.

The Beethoven was followed by a gorgeous performance of Three Poems in French by Earl Kim (1920-1998). Kim taught at Harvard from 1967 until 1990, and produced quite a number of vocal works, often to texts by Samuel Beckett. Three Poems in French emotionally concentrates and compresses poems by Verlaine (En Sourdine, Colloque Sentimental) and Baudelaire (Recuillement for soprano and string quartet. All three songs had been previously set by Debussy (and En Sourdine by Faure), and the music inherits a French character without imitating the other settings. Kim’s language in this piece is unabashedly romantic, even post-romantic—unafraid of dissonance without suggesting expressionism. Imagine Zemlinsky and Debussy mixed together and distilled, producing an alternately lush and astringent harmony while maintaining clarity and air. Angelo Xiang Yu on violin and soprano Deborah Selig joined the performers named above. The vocal lines are attractive, with conventionally melodic profiles, grounding the songs, while strings work farther afield harmonically. The pieces have distant nostalgic feeling, as of the recollection of a long-ago passion. There is little sentimentality. En Sourdine finds two lovers beneath a tree; as it sways in a breeze, the strings alternate between two notes sensuously; when the poet then talks of their despair while the nightingale sings, the accompaniment empties out leaving behind only a high pianissimo trill in the violin and an ominous ostinato in the cello. Colloque Sentimental locates another pair, elderly, walking through a “lonesome and icy park,” with high tremolos and pizzicatos evoking the weather. They have a Beckettesque exchange (“Do you remember our long-ago ecstasy?” “Why would you want me to remember it?”) which Kim evoked journalistically, without unnecessary underlining. The effects are not merely theatrical; in fact, Recuillement may be too beautiful for its text. Even as it describes an “evil multitude of mortals,/ under the whip of Pleasure”, “the dead Years… leaning /over the sky’s balconies,” “smiling Regret”, “the dying Sun”, the scene is more elegiac than terrifying. The song instead aims for the end where the “gentle Night” pulses into silence. This performance was ravishing.  Selig filled the Gardner’s concert cube with a warm tone that was never overbearing. The quartet played with stunning control and transparency. The presenters could have done better preparing the audience for the music; the program notes contained no discussion of any of the pieces, and lacked any biographical information on Kim; in addition, the provided texts omitted the critical last stanza of En Sourdine.

Puccini’s Crisantemi (Chrysanthemums) for string quartet came next. The five-minute, one-movement elegy did not benefit from following the Kim, sounding slight and morose, despite the skill and musicianship brought to bear.

For the closer, Piano Quartet No1 in C minor, Op.15, pianist Rafael Skorka joined Yu,  Kozasa and Ouzounian. These young players all come with impressive biographies, filled with the names of well-known teachers and institutions, and with lengthy lists of awards and appearances. It is no surprise then that their performances were technically impressive, shot through with masterful moments. The third movement has the cello joined by the viola and then the violin to state the opening motive in unison. The effect was of a single instrument that simply grew ever so slightly louder, just barely changing timbre. I think also of the middle section of the second movement: the piano danced about in a triple rhythm while the strings played in harmony, sounding like a spectral, ghostly salon orchestra drifting around the hall. The finale came to fine big finish, with all involved “sticking their landings.” It was pleasing and rousing, but there was little to think about once it had ended.

The Calderwood Hall hasn’t yet grown on me. On the floor, I felt a bit like I was in a surgical amphitheater, what with the pinpoint third-degree lights above my head and the rows of staring people above and around me. However, from my seat the sound was absolutely superb—the balance was perfect, the sound faithful, showing off the skills of these young players to excellent effect.
