\chapter{3 March 2013}

\textsc{overfull}

I don’t know why the symphonies of Ralph Vaughan Williams seem to have passed out of favor.  I became familiar with them through recordings, but live performances are rare indeed. They would be ideal pieces for large orchestras seeking to expand their repertoire while not losing a skittish subscriber base. A top-notch musical craftsman who was at the center of England’s revived musical tradition for decades, Vaughan Williams has a distinctive voice which is beautiful and eloquent, and in an easily accessible idiom. One of my most hard-bitten modernist friends, who has little stomach for people like Britten and Shostakovich, said that Vaughan Williams “knew what he was doing with music.”  All of the symphonies speak with this common voice, but they run a wide expressive gamut: While the first, “A Sea Symphony”, requires a chorus and is perhaps too expensive and ungainly, the remaining eight are fascinatingly varied: there is English folk-revival romanticism in the second, “A London Symphony”, pastoral beauty and calm in the third and fifth, anger and bitterness in the fourth and sixth, late-career experimentation with orchestration in the eighth, a valedictory ninth (written when the composer was in his mid-eighties). There’s even one interesting programmatic oddity, the seventh, subtitled “Symphonia Antarctica”, composed from materials Vaughan Williams created for the film Scott of the Antarctic.  The New England Philharmonic provided a rare chance to hear one of these undeservedly neglected works, presenting a passionate and convincing performance of the Vaughan Williams Sixth Symphony on Saturday at the BU Tsai Performing Arts Center.

The sixth drives forward with unrelenting intensity for three of its four movements, followed by a suddenly static and enigmatic epilogue in pianissimo. All four movements are played without pause, leaving the listener no time to synthesize the depth of what he has heard until the piece concludes. The opening movement begins with unambiguous conflict. Vaughan Williams sets chords a half-step apart clashing, heralding the introduction of musical material which is dark, filled with anger, barely suppressed rage. One can still hear the same characteristic melodic contours that are found in Vaughan Williams most meltingly beautiful music; here, they feel like a signal of betrayal, of something lost. The piece was written between 1944 and 1947, and it is hard not to see Vaughan Williams’ reacting to the horror of World War II in this symphony. The outbursts of the first movement precede a second ominous “rat-a-tat” rhythm in the second. After a bitter scherzo, which contains a weirdly sensual tenor saxophone part in the trio, the music suddenly stops its headlong rush and evaporates into an epilogue of melodic fragments that slowly resolve themselves into a conclusion, all played very quietly.

Conductor Richard Pittman had a firm hand on the piece’s excesses without diminishing its passion. His movements were spare and measured—usually just one hand was beating time, but you could see his head moving constantly, directing the orchestra by the force of his gaze. One of the dangers in Vaughan Williams is mid-range congestion; there is a lot of music here, often with multiple subsidiary lines moving under the primary melody, often played at a high level of volume. Pittman and the orchestra kept the musical arguments clear and beautifully shaped.  Playing in more exposed areas, especially in the difficult last movement, disclosed some roughness in the strings, of both attack and intonation, and overall a feeling of tentativeness.  I might also have wanted to hear a slightly greater difference in tempi in the slow movements, which felt like they moved on too quickly. But, when playing in full ensemble, the sound was quite ravishing, and the symphony made its full impact. There was an audible “wow” from behind me when the last note disappeared into the silence, a critical assessment with which I fully agree.

The program was to have included a second major work, a premiere of a violin concerto by Bernard Hoffer written for Danielle Maddon. Unfortunately Maddon was taken ill and was unable to play (the piece will be scheduled into next year’s season). The remainder of the program consisted of two brief recent pieces. Both composers were present for the performance, and spoke to the audience about their pieces before they were played. The first, Vigil by Michael Gilbertson, was the winner of the New England Philharmonic’s Call for Scores. Vigil was written by Mr. Gilbertson when he was 19; now studying for an advanced degree at Yale, he is still youthful enough to get a healthy laugh when, speaking to the audience, he deprecatingly said “I, too, was once a young man.” Vigil is a reference to Russian Orthodox vespers, of the kind that inspired Rachmaninoff to write his famous piece. The first of the three sections three sections starts out promisingly, with exotic colors and a mournful melody. The melody is blurred and diffused as it makes its way around the orchestra, often accompanied by chords with tart “wrong-note” voicings. The second section, said by the composer to portray the effects of an “intense religious vision,” struck me as rather less successful, becoming more boisterous as it proceeds, with a few film-music clichés thrown in as to a predictable climax. The third movement quiets down, but ends suddenly without resolving the tensions that have been built up. Mr. Gilbertson clearly has a gift for orchestral sound and color, in which Vigil perhaps overindulges —perfectly reasonable in a piece by a talented 19-year-old. The New England Philharmonic gave it a full-throated realization. In the absence of the Hoffer, the orchestra played Vigil twice, a practice always appreciated when hearing a piece for the first time.

I might, however, have rather heard Peter Child’s Jubal  twice—not because of any flaws with Vigil, but because in Jubal the musical material came at me so furiously that I found myself a bit overwhelmed. Mr. Child, who is a professor of music at MIT, provided a lot of background to the naming of the piece, which is drawn from Genesis 4:21, where Jubal is the originator of music, and from a fragment of a poem by John Dryden.  While this was interesting, I found it less useful as a listener’s guide than Mr. Child’s pre-performance description of the music as seeking “a heightened emotional rhetoric” involving “vivid contrasts of material.” At its most involving, the piece was like watching a crazed race where the participants are weaving in and out behind each other in a local chaos that was nevertheless in the service of a deliberate and controlled sense of forward progress.  Jubal cast shadows forwards and back over the other two pieces of the program. There were orchestral colors that were eerily similar to those used in Vigil—Jubal’s end recalled Vigil’s beginning.  In addition, the overall plan of the piece was identical to that of the Vaughan Williams: it is a compressed symphony in four movements (fast-slow-fast-slow) played without pause, the third movement of which was a wicked scherzo.  These correspondences are a reminder that the Philharmonic’s reputation is built on its programming, which was really this evening’s triumph, providing more than enough to think about despite the loss of the concerto.

