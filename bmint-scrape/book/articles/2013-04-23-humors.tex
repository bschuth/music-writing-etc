\chapter{23 April 2013}

\textsc{Outbox}

Boston Baroque’s program of Haydn at Jordan Hall Saturday night gave two late-period works: the Symphony No. 102 and the “Lord Nelson” Mass in an apposite pairing following a difficult week in Boston. The programming allowed Haydn a not-quite-realized-chance both to amuse and terrify us through these performances.

The climate of the symphony is sunny and goodnatured, with jokes both subtle and coarse. After some initial uncertainty in the entrances in the Adagio that opened the first movement, the orchestra played with polish and characteristic tone. While Haydn’s notes by themselves do most of the work, there was a narrowness in the interpretation that led to missed opportunities. Conductor Martin Pearlman favored predictable phrasing, where motives grew softer as they concluded, robbing some material of its urgency. The repeated notes in the first subject of the Allegro faded away before they could register, lacking the weight and articulation. The peak of the opening line of the Menuetto was a waystation to the last note rather than the culmination of the phrase.  This gave an effect of slight fatigue where the music should be gathering energy. The final movement shook itself free, dancing vigorously, even frenetically. From his podium remarks and program notes, it appears Pearlman understands the joy and fun in this music. He made sure the hesitation jokes at the end landed, as well as surprising fake-recapitulation in the first movement. But most of the performance was understated, even withholding.

Perhaps there was some method to this approach. After intermission the interpretive restraint of the symphony gave way to the onslaught of sound and fury that was the Kyrie of the “Lord Nelson” Mass. Coming after the gentility of the symphony, this gave it an additional dimension of shock and surprise. Haydn’s original subtitle for the Mass was “Mass in troubled times” (Missa in angustiis), and it begins in the midst of turmoil. It was really quite breathtaking: the chorus cried out for mercy with passion while soprano Mary Wilson tore through with lines that had a touch of coloratura to them, a cry of desperation and anger. In fact, the singers were uniformly excellent: mezzo-soprano Abigail Fischer has a voice that reminds me of Kathleen Ferrier, but a bit firmer and gathered; tenor Keith Jameson had a fine, sensitive tone used mostly as a foil to the other voices; and bass Kevin Deas filled the room with a profound and resonant “Qui tollis” in the Credo. This music was much more passionate and unbuttoned than the symphony; but it too suffered from its own narrowness. The movements following the Kyrie are, with the exception of the Benedictus, much brighter and less tortured, but the intensity of the first movement cast a shadow forward over the remainder of the performance, threatening to turn Haydn's later celebrations into bombast. The one moment of rest and respite in the piece comes in the Credo at Et incarnatus est, as the soprano and chorus commemorate the fact of God made Man, just before they invoke the crucifixion, but Wilson wasn't given much chance to show what she can do with this more tender, less dramatic music. It moved quickly, with little inflection, and the moment was over almost before it started.

For those who were not bothered as I was by the interpretive choices, there was much to enjoy, and the characteristic sound of the period orchestra was gorgeous, especially in the brass. The chorus was uniformly excellent. The jokes in the final movement of the symphony elicited actual laughter, and the high emotional pitch of the Mass provided a measure of catharsis for a community that needed some.
