\chapter{30 July 2013}

\textsc{Boxxx}

It’s probably just as well that Andris Nelsons was unable to make it to Tanglewood Saturday night to conduct the Verdi Requiem with the Boston Symphony Orchestra. While we no longer fret about the ecclesiastical appropriateness of this opera dressed in religious clothing, it still occupies an anxious place straddling genres. Unmoored from character or dramatic situation, Verdi’s music nevertheless spins out powerful emotions with only the most general grounding. It is a showpiece, a warhorse, a piece guaranteed to please a good portion of the crowd unless it is grossly miscarried. While Nelsons surely would have put his own imprint on it, we may do better getting our first taste of him as the BSO’s music director in October, in a program of Wagner, Brahms and Mozart.

Instead, Carlos Montanaro came in on short notice to take the helm. Montanaro’s biography consists almost entirely of Italian opera, often performed in central and Eastern Europe – he has recently been named Music Director at the Teatr Wielki in Warsaw. In his hands, the Requiem did not linger – the “Dies Irae” made its first appearance at a tempo that was on the upper edge of suitability, and may have been just a touch faster each time it returned. The interpretation was athletic, with moments of crisis and terror, though the impression was of strength and prowess, not eschatological inevitability. While Montanaro did not forego rubato, he used it sparingly, and there was often a sense of pressing forward rather than relaxation. This had the benefit of making the few moments of calm (“Recordare, Jesu Pie”, “Lacrymosa”) particularly effective. There may have been some expressive sacrifice here to help the “Dies Irae” sequence cohere dramatically. Unfortunately, the subsequent movements did not have the same sense of unity, and there were moments of ensemble uncertainty, and even some disagreement with the soloists over tempo. The brass and percussion made a glorious sound throughout – the strings sounded a bit dry and the winds recessed in the Shed.

But apart from the heavy-duty brass, the Requiem is less about the orchestra than the voices, and the soloists and chorus were uniformly fine. The soloists were a remarkably varied group—although the Requiem lacks “characters”, the qualities of the solo voices were distinct and individual. Andris Nelsons’ wife, soprano Kristīne Opolais, has a lush and beautiful tone, with softness around the edges, trading dramatic incisiveness for a sensual touch. She showed remarkable control in high registers—the few times Verdi sent her to the top of her range her sound never became harsh. It appears we may get to see a lot of Opolais in Boston, and I imagine the public will be quite taken with her; although appearing in a pitch-black dress suited the text, she was anything but funereal.  Mezzo-soprano Lioba Braun’s voice was compact and had a dense color, while bringing an aching quality to her slow solos. Braun has sung Isolde with Nelsons, which seemed a bit surprising, as she did not have a particular big sound on this evening. She was frequently covered by the orchestra and by the male soloists, but displayed great musicality when the music quieted down. The soprano/mezzo duo in the “Recordare” was the interpretive high point of the evening, their voices blending without losing individuality. By contrast tenor Dmytro Popov has an almost absurdly loud voice, and a robust theatricality; he threatened to steal focus every time he opened his mouth. He held nothing back in his first entrance, a “Kyrie eleison” that in Verdi’s hands sounds like a call to battle rather than a plea for mercy. This over-the-top entrance always threatens to make me smile.  Popov hit it with such intensity that it felt like he was daring me to make fun of it, and I didn’t have the courage to do so. Bass Eric Owens (another substitution, as Ferruccio Furlanetto was ill) had a pleasantly reedy and resonant instrument, and delivered his part with dignity.

The Tanglewood chorus under John Oliver was in excellent shape. It was their concentration and focus that allowed the rapid tempo of the “Dies Irae” to retain its intensity. In the tricky double-chorus of the “Sanctus”, which moved at a similarly rapid speed, the individual lines remained audible. They provided a solid foundation for an uneven night of music.
