\chapter{26 November 2013}

\textsc{Boxxx}

The Radius Ensemble’s Saturday night concert at Longy used a familiar programming template: a concert with a title (“Fusion”), comprising a couple pleasant obscurities, a new work (a commissioned world premiere, in fact) and a big familiar piece at the end. Informal speeches from the stage with a gently pedagogical air were also part of the experience. It suggests a strategy of “outreach”, which appears successful to gauge by number of people at Pickman Concert Hall, and as well as by the audience’s good-natured laughter at the slightest provocation: at a display of microtones, at an apology for the sharp wind that evening, at the description of a movement as a polka. The main event, the Dvořák String Quintet in A Major, Op. 81, was preceded by a trio by Eugene Bozza, a horn solo by David Amram, and a world premiere of a piece by Berklee associate professor Jonathan Bailey Holland.

Bozza’s Suite Breve for reed trio, a slight set of four movements written in 1947, received a fair degree of laughter itself.  A lesser figure in the early and mid-20th century French compositional scene, Bozza shared a chromatic, polytonal approach to harmony found also in Milhaud, Francaix and Poulenc. Written for the sharply differentiated voices of flute, clarinet and bassoon, the Suite is a curious mix of “not quite”: not quite a study in counterpoint, despite the imitative techniques in the rather languorous slow movements; not quite a comedy, as sophistication and buffoonery battle to a draw; not quite a virtuoso display despite severe technical demands, including asking the oboe and clarinet to leap to the far boundaries of their range while playing cantabile lines.  Radius Artistic Director Jennifer Montbach played oboe, joined by clarinetist Eran Egozy and Joshua Baker on bassoon. The instrumentalists played with confidence and keen sense of conversation. The performance was genteel, with sharp edges polished away, leaving a piece that provoked mirth without ever acquiring wit.

Horn player Anne Howarth followed with David Amram’s Blues and Variations for Monk for solo horn. Amram belongs to a strain of American composers who straddled genres between jazz and classical.  Stylistically itinerant, he wrote soundtracks, concert pieces, and sat in with jazz players with his French horn. Amram was a friend of Thelonious Monk, with whom he often played informally, frequently improvising over a 12-bar blues. Written not long after Monk’s death in 1982, the Blues and Variations has the quality of an etude, a set of written out “improvisations” or reactions to the harmonic structure of the blues. Howarth suggested in her comments that one might hear the piece with ears for jazz or for classical music, but most of the music had an abstract quality that defeated the ability to hear it as jazz in the absence of any of the usual support for the blues. The few times it indulged in overtly “jazzy” figuration, the gestures stuck out awkwardly, threatening to hijack the piece by the strength of their allusion. Howarth played with a warm but authoritative tone, and with a sense of intelligence and lively attention to the structure of the music, so that one could follow the implied progressions underneath Amram’s often extended procedures.

Jonathan Bailey Holland has plenty of local and New England connections, holding a Ph.D. from Harvard, teaching at Berklee and the Vermont College of Fine Arts.  The Clarity of Cold Air, written for flute, clarinet, violin, cello, piano and percussion, was commissioned by Radius and received its premiere. The composer states that the title preceded the writing of the music, and the result suits that title without being programmatic. It has a bright, familiar chilliness to it, filled with sustained notes in the high registers of the instruments. It wanders slowly and quietly from one pole of tonal stability to another, transitioning from one to another through clouds of notes, often colored with microtones. Holland has a sure hand for dissonance: just as the densest ice is the clearest, it was the moments most filled with pitches that had the greatest sense of transparency and articulation; by contrast, the resting moments of stronger tonality sounded bland by comparison, so that one looked eagerly to the next sequence. The variety of experience evoked was that of watching clouds cast shadows on snow on an overcast winter day; if one paid attention, one could observer a constantly changing world constructed from a limited palette. Sarah Brady (flute), Charles Dimmick (violin), Miriam Bolkosky (cello), Sarah Bob (piano), Aaron Trant (percussion) and Eran Egozy produced a gorgeous, coldly glowing sound.

The Dvořák quintet is justifiably famous, the regular foot-tapping on the back of my chair by the lady behind me attesting to its direct appeal, as did the head-bobbing of several audience members during the third-movement Furiant. The performance was unsettled. Perhaps this might be laid to the fact that only two of the players were actual Radius artists (pianist Bob and cellist Bolkosky); the other players were credited as “guest artists”. There were moments of grace and beauty side by side with moments of interpretive conflict. The ensemble fortes in the opening of the first movement had a gritty tone the first time through, though this sorted itself out at the repeat. Dimmick and Shaw Pong Liu, the violinists, often seemed at odds tonally. However, violist Wenting Kang provided a strong center whenever she was called upon to play the melody. Yet another impressive protégé of Kim Kashkashian at NEC, Kang played with a pure, muscular tone—passionate without ever losing a sense of control. The players did have an impressive shared rubato, with a sensuous ebb and flow in both the opening movement and in the tempo-shifting second movement dumka, perhaps the most successful section. The final movement, the polka that amused the audience when they were told of it, provoked both chuckles at the start and loud applause and even some shouting at the end.
