\chapter{23 January 2013}

\textsc{Undefrull vbox?}

An unusual and thought-provoking concert of improvised music took place on Sunday afternoon at the Loring-Greenough house in Jamaica Plain. Mimi Rabson’s String Theory trio is made up of Mimi Rabson and Alisha Raven on five-string violins (giving the instrument the lower range of a viola), and Junko Fujiwara, cello.

Improvisation has an unstable place in Western music making. It is a game, a way to allow for free play in the moment within a set of rules. In popular music or traditional jazz that means handing over a chorus or two of fixed chords to a soloist, who is expected to exercise their ingenuity by creating their own off-the-cuff material within that harmonic space. This is such a commonplace in commercial music that it is easy to forget get that improvisation was frequently a part of Western art music as well, though the rules used to structure the experimentation were different. Bach’s ability to improvise fugues used conventions of counterpoint as its framework; the child Mozart’s ability to improvise arias in French or Italian style depended on conventions of melodic and harmonic complexity and structure. It is fair to say that many of us who remain dedicated to “classical” music no longer expect, or perhaps even desire, improvisation to be part of our experience. More common is a sense of devotion to a musical text, abetted by a desire for greater and greater control from the composer, from Mahler’s extended explanatory notes to the serialized control of all aspects of musical sound of some music from the past century. When we do encounter improvisation, it is often either merely “aleatoric” music used to create a mood or texture (think Penderecki, Lutoslawski, Xenakis); or is employed as a small part of a larger process (i.e. Terry Riley’s In C, or the fragments that Steve Reich asks players to pick out from the structures created from his shifting rhythmic motives).

The pieces presented on this afternoon attempted to reclaim the variety of improvisational possibility that was available to Bach and Mozart, busting out of the “solo break” prison into which most musical improvisation is now locked. Improvisation took the foreground. The pieces used a variety of strategies to make room and rules for real-time creativity. Some had quite a lot of ink on the page, specifying rhythms and general form, but few or no note-heads. Others had more text than musical notation, and one consisted of nothing but written directions. One, composed around a poem, had what appeared to be figures and shapes drawn in different colors.

Assessing these pieces presents the reviewer with something of a conundrum, as the usual categories of analysis don’t apply very well. The specific rules for the various pieces were not spelled out to the audience ahead of time, so was not possible to match one’s guess at what the rules would provoke with the actual outcome of the music; neither was it possible to assess the potential of those rules. Since the pieces are meant to be vessels that provoke music making rather than fully thought-out compositions, one struggles to find critical purchase on the music presented.

Ten short movements were presented over a little more than an hour. They ran a gamut of forms: slow atmospheric sustained harmonies with melody floating above it; faster, nearly chaotic movements where pulse was obscured and material flew by from player to player; ostinatos coming into and out of focus (one piece was described as expressing a cellist’s rage at the monotony of the Pachelbel canon).  Several of the pieces had multiple clearly defined sections through which material passed -- sometimes the passage was linear, sometimes there was a sense of return to a previous state. The forms were idiosyncratic, sui generis. While this may have afforded more freedom to experiment with form, it sacrifices some of the listener’s ability to easily perceive the structure of the music.


The players brought intensity and passion to their realizations. There were moments of great excitement, where you could sense an idea being grasped and exchanged and modified. These moments were fleeting and resist rendering here in text – but they were the essence of the creative act being enacted before us. One piece in particular, “Flashback”, especially caught my ear – but afforded a glimpse at the page, it may ironically be the composed interlocking rhythms in that piece that engaged me rather than the material being improvised.

The composer Lukas Foss attempted a similar project in the late 1950s; he captured the elusive quality of this music in an article entitled “Improvisation versus Composition” in the 1962 Musical Times:

Improvisation is not composition.  It relates to composition much in the way a sketch relates to the finished work of art.  But is not the very element of incompleteness, of the merely intimated, the momentarily beheld, the barely experienced what attracts us in the sketch?  It is work in progress…. it is a spontaneous, sketch-like and—incidentally—un-repeatable expression, full of surprises for the listener and for the performer as well… Viewed in terms of a composed piece, improvised music remains ‘on the way’, a mere hint, raw-material— ‘exposed’ rather than composed.  And so it should be.  That is the virtue and that is the limitation of improvisation.

Foss captures the evanescent pleasures of this form of performance, the fleeting lifetime of the creation and its spontaneity is what held your interest. Sunday afternoon was perhaps lacking in enough “surprises for the listener.” This may be due to the homogeneity of the string trio, which at times made it difficult to follow individual contributions. One piece (“Swoop”) was described as originally written for violin, trombone and double-bass – you could imagine how those wildly different timbres might help you follow the interplay between the players. The melodic material generally hewed to tonal expression, albeit with generous dissonances; but this did not vary much from piece to piece. They all inhabited the same bluesy/Bartok-y harmonic space, with not enough structure to push against. But the greatest surprise is that this music is happening at all, helping to carve out a space in a musical culture that can focus excessively on technicality and on the virtues of textural fidelity.

The Trio continues their improvisational project on February 22 at 8:00 p.m. at the Church of the Advent, at the corner of Mount Vernon and Brimmer Streets.
