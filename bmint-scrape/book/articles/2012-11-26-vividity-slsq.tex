
\chapter{26 November 2012}

The St. Lawrence String Quartet brought vivid interpretive imagination to pieces by Haydn, Golijov and Beethoven Sunday afternoon at the Concord Academy Performing Arts Center as part of the Concord Chamber Music series. The quartet has been playing for twenty years; its current personnel are Geoff Nuttall and Scott St. John, violins; Lesley Robertson, viola; and Christopher Costanza, cello.

One of my most memorable concert experiences was hearing the St. Lawrence play in New York several years back, when they also includeed a piece of Golijov’s, The Dreams and Prayers of Isaac the Blind. I was struck by the force of the players’ collective personality and dynamism. I was eagerly anticipating seeing them again this weekend, and they did not disappoint. This is what live classical performance should be — a display of passion, conviction and intelligence. It was also interesting. The program opened with Haydn’s Op. 76, No. 2, which carries the subtitle “Fifths”. The Op. 76 quartets are among Haydn’s last works, written after the composer had escaped the provincial Esterhazy court and become a prominent figure in London and Vienna. The piece is astonishingly inventive — the first movement uses its simple initial material (a pair of descending fifths in D minor) to create an unorthodox sonata form that is both closely argued and filled with almost perverse variety. All of the movements share a tendency to surprise; there are sudden outbursts in the second movement’s theme and variations, and the fourth movement decides nearly at the last minute to finish in D major. The third movement’s scherzo is a two-voice strict canon (!).

The St. Lawrence violinists alternate playing the first and second parts; Nuttall took the lead in the Haydn; St. John played first in the other two pieces. Nuttall is the most physically demonstrative member of the quartet, and he made excellent use of both his playing and his movement to bring to life each twist and turn of the score. His feet left the floor with regularity, as his center of gravity moved in sympathy with the shifts of mood and texture (as he chose to wear red shoes, with red plaid socks, making his feet unavoidable, I have to interpret his footwork as an integral part of his performance; I thought he did an excellent job). The foursome did not shy away from strong choices: repeated passages of double-stopped quarter notes in the trio got a little heavier and draggier and grittier each time we encountered them, as if the sheer sonic weight of those sounds were so enjoyable to conjure that it was worth wallowing in them a little; the final figure in the fourth movement had a “snap” glissando put into it which smacked just a little of Texas competition fiddling. Mr. Nuttall made a virtue of variety in dynamic, attack, phrasing and tone, opting for a particularly aggressive sound in the outer movements, but finding a rich, roundness to start the second. There were other choices I wasn’t so sure about — the minuet went so quickly that the oddness of the canon was elided, for example — but even those choices were interesting.

Osvaldo Golijov should be is familiar to readers of this publication. Having lived in the area for some time, he is currently teaching at Holy Cross, and has been the subject of no little attention in the last decade with recordings on Deutsche Grammophon and Nonesuch. His music frequently combines Jewish themes (both musical and extra-musical) with a primarily tonal language that draws from musics of many cultures. The composer was present, and gave a brief but entertaining introduction to the second piece on the program, his Qoholet, Hebrew for Ecclesiastes. Golijov called nameless author “the Jerusalem Thoreau — but a little less optimistic.” The piece is in two movements, ostensibly about “what stays, and what changes.” The first gives the first violin an aria of sorts to play against a motoric rhythmic accompaniment; the second is a meditation that the composer likened to the flowing water of a river, with tone-painting in tremolos, sustained notes and antiphonic pulsing that successfully conveyed its aquatic imagery without becoming intrusive. The piece was attractive enough, but did not make a strong impression. The excellent program notes by Steven Ledbetter state that “there was talk of a three-movement work, but as it stands now, it contains two movements”; it does sound as if it is awaiting further development.

After intermission the Quartet played Beethoven’s first Razumovsky quartet, Op. 59, No. 1. While as full of surprises as the Haydn, it is certainly a structurally more ambitious piece, and the interpretation was suitably different, emphasizing the architecture of the first and third movements especially, while not losing any detail of dynamic or attack. St. John’s leadership was more nuanced, perhaps a little less risky (he wears black shoes). The Adagio of the Beethoven was a respite. In a program filled with tension and intensity, I found myself needing the moments of repose afforded here. This was monumental Beethoven, impressively scaled.

The quartet offered an encore after the Beethoven, the Scherzo from Dvořák’s Op. 105. It was an excellent summary of the afternoon the rhythmic opening sharp and opinionated, the trio a moment of repose — that is, until that moment where the lines are suddenly independent, a moment I had not really noticed before, but which here was an exciting and unexpected moment of controlled chaos.

There are certainly some compromises made by the St. Lawrence to afford their kind of interpretive freedom. The sound can occasionally be harsh, and there were moments where pitch was a little uncertain. The acoustics of the Concord Performing Arts Center did not do the Quartet a lot of favors in this regard. It is small enough that no notes were lost and projection was not an issue, but the sound is unreverberant and upper harmonics almost non-existent, giving very little cushion to the sound. But for me, these are quibbles. I crave the enthusiasm and engagement the St. Lawrence provides. You can argue over this music.
