\chapter{12 August 2013}

\textsc{Boxxx}

The penultimate concert of the Tanglewood Festival of Contemporary Music was unusual for a couple of reasons. First, it started at 10:00 a.m. on a Sunday morning, which made for a bright, sunny but chilly start after a cool night; second, after a steady diet of Carter, Marco Stroppa and Helmut Lachenmann, the program had a very different emphasis. Stroppa was still present, but there was also the early String Quartet No. 1, “Metamorphoses nocturnales”, by Gyorgy Ligeti; and music by Americans Conlon Nancarrow and Steve Reich. The odd concert time did not seem to hurt attendance; the largest audience of the Festival so far was there, with the main floor of Ozawa Hall filling nearly to capacity, with a handful of others out on the lawn.

Marco Stroppa has been a featured composer at the Festival, and the one whose works have been most uneven and frustrating. His solo cello piece Ay, There’s The Rub, like the previous concert’s Ossia, continued the exploration of the components and resonances of sound inspired by electronics but using only a wholly acoustic instrument. I would like to say I enjoyed this piece more, but perhaps Stroppa and I simply don’t share his fascination with interesting sounds that do not end up cohering in a larger structure. Stroppa asks the cello to play with a dissonant scordatura, retuning the strings from C-G-D-A to Bb-G-C\#-A. The cello plays a large number of harmonics, and with this retuning unique intonations and beating effects become audible. However, these effects are not connected to any compelling sense of musical architecture or drama, and one begins to tires of bars filled with frantic vibratos, glissandi and sudden clusters of rapidly articulated sixteenths. As with Ossia, the overtly literary titling seems unnecessary; the program notes tell of a melody meant to represent Hamlet’s father, but even with this piece of information the connection to the quote that makes up the title is unclear, and while there is certainly plenty of “rubbing” involved in the playing that hardly explains the reference. By the end, the melody is transformed into what the notes call “seagull cries” and it ends with a note of authentic, but disconnected, melancholy. It was played by BSO cellist Mickey Katz, who then announced the additional of an additional work for solo cello, a selection form Henri Dutilleux’s “Three Strophs sur le Nom de Paul Sacher”. Typical of Dutilleux, the piece is elliptical and compressed; fragments of expressionist melody alternate with obsessive drumming. Katz’ playing was precise and pointed; both pieces emphasize unusual sounds and sudden changes in direction, and Katz was never anything but secure, focused, and ready to pounce on what the composer presented.

Ligeti’s first string quartet is in a single movement with anywhere from 15 to 18 separate episodes, depending on how you want to count them. It is pathologically restless and stylistically promiscuous, although at heart it is an homage to the mid- and late-period Bartok which for the most part uses a non-tonal harmonic language that finds unification through often violent rhythmic gestures and repetition of chromatic gestures. About halfway through an unprepared for V-I cadence simply appears, and then is promptly forgotten, as if a piece of music needed to have at least one of them and he decided to get it out of the way as soon as he thought about it. The piece might almost be humorous if there were less desperation in its ceaseless activity. Written when the composer just 30 and had not left Hungary, it is a monument to his home country’s greatest composer, whose most advanced works were not permitted to be performed. The rapid-fire changes in style make the work more accessible than that of Bartok—if you don’t like the current episode, just wait a minute—and lend it a theatricality that makes up for any greater sense of unification. The quartet was performed by Matthew Vera and Thomas Hofmann, violin; Adrienne Hochman, viola; Francesca McNeeley, cello; with tremendous passion and aplomb and here is as good a place as any to register the consistently super-human technique and ensemble playing displayed by the performers at the Festival. Pieces that would appear to be nearly unplayable on the page have been played not merely well, but with a polish and with sensitivity to the needs of each individual composer, no matter how bizarre or unlikely that composer’s aesthetic may be. In their hands, the most challenging works of musical modernism might find new champions and, depending on the durability of the writing, new audiences. It’s a spectacle that is both reassuring and hopeful, while being just slightly terrifying; there is something uncanny about music with these demands being dispatched by the relative youths on stage.

Among the music that was once considered to be “unplayable” was the entire oeuvre of Conlon Nancarrow, whose music has had a significant impact on composers while remaining mostly unknown to audiences. Nancarrow wrote almost all of his music for player piano, famously punching holes by hand in piano rolls while living in self-imposed exile in Mexico. Nancarrow’s musical interest was primarily rhythmic, and punching the holes himself allowed him to engage in rhythmic experimentation that seemed to beggar notation, much less live performance. A recording of Nancarrow’s own player piano became available in 1969, but Nancarrow was not pleased with all of it. A complete recording of all his studies became available in 1989, but is known mostly to new music specialists. Some of the music makes demands that still would frustrate performance in concert pieces that unleash extended torrents of notes played at superhuman speed, and those that experiment with rhythms with irrational relationships based on e or other numbers. However, a cottage industry in Nancarrow arrangements has arisen – Boston clarinetist Evan Ziporyn has arranged them for the Bang on a Can ensemble, as has the group UK contemporary music group Icebreaker. On this occasion, arrangements of Study No. 6 and 7 for two pianos by Thomas Ades were performed by Katherine Dowling and Nicholas Namoradze. Study No. 6 takes a sort-of ostinato (it bends and stretches) and lays a melodies on top of it – the material has a hazy Spanish/tango-ish feeling, and for all of its rhythmic sophistication it also has a kind of swing to it. Study No. 7 is a much more challenging work that starts with a line of eighth notes over which an unstable melody is chained; the music develops through progressive distortions and recastings of material. Lines drop in and drop out, show up in slightly different rhythmic disguises and ratios, and ends with manic flurries of scalar patterns to end in a sudden upward rush of scales in three different tempi over a hiccupping figure of thirds in triplets. Dowling and Namoradze’s performance was spectacular, remaining in perfect synchronization while playing rhythmic lines that were precisely uncoordinated. The first study remained wonderfully off-kilter – you might want to think of it as a drunk dance, except no drunk could be so virtuosically off-center. Having real musicians play this music leads to some differences from the player piano versions that are a little hard to get used to. Nancarrow’s instrument struck all notes with the same rather hard and square attack, regardless of their speed, and this lends a weirdly electronic quality to the roulades of scales he was so fond of. Dowling and Namoradze play these lines like the trained professionals they are, linking the notes together and making a smooth gesture. It is more musical, more beautiful, to be sure; but it does take some of the craziness out of the performance. One’s response to this may vary based on how much interest you can take purely in rhythm, as Nancarrow’s melodic material is often undistinguished. I have always found the music fascinating, but my wife found the experience akin to listening to pianists practicing late at night at music school, a memory she did not wish to revisit.

The major work on the program was Steve Reich’s hour-long Music for 18 Musicians, played in the actual event by a more humane 20 players. The instrumentation is diverse, but typical for Reich: a panoply of mallet instruments and other percussion, four pianos, two clarinets, a violin and cello, and female voices. The entire ensemble is amplified for balance. This piece marks a critical shift in Reich’s early style, which found an entire aesthetic in the interlocking of simple tonal motives. In his previous works the transparency and clarity of the process that generated the music was paramount; in Music for 18 Musicians the music expands in expressive scope and lushness of orchestration. It is on the same time scale as his Drumming, but much less severe and didactic. The piece opens with a cycle of 11 “chords” played as a constant pulse of rapid eighth notes. I place “chords” in quotes as they perform no harmonic work when placed next to each other; they outline a collection of pitches. These are then used to create independent sections which flow one into the other, always constructing a base texture out of the same interlocking figures as in his earlier work, but overlaid with long tones, contrasting figures and pulsing notes. Part of the attraction (or repulsion) the work exercises comes from a certain trance-like quality to the music – but listening to full hour-long work requires an attention closer to meditation than trance. The ability to note and register small changes in the music holds one’s interest, and gives the piece its sense of evolution, and progress. There are also large-scale changes where the color of the work changes gradually or dramatically. About halfway through the music is carried mostly by the interlocking playing of four pianos in the middle to lower range of the keyboard, a texture of dark colors and indistinct attacks played within pitch areas that give the music as much dissonance as it ever entertains. This suddenly gives way to a new, much brighter chord, orchestrated with bright mallet instruments and the first appearance of maracas. To me, this generates as striking a “sunrise” as you find in Daphnis and Chloe, and engenders a kind of restrained and protracted joy. There exist several recordings of this piece, but seeing the music live gives it a profoundly communal and physical dimension I had never imagined. The physicality is both strenuous, in the way the mallets fly through the air, driven by arms of the percussionists; and reserved, in the way the female voices lean into and away from their microphones to create slow swells of sound. The piece is played without conductor, and the notation allows for varying repetitions of segments. The transitions between sections are signaled by which ever player happens to be responsible for leading at the time. Players drop in and drop out as the score evolves, often trading places (twice we had the “changing of the maracas”, as one player picked up a pair and brought it up to the microphone as the current player slowly dropped them down). Moving through the piece required attention from all players on whoever was leading, and required a constant group listening. This gave a sense of communal creation to the piece that had as much in common with playing together in a band as it does with traditional playing that is bound to a non-negotiable base text. When the circle of chords reappears at the end of the piece, there was a tremendous feeling of having arrived after a long journey together. Now that this piece is almost 40 years old, I would have assumed most of us knew how we felt about this music, but to judge by the number of audience members who ended up walking out during the performance, some during the final recapitulation of chords when the piece was almost through, Reich still has a way to go to be recognized by the classical audience. To be fair, this was by far the longest concert of the Festival, clocking in at two hours and twenty minutes, and perhaps there was a simple problem of endurance for some. However, much like the previous night’s reception of Lachenmann’s GOT LOST, those that remained gave a spirited ovation, and it was the first standing ovation of the Festival that succeeded in engaging the whole house.

Ed. Note: The reference to the encore was changed.