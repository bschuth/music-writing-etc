\chapter{10 August 2013}

\textsc{Box}

Since significant attention and concentration is needed merely to understand the outline of a new piece, it can be difficult to provide an intelligent assessment. If the processes used by the composer are especially challenging or unusual, snagging a few adjectives that capture a singular aspect can feel like success. Placing the work in context or making any thoughtful critical judgments may be nearly impossible. One is thus challenged, stretched and grateful as one approaches each new concert at the Tanglewood Festival of Contemporary Music, where several works by featured composers Helmut Lachenmann, Marco Stroppa and Elliott Carter are being presented over several days. Friday saw the second concert of the Festival, containing chamber works by each of the featured composers.
We need to talk about Helmut Lachenmann.  This music bristles with odd noises. The scores are impenetrable—from a 10 foot distance they look like Western music, but any closer and you begin to see the squiggles and curves, most of the note heads modified in one way or another to signify some odd sound production or note modification. Even for a trained musician, there’s no way to match the ink on the page to any sound you might be able to conjure in your head. But now I have consumed almost an hour of these sound arrangements (one feels odd calling it “music” to anyone who has not heard it, so much is noise part of it) and have not yet been bored or irritated by it. A young Tanglewood student next to me said “He really knows how to keep the ball in the air,” and that’s as concise a description of the skill behind this music. There is something freeing in the extremity of this music, in the fact you must abandon most of the listening skills you usually employ to be able to hear it. One might listen to pitches flying every which way in post-Boulez serialist piece and know you are missing something. You also suspect there is some underlying theory behind such seeming chaos even though the music is inscrutable. In Lachenmann, there is no presumed theory to fall back on. All of the components of music function without reference to our previous experience. A given pitch does not appear to play a role in any harmonic or melodic scheme, but is merely a point of departure, a cause for argument or disagreement, to be picked up and transformed or merely opposed. There is no constant pulse, though from moment to moment there are rhythmic gestures, from the implied rhythm in the fibrillations of tremolo to the fleeting pulse implied by a chain of notes of equal duration. The first minutes of either of these pieces leaves the listener in momentary confusion, and often small tremors of laughter could be heard from individuals in the hall. But once your ears accommodate, much like walking into bright light from a dark room, the underlying dramatic unfolding of the work takes hold, and somehow this collection of “anti-musical” sounds grabs your attention. On Friday’s concert the Festival presented Grido, Lachenmann’s third work for string quartet. Grido is a swirl of glissandi, a collection of grindings and groanings, with occasional pure tones slicing through the texture. It was occasionally noisy, but frequently hovered at the very edge of audibility. Could you actually hear the sound of the bow on the scroll of the cello? It is restless, constantly changing, and immersive. Though the JACK Quartet played the piece with confidence and aplomb, the piece requires a sui generis virtuosity which was not exactly effortless. The burden of concentration was evident on the faces of the performers, who nevertheless seemed completely assured.

Marco Stroppa’s second work was a large scale piece for amplified piano and tape called Traietorria (trajectory). While hearing a second piece of Lachenmann’s confirmed that this unique music worked consistently and with different instrumentation, this work of Stroppa’s functioned mostly as a signpost for how far the composer has come. Traietorria is an early work from 1989, written when the composer was 30. Concert music with tape or computerized components has many pitfalls. The first is that most computerized or electronic music has aged poorly. Works from the 1960s and 1970s for tape often sound just as dated as popular music from that period. Stroppa’s electronic effort isn’t quite that out of date, but its pure tones and crunchy attacks, the swirls of notes and bass thuds, can’t help but sound like highbrow planetarium music. The interesting ideas in the music are frustrated by the cliché present in their surface arrangement. The electronic music is played from four speakers at each corner of the hall, offering an “immersive” experience. Sitting behind the composer during the piece, you could see him manipulating the levels of the output to the speakers, subtly varying the sonic picture. At its best you could hear elements of the music move into and out of the foreground, helping to comprehend the polyphony; at its worst, it reminded me of 1950s stereophonic demo records, where sounds ping from one speaker to another. On stage, Pierre-Laurent Aimard (the Director of the Festival) sat at the piano and played material that had enough technical demands to be interesting dramatically, but whose content was simply not enough to hold attention for the more than 40-minute span of this piece. Of the pieces  three parts, the final one was the most interesting, but perhaps only because it was the most active and most bombastic. Aimard played with great attention and technical mastery, but was left sitting motionless at the piano for the last five minutes, as the computer played on. I doubt Stroppa finds any discomfort or irony in this picture of one of the most accomplished pianists of his generation sitting in on stage while waiting for a machine to bring this art work to a close. He hugged the stack of speakers in the previous evening’s concert, but it reminded me of other discomfiting moments in similar technological music where human beings are made subordinate to the electronic device. Traeitorria struck me as the one dissatisfying work so far in the Festival, although it did make clear the refinement and growth evident Stroppa’s work from Thursday night, Let Me Sing Into Your Ear. While I’m still ambivalent about the 10-foot “sonic hologram” that towered over the conductor during that piece, the use of electronics was much more intimately tied to the production of music by live performers, and the language of Let Me Sing was tremendously varied and engaging; it makes one wonder where the arc of Stroppa’s development may end up in 10 years, when his love for electronics and technology will have even more time to evolve.
The afternoon opened with the U.S premiere of Elliott Carter’s final work. Dedicated to Pierre-Laurent Aimard, the piano trio entitled Epigrams comprised 12 extremely brief movements. The last movements were written in September of 2012, a month before Carter died, so there has been some minimal editing to make the texts performable. The previous evening’s Instances provoked me to consider if there was some sense of farewell in this last music; Epigrams raises that suggestion again, even more vividly. Carter’s music is often described as “mercurial”, capturing the sense of a fleet mind moving from idea to idea with rapidity and alacrity. That same sense of movement is retained in Epigrams, but this music keeps stopping and starting, suggesting a sense of fatigue, of being short of breath. The music frequently moved from flurry to near stasis, with long extended notes that moved only slightly. Isolated sharp eighth-notes become a motif, something also heard in Instances. The final movement of Epigrams might be taken as emblematic – it starts with a snapped eighth note in the violin; then flurries of sixteenths in various groupings alternate with sparer episodes; first there are a few bars of tremolo chords, then interlocking eighth-note triplets with only one instrument playing at a time, and the final chilling episode, 4 bars (almost 15 seconds) in which only three isolated eighth-notes are played occurs in a vast field of silence. This all happens in a movement of about 75 seconds. The effect is of music trying to rouse itself, and only succeeding fitfully. The piece was played by its dedicatee, Aimard, with Sarah Silver and Michael Dahlberg, two members of the 2013 New Fromm players. As has been the case throughout the Festival, their level of technical mastery was formidable, and the challenges of the score were smartly dispatched.

\textsc{Horizontal Line Here}

After two days of contemporary music one might look forward to a more typical classical concert at the Shed as a kind of vacation, a chance to luxuriate in the familiar. The program was indeed well-known, with Gil Shaham playing the Sibelius Violin Concerto, followed by the Brahms Second Symphony, all under Christoph von Dohnányi. The torrential rains that had started the night before had departed by concert time, but there was still a humid miasma, and one could see the fog rolling over the ridges towards us. Perhaps it was the heavy and wet air that made the Sibelius sound small and underpowered. The soloist’s movement suggested the powerful impulses present in the music, but what reached our ears did not match the gestures. Shaham spent quite some time retuning after the first movement, further suggesting that the conditions were causing him some frustration. A single clap rang out from the audience when he finished tuning, provoking him to give a gesture that suggested thanks and perhaps a request for forbearance. The sound recovered somewhat for the Brahms, now that the orchestra no longer needed to worry about covering the solo violin, and we were treated to a solid if unsurprising performance, a familiarity in which one couldn’t quite luxuriate.