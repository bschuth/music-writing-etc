\chapter{23 April 2013}

\textsc{Outbox}

A surfeit of excellent pianists associated with the music department at Boston University put together “PianistX8,” a grab bag of works for multiple pianos and piano players for Alea III Contemporary Music Ensemble yesterday. One can slice this program in a number of ways: there was one piece for piano four-hands, four duo piano pieces, and two pieces for four pianos – one with one pianist per instrument, and one with two. The works fell into two broad categories: those that were written for multiple pianos, and those that were arrangements of other pieces. There were crowd-pleasers (Piazzola, Gershwin), unfamiliar works (Feldman, Antoniou) and “classics” (Stravinsky, Copland, Lutosławski). For the most part, whether the pieces were weighty or lightweight, the evening was characterized by technical mastery and a profound sense of mutual listening.

Witold Lutosławski’s Variations on a Theme by Paganini takes the famous melody from twenty-fourth caprice and subjects it to a relentless headlong series of reinventions for piano duo. The melody stays audible throughout while the harmonic language is profligately, exhiliratingly dissonant. It is an anxious, dramatic curtain-raiser and was played with precision and verve by Yoojin Lee and Joo Young Moon.

It was news to me that Copland’s Billy the Kid was originally premiered in its piano duo version, played this evening by Victor Cayres and Thomas Weaver.  Played with sensitivity and acute attention to the coloristic possibilities of the piano, the piece offered up very different perspectives. “The Open Prairie” may have lacked the sense of immense vistas that the orchestra provides, but in its place was a sense of deep resonance; tweak the tuning of a few of the more frequent intervals and you might start thinking of the restful moments LaMonte Young’s Well-Tuned Piano.  With the percussive quality of the piano substituting for instrumental color, a certain nostalgia was stripped from the piece, bringing rhythm to the foreground.  The performance gave a fresh sound to a familiar work.

Morton Feldman’s Piece for 4 Pianos also explored resonance, the quiet sounds from the instruments quietly overlapping one another, recalling gentle waves at lake side, slowly subsiding. The performance (by Cayers and Weaver, with Bing Shen and Pei-yeh Tsai) was exact and beautiful, the players matching each other’s low dynamic and plain tone to make a complex but gossamer texture. As the music evaporated, it unfortunately disclosed the weakness of the CFA Concert Hall as a venue. It has an adequate acoustic, but in addition to having the visual ambience of an abandoned Ikea, it is poorly insulated from outside sounds. As the Feldman drew attention to the quiet at the heart of the music, the room was littered with road noise, muffled chatter from hallways, and musicians in practice rooms.

The program does not tell me where the duo piano arrangement of Gershwin’s An American In Paris came from, and I have not previously encountered it. To my ear, the arrangement draws attention to the genius of Gershwin’s orchestration, as this version is frequently muddled and lacks the piquancy and personality that animates the full orchestral version. Some of this may need to be laid at the feet of Alexia Mouzà, a player of undeniable strength and power with a steely tone who played with little sense of balance, frequently swamping Leon Berndorf, her partner. Her rhythmic playing was also unidiomatic, “swinging” notes uncomfortably at best.

The Rite of Spring has several pianistic incarnations—as a pianola roll, as a piano duo, and as performed this evening, for four-hands. KwanSeop Shim and Bing Shen gave just the first half with great technical polish and energy but an interpretive sameness. That is too bad—the Rite would seem to be full of erotic potential as a four-hand piece, what with its sensuous violence and the frequent arm-crossings. A note in the program states that this is partly a commemoration of the centenary of the first performance of the Rite, and that there will be a concert of Stravinsky presented in November—perhaps then we will get both halves of the piece.

Anna Arazi and Pei-yeh Tsai had great fun with Astor Piazzolla’s Fuga Y Misterio, a brief essay in tango, moving through three short episodes, fast-slow-fast, and ending with a theatrical flourish. Finally, Cayres, Weaver, Lee, Moon, Shen, Shim, Arazi and Tsai came on stage for Three Portraits for Eight Pianists on Four Pianos by Theodore Antoniou, Alea III’s music director. The piece was apparently written for a total of twelve pianists on six pianos, but was played in this “reduction.” The Portraits are of three good friends of the composer, who, judging from the music, are all quite formidable and imposing personalities. There is a similarity to the three pieces, a darkness and density. The first movement begins with fragments that build in irascibility over intermittent ostinatos; the second is built out of low clusters that gather into great tremolos echoed between the instrument; the third starts with a galloping rhythm and ends with culminating crashes moving across the stage as if from some 1950s LP showing off stereophonic recording. There was a dusky shimmer in the sound as it filled the hall with  an exotic and sumptuous reverberation.
