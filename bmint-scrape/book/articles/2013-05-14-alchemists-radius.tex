\chapter{14 May 2013}

\textsc{Boxxx}

“Alchemy” from the Radius Ensemble at Pickman Hall Saturday was less a celebration of transformation than a study in what transformation might be, and how composers may struggle with it

The opening work was a curious transformation indeed, with some reverse alchemy at play. Mozart’s string duo in B-flat, K. 424, was written to help Michael Haydn fill out a set of duos under deadline. The music is pleasing enough, with occasional virtuosic diversions, but is not especially inspired Mozart. The program notes suggest that Mozart consciously suppressed some of the more distinctive elements of his personality to allow the piece to pass as Haydn’s and not his own. A persuasive and polished performance by violinist Katherine Winterstein and cellist Miriam Bolkosky made the most of this minor work Originally written for violin and viola, it was performed in Werner Rainer’s arrangement for violin and cello, but with some altered passages restored—Bolkosky occasionally traveled to some farther regions of the fingerboard than might have been typical for the period, which gave some varied color to the experience.

John Harbison wrote his Wind Quintet in 1978, and he seems to have been of an ambiguous cast of mind with regard to the ensemble. The program notes quote Harbison as saying a wind quintet “is not a naturally felicitous combination of instruments,” and that he approached the piece with an intention to “present things clearly… emphasiz(ing) mixtures and doublings and maintain[ing] a classically simple surface.” The resulting impression is of a tentative exploration of possibilities. The work is in five movements, the first four of which have titles that are traditionally secondary: Intrada, Intermezzo, Romanza and Scherzo. A Finale completes the set. The doublings and simplicity the composer is striving for come across as a great deal of playing in rhythmic unison with subtle shadings of timbre—subtle enough to risk losing interest over time. The Intrada alternates chorales and exclamations; the Intermezzo develops a straining horn melody; the Romanza is a romance in form not spirit, wearing its heart anywhere but on its sleeve. The work begins to loosen up with the spasmodic Scherzo, which begins to tilt towards dance; and the Finale, with its rhythmic repetitions, flutter-tonguing and shrill fanfares finally cracks a smile. The piece received an appropriately cool and polished performance from Sarah Brady, flute; Jennifer Montbach, oboe; Eran Egozy, clarinet; Janet Polk, bassoon; and Anne Howarth, horn.

For me, the alchemy in John Morrison’s Lonesome Whistle was that it held my interest and piqued my curiosity, despite being written for solo flute, an instrument that I do not usually respond to. Played in a darkened hall, it hovers at the edge of audibility. Much of the music is produced by extended techniques, mostly slapping of the keys to produce a quietly percussive ghostly pitch, or “whistle tone”, produced by blowing very gently into the instrument to produce harmonics. It is a meditation on Hank Williams song by the same title, which is notable for Williams’ onomatopoeic emulation of a train whistle, bending the “o” in “lonesome”. The music’s connection to the song is oblique, with only a few full-breathed moments that glancingly evoke intervals from the song. It is Williams from beyond the grave, inhabiting an afterworld like Achilles in the Odyssey, still somehow present but reduced to “lording it over the dead.” Sarah Brady’s performance was compelling, never drawing attention to the technical demands of the piece while fully realizing their unearthly qualities.

The final work on the program was Christine Southworth’s Jamu for ensemble and Balinese gamelan. The Radius Ensemble was joined by the Gamelan Galak Tika from MIT, and the piece was conducted by their leader, Evan Ziporyn (Southworth is the general manager of the group). In three movements, “Kind Song”, “Fibonacci” and “Monkey Steals His Paint”, the work combines the gamelan instruments, based on a non-Western pentatonic scale, with violin, cello, bass clarinet and piano. “Kind Song” had the cello playing long improvisational lines over the violin playing rapid pizzicato—it had an improvisational, prepatory feeling, like a doina in klezmer music. The next two movements are based on a rhythmic pulse—the composer says she wanted it to be “dance music”—within which the gamelan plays interlocking figures. The Western instruments support this core in various ways—the bass clarinet typically secures the bottom, while the violin and cello twine lines above. The surface of the music was attractive, but I found myself searching for something more to hold on to, and not succeeding. This may be due to some trouble with the balance between the gamelan and the other instruments—the gamelan and bass clarinet occupied the right side of the stage, while the string instruments were on the left. I was seated on the right, and frequently could not make out what the strings were playing. Perhaps there were interactions there that would have further engaged my interest, above and beyond the physical pleasure induced by the shimmering  tintinnabulations of the gamelan.
