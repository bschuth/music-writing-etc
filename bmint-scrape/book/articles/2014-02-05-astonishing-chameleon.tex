\chapter{5 February 2014}

The Chameleon Arts Ensemble assembles interesting programs with fragrantly poetic themes—among this season’s offerings are “a vision so composed,” “where light and shade repose”, and this past weekend’s “melodies, at a distance heard” (lower-case sic throughout). The arch quality of the title, with its rhetorical inversion, invokes an indistinct impressionism, but happily the concert itself was a marvel of emotional and technical intensity. There was very little “distance” from this music, which, if anything, pressed against the listener. The program was performed twice this weekend; I heard it at the Goethe-Institut Boston on Sunday.

The theme came from the direct and indirect influences that earlier music had on the pieces we heard, through direct quotation, encoded inspiration, and dogged emulation. Pianist Vivian Choi and cellist Rafael Popper-Keizer began the afternoon with an early set of 12 variations by Beethoven (WoO 45) on “See the conqu’ring hero comes” by Handel. Written in Berlin at the same time as the first two cello sonatas, the piece is fascinatingly transitional. Handel’s melody was some 50 years old and was a favorite both of Beethoven and of the Prussian King Friedrich Wilhelm II. It remains popular, of course: those who have listened to children working their ways through Suzuki Book 2 will know the theme better than they realize. Many of Beethoven’s variations have some similarities to the variations in the final movement Clarinet Trio, Op. 11, written a year or so later: this is most notable in those variations that are particularly piano-heavy, with the cello providing obbligato outlining of chords and motives here and there; or in those variations that seem primarily figural. But then there are places where one looks forward to the Beethoven who could create worlds out of unlikely fragments—the eighth variation takes Handel’s A-B-A major-minor-major theme and turns it into a intense storm with a hushed chorale at its center, a miniature tone poem. The eleventh spins out long intricately varied quasi-cadenza for piano, which the cello then joins with its own operatic outpouring. Choi and Popper-Keizer played with the abandon and intensity that would characterize the entire afternoon, Popper-Keizer finding a bright edge to his sound that permitted Choi to indulge her impressive forte without covering the cello.

The second piece was another piano-and-string instrument set of variations, but emanating from quite differing centuries: Benjamin Britten’s 1948 Lachrymae for viola and piano was based on music by John Dowland from 1605. Pianist Choi was paired with violist Scott Woolweaver. The primary melody, based on Dowland’s “If my complaints could passions move”, only appears at the end. The preceding episodes are constituted from fragments of Dowland, embodied initially in inchoate tremolos which develop structure and dramatic shape as they evolve; it is like the metamorphosis of an insect, where new structures emerge from a disintegration. In a work much more even-handed than the Beethoven, Woolweaver and Choi displayed great interpretive creativity, from a tentative, almost uncertain opening, passing through strongly characterized episodes of muscular lyricism and twitchy dancing, finishing with a hushed and unexpectedly devastating peroration of Dowland’s theme.

Philippe Hersant, born 1948, reached back to 1723 for his inspiration, using Marin Marais’ Variations sur la Sonnerie de Sainte-Geneviève-du-Mont as its source. Marais’ original is built on a relentlessly repeated three note carillon theme, generating an extended set of variations. Hersant’s 1998 Piano Trio borrows not only the three note motive, but also Marais’ atmosphere of compression and constant musical invention. The same three note theme is sent skirling and weaving constantly throughout the piece, which is not structured in variations but in highly contrasting episodes. The carillon theme is omnipresent, sometimes in the foreground, sometimes submerged, occurring in different tonalities, at different rates of speed, and with wildly varying timbres and techniques. It appears embedded in the incredibly fast swirls of notes played by the strings at the opening and the end, and in an ostinato in the piano that builds and crashes and then builds again. This makes for a concentrated single movement work of high emotion and drama, performed with deft passion by pianist Elizabeth Schumann, violinist Kristin Lee, and cellist Popper-Keizer.

This made for an exhilarating but tiring first half; the second half brought two works with less direct connections to their forebears, and with different qualities of intensity. Judith Shatin’s Ockeghem Variations in five movements for wind quintet and piano makes allusions to the 15th-century composer’s Prolation Mass, but rarely overtly. Shatin evokes Ockeghem’s pleasure in complexity by building a scale using his name. After the exertions of the first half, the first movement of these Variations, “Lustrous,” was a welcome balm, the notes of the Ockeghem scale floating past one another, colored by characteristic timbres the winds, and ending with a surprisingly definitive ending described in the notes as one of Ockghem’s “cadential moves”. Each of the subsequent movements (“Ringing”, “Electric”, “Floating”, “Resounding”) introduces more notes into the scale, and plays with them with textures that evoke their titles—“Ringing” using repeated notes to suggest bells, “Floating” focused on long lines in the horn and oboe. The Variations provided moments of interest but stayed at arm’s-length from the listener. While the timbral complexity of the ensemble provided some delight, there were also tuttis where the six players were created a dense and muddy texture. The room at the Goethe-Institut might deserve some of the blame: A lively but not reverberant space, the small groups filled it effortlessly, the piano had especially generous and loud presence. This particular ensemble may have been a bit much for the room, never becoming overbearing but losing some transparency when the complex sounds of the winds sounded simultaneously. In addition, it might have been worth swapping the Hersant and Shatin in the program, allowing the audience to hear the Shatin with ears not so wholly attuned to the emotionally deeper pitch that the Britten and Hersant provided. Pianist Choi was joined by Deborah Boldin (flute), Margaret Phillips (bassoon), Nancy Dimock (oboe), Gary Gorczyca (clarinet), and Whitacre Hill (French horn).

Camille Saint-Saëns’ first Sonata for violin and piano in d minor, Op. 75, closed the program. The allusion here was to the beginning of the program. This Sonata recalls Beethoven’s Kreutzer through the distorting lens of Saint-Saëns’ own aesthetics and ambitions. Here, the connection to the past is more tenuous, an emulation of style and goal rather than a direct borrowing of material. The result is a curious work that can feel exhilarating and workmanlike at the same time, as the composer pieces together large scale forms, joining them with music of surpassing technical difficulty and showiness. Written ostensibly in four movements, Saint-Saëns joins the first two and last two together, creating a work of two huge pieces with only one pause. Violinist Lee and pianist Schumann attacked this beast with incredible energy and enthusiasm. Their interpretation was solid, hard-edged, and had the sense of being played just at the edge of control, but with complete confidence. There might some humor to be found in the music, but the players did not risk it; they took Saint-Saëns at his Beethoven-emulating word and kept things serious. The second movement Adagio, the one place where the composer allows some repose and reflection, was gorgeous and understated. The final movement, filled with almost ridiculously showy passagework, was played with such velocity and command that any reservations you might have about the music was swept away with your astonishment at its execution.