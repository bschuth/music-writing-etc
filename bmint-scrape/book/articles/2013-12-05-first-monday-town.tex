\chapter{5 December 2013}

\textsc{Boxx}

Genial host and founder Laurence Lesser suggested that First Monday’s unusually large audience at NEC’s Jordan Hall audience might have come for the great Schubert C major Quintet, D 956, a substitute for the originally scheduled Shostakovich Trio, which had been cancelled because of Russell Sherman’s indisposition [he’s recovering from a broken hip]. The Shostakovich would have made for a more coherent evening, as the other pieces on the program, a late Haydn trio and the Britten Second String Quartet, made for awkward opening acts. Lesser did what he could in his remarks to redress the imbalance, but ended up overselling pieces that whose greatest strengths were their idiosyncrasies.

He made Haydn’s Trio in F-sharp Minor, Hob XV: 26, sound quite unusual. It does end in the minor, something Haydn was usually loath to do, its dedication to a woman not his wife carries a faint whiff of scandal. It turns out that the piece is not in itself all that idiosyncratic, but this evening’s performance certainly was. John Gibbons made the surprising choice (for him) to play the modern piano, rather than a fortepiano, and produced a flowing, velvety sound that was nevertheless dominating. Against this violinist Kristopher Tong and Laura Blustein produced a sound that was sinewy and even astringent at times. The contrast was ear-catching, and the pianistic effect was exquisite albeit limiting. In the first movement, Gibbons’s rapid descending triplets, like ripples on a pond, were returned in the strings as distinct notes with sharp biting edges. The brief development, consisting mostly of cascades of 16ths marked with alternating pianos and sforzandos, traded drama for placidity, a choice which better suited the middle Adagio cantabile. The last movement was pale; its constantly returning theme wanted less calm and more rhythmic vitality. The unusual approach did retain one’s interest in a piece which easily could have disappeared from memory, followed as it was by two increasingly larger and more imposing works.

Britten’s Second Quartet was written in 1945, and Lesser could not resist attempting to connect it to the historical facts of that year, even suggesting Britten’s visit to Bergen-Belsen might somehow be reflected in the music. Much of Britten’s work makes powerful use of extra-musical material—most obviously in his theatrical work, where he had a genius for suiting musical gesture to text. The odd shape of this piece, consisting of two somewhat unhinged shorter movements preceding a huge 20-minute Chacony, fairly invites speculation. There is also a diffident quality that tempts one to hunt for associations to give the experience greater depth. So far as I can tell, Britten himself did not suggest any connection to the historical events simultaneous with its composition. The Chacony is explained best by the fact that the quartet was a commission written to commemorate the 250th anniversary of Henry Purcell’s death (a fact Lesser also noted). It was written right at the moment when Britten began writing his best known works—Peter Grimes also premiered in 1945, the Young Person’s Guide a year later— but it lacks their personality. Great swathes of the piece are essentially melody plus accompaniment; Britten, excellent craftsman that he was, sustains interest through timbre. Often the melody is carried by two or even three-quarters of the quartet, playing in octaves. The first movement (Allegro calmo) is a sonata-form, though the melodic material is in constant change, blurring the form. Its most memorable moments are a few stretches of hallucinatory passagework that create a ghostly and uncertain sonic landscape. The second movement is a brief and aggressive tarantella, the angry id to the unsettled ego of the first movement. The concluding Chacony consists of 21 variations, broken up by three brief cadenzas for individual players (the second violinist lacks one); the variations are often not strongly differentiated from one another, and the cadenzas were disappointingly technical exercises that verged on etudes. Despite the great mass of the piece it does not fail to retain interest; the chaconne bass is easy enough to follow, even in some of its more disguised forms, and each instance is brief enough that if it doesn’t please, you needn’t wait long for the next. The close is angrily majestic, ending with a dramatic set of slashing chords that fooled at least some of the audience into thinking the piece over before it finished. The Borromeos (Nicholas Kitchen and Kristopher Tong, violin; Mai Motobuchi, viola; Yeesun Kim, cello) played throughout with a beauty and purity of tone that put Britten’s sonic handiwork in its best light. There was room for more ugliness in the tarantella; but it was a fair trade for the crystalline octaves in the outer movements.

It was the Britten quartet’s misfortune to share a program with the Schubert C-Major, beside which it could only seem diminished. The Borromeos plus cellist Laurence Lesser sounded simultaneously epic and intimate, negotiating the emotional chasms of the first movement with quicksilver assurance, spiky, angry arpeggiations seemingly evaporating to make way for the amiable second subject. In the second movement, as the first violin restated the theme in pianissimo pizzicato, Kitchen hovered at the very edge of audibility, making a climactic moment out of near silence. All of this was accomplished without unnecessary drama or heaviness; if anything, there was an unhurried, cheerful quality to the playing, a kind of cordiality that emerged most obviously in the few moments of overt portamento that snuck out here and there. After the hush and focus of the second movement, the last movements had a gemütlich quality to them, despite their size. Lesser, playing one of the two cello parts, blended in perfectly where that was necessary, and provided a contrasting personality where needed—for instance, in the return of the main melody in the second movement, where the cello takes quite a long time to settle down, producing rumbles of remembered past themes as the others attempt to return to rest.

The Borromeo makes much of the fact that they all read from the full score; to do so, they read from computers controlled by foot mice. This turns out to be less distracting than it sounds, though I found the glowing apples on their laptops an unwelcome addition to the stage picture. For the record, Laurence Lesser played from a paper cello part, and his performance did not seem to suffer thereby. The quartet sat with the violins split—Kitchen in the traditional first violinist seat and Tong on the right, where one more traditionally expects the viola. The arrangement gave violist Mai Motobuchi a welcome chance to project sonorously without turning out and also enhanced the contrast between the two violins.
