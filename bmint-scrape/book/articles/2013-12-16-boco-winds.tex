\chapter{16 December 2013}

\textsc{Boxx}

The Boston Conservatory Wind Ensemble’s Friday program was both an epitome and a conundrum. The pairing of two familiar “chestnuts,” with obscurities epitomizes wind ensemble programming; yet the performances only blossomed in the traditional material, leaving one with a quandary over the relevance of this particular configuration insofar as the familiar repertoire for such ensembles is limited in quantity and rarely deep.

Last year I attended a concert of this ensemble at the Old South Church, and declined to review it as it consisted of three premieres which I found almost impossible to assess—the acoustic of the church was dense and reverberant, and the music was frequently difficult to parse as it swam and echoed in that large room. The first work on Friday’s program suffered from the same problems; Nico Muhly’s So to Speak is a meditation on Thomas Tallis’s anthem Loquebantur Variis Linguis, which in its clearer moments had an exciting if disorganized energy; the program notes suggest that at moments the music aspires to the condition of “singing in tongues.” But writing for massed winds is tricky, as it is all too easy for the sound to congeal into an undifferentiated mass, which happened all too often.

Muhly had originally written So to Speak for orchestra, and had “all but forgotten” it until Hewitt had asked for an arrangement for wind ensemble. From this outing, it sounds as if it could use some additional work to make an impression. The piece is written for a slightly unusual large ensemble, including a part for solo violin, and while Muhly clearly enjoys the range of timbres that affords him, the piece feels unwieldy. That solo violin, for example, is intermittently audible, but was swamped by the rest of the ensemble. I was literally closer to the violinist than anyone else in the hall, and I was not able to hear the part after its initial notes. Much of the music had this unclear, clotted character, which makes it opaque to evaluation.

Some of this must be laid upon Hewitt and the ensemble, but the next item, an all-instrumental arrangement of Richard Strauss’s early song Allerseelen (1885) used an even larger ensemble than the Muhly, and was far more successful. The arrangement was made by Albert O. Davis, in a style that any player of wind ensemble arrangements would immediately recognize: high woodwinds playing over chorales of brass, the predictable and yet somehow perfectly appropriate use of drums and cymbals in an arrangement of a song with piano. It is a commonplace that wind ensembles evoke the organ, both relying on columns of vibrating air to make their sounds; Hewitt elicited a suitably organ-like sound from the ensemble, a ceremonial but slightly distant effect, which fitted the song’s self-consciously nostalgic-romantic tone. It ended with a swoony Meistersingery peroration in perfect mid-century “band arrangement” voicing, the ensemble producing the familiar glowing, brilliant sound of “classical” pieces played by bands. As ravishing as the effect may be, such an arrangement is at best a curiosity.

The first half concluded with, Gustav Holst’s Hammersmith Prelude and Scherzo, is that odd thing, the mostly unloved “chestnut”. More ambitious than the composer’s charming but lightweight two Suites for band, Hammersmith attempts to depict the environment and spirit of London’s Hammersmith district. Imagine a Moldau painted primarily in browns and greys; a scherzo that rarely aspires to an outright laugh. There are a few light moments such as the brief solo in the piccolo that evokes a drunken bird whose swooping gestures are batted about before developmental demands push past it, but not enough to make for an engaging performance. The BoCo students did what they could, but it is ultimately chilly and unsatisfying.

The second half was devoted to Mozart’s Serenade No. 10 for Winds in B-flat Major, K. 361/370a “Gran Partita”, for 13 players, including four clarinetists, and in this performance the optional contrabass instead of the contrabassoon. The original calls for two basset horns and two clarinets, but basset horns still being a luxury item for students, the players acquitted themselves perfectly well on plain old clarinets. Mozart knew what he was doing when writing for the clarinet family, and much of the joy came from the sounds generated by the clarinetists, both in various solo roles as well as in ensemble. The Gran Partita is immensely charming; in fact, there is probably a bit more charm than can be consumed at a single focused sitting; I always find myself wishing I could be hearing it in an environment where I might have something to eat or drink while listening. The melodies are charming, the harmonies occasionally surprising; but truth be told, there’s a lot of music repeated here (two full minuets with multiple trios, for example); and I can’t be the only one who waits for the penultimate theme and variations to tire of its manipulations so that the final, hyperactive allegro can finish off the night. The execution was a model of clarity and tonal beauty; the congestion of the Muhly seemed very far off indeed, and apart from some loss of pitch and articulation toward the end that might be reasonably attributed to fatigue, the playing was thoughtful and well shaped. There were even moments of joy, with the finale appearing to induce a state of ecstatic breathlessness among the woodwinds, which they recovered from during the extended applause at the end of the evening.
