\chapter{21 Febraury 2014}

This week’s Boston Symphony Orchestra program yokes together Anne-Sophie Mutter, a superstar of decades long standing, and Manfred Honeck, the music director of the Pittsburgh Symphony Orchestra since 2008. The first half of the program belonged to Mutter, who has been championing Dvořák’s violin concerto, coupling it with the same composer’s Romance for violin and orchestra; Honeck used the second half to tackle Beethoven’s Eroica. Lee Eiseman’s intriguing interview with Honeck can be read here.

Dvořák’s violin concerto is multiply, belated–composed in mid-1879; it was revised and re-revised, premiering a full four years later. Joseph Joachim, who inspired and encouraged its composition, never performed it. It didn’t make it to Boston until 1900, and according to the program notes from last night’s concert, the last time it was performed as part of the BSO subscription series was 15 years ago. The rarity of performances is a shame. Architecturally eccentric, it takes some acquaintance to make sense of it. Especially in the first movement, Dvořák anticipates structural innovations that typically present themselves when listening to music of our own time, hinting at but not committing to sonata form, forcing the listener to pay attention moment to moment to discern the bones that support the structure. It is emotionally intense and mercurial, never wallowing in either beauty or bathos, and occasionally startling the listener with sudden outbursts. It is also very, very hard to play.

This makes it an ideal vehicle for Anne-Sophie Mutter, whose flinty, intelligent performance emphasized the modern in a composer who commonly is thought of as rustic and folksy. She and Honeck released a recording of this piece with the Berlin Philharmonic in 2013, and their long mutual familiarity with it was obvious in the first movement’s constantly evolving dialogue. The music changes direction rapidly, melodic statements giving way to stormy passages of cadenza-like material, then dissolving into motivic fragments that gradually re-coalesce. In Mutter’s hands, this became a conversation that was both eloquent and turbulent, that gave a clear sense of direction while keeping the ultimate destination uncertain. This was thoughtful playing that did not try to ingratiate itself—I have never heard portamenti that were so devoid of schmaltz, yet were absolutely heartfelt. The second movement opens with a quiet and aching melody, but the listener hoping for a sweet moment of respite is soon disappointed—Mutter finds an emotional desperation in the music below its surface that undercuts its surface beauty. She possesses a varied tonal palette, and has a whole range of timbres that contain a just a touch of acid. Carefully deployed, the music never even threatened to curdle into mere prettiness; this movement was a sad song sung by someone far too aware to slip into a thoughtless nostalgia. There was an especially striking moment, where the melody appears pianissimo: playing with a wiry tone just slightly out-of-tune, she evoked a hurdy-gurdy, and we dropped into the same desolate landscape as Schubert’s “Der Leiermann” from Winterreise, or of a 20th-century existentialist play. The final movement, a rondo-like juxtaposition of two Czech dances (a furiant and a dumka), was brilliant and virtuosic, both in composition and execution. There is a fierce joy in this music; the dancing is muscular and demanding, the rhythms pounding, and the final bars consist of four major chords in rapid succession, a quick set of bracing slaps to the face. Mutter played with utter technical mastery and furious concentration. Honeck and the symphony were sympathetic accompanists throughout, especially in the first movement, where a sense of interdependent conversation is crucial. Although the piece that has a reputation for being over-orchestrated, Mutter was never in danger of being covered, though at times there was an uncertain balance between the winds and the strings. Mutter was granted an almost immediate standing ovation from the sold-out audience, many of whom were certainly there just to see her.

While Mutter is many things—strong, intelligent, passionate—I don’t know that I usually think of her as either vulnerable or naive, two qualities that may be necessary to make Dvořák’s Romance for violin and orchestra a complete success. A recycled movement of a string quartet, this was the evening’s opening piece and it never felt comfortable or settled. There were some rhythmic and pitch disagreements in the orchestra, and while you never doubted the sincerity of the violin, neither were you beguiled by it. The piece is pleasant but slight, and one’s memory of it was fairly obliterated by the overwhelming performance of the concerto which followed.

In Honeck’s interview, he speaks of the shock which Beethoven’s third symphony induced in its original hearers, and of his desire to discover that shock in performance—this as opposed to the smooth and beautiful heroism Karajan might have sought. In the event, one could certainly hear how Honeck crafts the sound to make it new to our ears. In the first movement, for example, the texture was made spiky by playing with additional detachment, letting much more space in between sharply articulated notes. Dynamics would suddenly rise and subside, subtly italicizing phrases. The tempo was the “allegro con brio” Beethoven asks for, but not notably fast. A famous dissonant moment discussed in the interview did have a peculiar, almost glassy timbral quality I have not heard before. But it is fair to say that Honeck’s reading this evening was not focused on the first movement, which was played without exposition repeat. It had a surprising lightness to it, and seemed to end a bit too soon.

The second movement was Honeck’s showcase. This was an operatic reading of the funeral march, emotionally acute and pointed. The orchestra played beautifully, with a wide range of hues, which appeared in blocks and slabs, like a color field painting. The major key interlude in the middle of the movement was sunny and ameliorative, a clear look forward to the complex happiness of the first two movements of the Pastorale. The episode at the end of the second movement which Honeck describes in his interview was weird and uncanny, imbued with a touch of Mahlerian grotesquerie; and the fate motive he is so excited about in the third horn has never been so prominent. It fairly blared.

The following scherzo was palate-cleansing. It ticked along quickly and satisfyingly, a high-energy jest in the wake of the weighty adagio. The finale started in a similar light vein; the statement of the melody was quite fast (“allegro molto” for sure), and very little time was spent dwelling on the fermatas at the end of each phrase. By the time the flute has its famous solo, the speed was bordering on breakneck. But Honeck slowly and deftly reintroduced heft and drama as the music progressed, so that it landed with full force and brilliance as it came to an end. This movement is one of the most frustrating in Beethoven, a weirdly formalist structure built on a halting and skeletal theme; on this evening, it had an unexpectedly Haydnesque quality at its start, but an undeniably Beethovenian finish.

On the podium Honeck is a fascinating combination of lithe movement from the legs and square, almost geometric gestures in the upper body. His attention constantly darts around the orchestra. He rarely draws attention to himself, though he is not above gathering himself up and then exploding outward to mark a climax or the end of the movement. This same sense of considered dynamism was evident throughout the Eroica, but all of his interesting touches haven’t quite set up into a coherent whole. Perhaps this is due to Honeck’s ideas needing more time to mature, or the fact that he and the orchestra have worked together only once before, in 2005. It may take another return visit to settle the issue; certainly there’s enough interesting music-making here to make it worthwhile.
