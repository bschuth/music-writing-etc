\chapter{3 November 2013}

\textsc{Boxx}

At its best, concert music should aspire to what Elliott Carter called “durability”—music that can be heard more than once and still hold one’s attention, giving up new insights. With music of the last 50 years, though, it can be hard on first-time exposure to determine how to listen at all, much less to try to explore the details of what a piece might be saying. At times we are granted a second performance at the same concert, which at allows us the chance to get past the surface and listen more deeply. Callithumpian Consort’s Thursday concert at the Isabella Stewart Gardner Museum offered a different luxury even less rarely afforded new music audiences—four pieces by the same composer on the same program. Chaya Czernowin has been on the Harvard music faculty since 2009, arriving there by way of UCSD and the University of Music and Performing Arts in Vienna. Four of her pieces, written between 1996 and 2013 were played, the last written being given its premiere, with the composer in attendance. Works by Giacinto Scelsi and Morton Feldman provided leavening and perspective.

With one unusual exception, the works came from a stylistic world one might lazily call modernist, a universe of works that consciously avoid most of the organizational conventions that held music together until the upheaval of the 20th century. Such music eschews melody as popularly understood. It does not function under the conventions of diatonic harmony, or if it does, does so obliquely and without traditional expressive intent, and avoids using totalizing serial techniques. The music may have a regular coordinating beat, but is written so as to obscure the sense of pulse, although local rhythmic motives may appear. The traditional language of the reviewer wilts in the face of this music; one finds oneself writing about “gestures,” attempting to find development in the deployment of timbres both traditional and “experimental”, listening for echoes or repetitions of any element of the material to establish a sense of structure. The best of this music can be gripping; I waxed rhapsodic in my reviews from the Festival of Contemporary Music at Tanglewood about the music of Helmut Lachenmann, music that aggressively rejects almost everything that traditional classical music takes as a premise.

The compositions at the Gardner don’t have quite same personality. Even having heard all four of them, I am sure I would not recognize another of Czernowin’s œuvre. I am not sure that I would have recognized her four as coming from the same pen without the program in front of me. Without knowing Czernowin’s body of work better, I cannot tell if the voice—obscuring eclecticism of these pieces is typical of her work in general, or if the selection of pieces intentionally sought to find contrast. It was not until the final work, the premiere of Slow Summer Stay III: Upstream, that I felt I was in the presence of music that might just have the promise of durability.

The earliest work on the program, 1996’s Afatsim, was the only piece the composer spoke about during the concert. She told us that she wrote the piece for nine instruments, conceiving them as only four; three of them consisted of pairs of actual instruments, and one acted a cluster of three. In performance, this conception was difficult to parse; having been briefed on this schema, it did appear that the pairs of instruments played together more often than not, although at first blush it appeared that the material they played was joined together at best tenuously. With the exception of the bass flute/viola pairing, which produced ghostly rustles and shivers, none of the other “instruments” had enough personality to be heard as a unit. There was a lot of musical material in play at once, but it lacked a strong enough profile to be heard without exceptional concentration, and the effect by the end was a common one in music of this style: a great deal of activity finally coming to an end.

The concert opened with a piece of Czernowin’s from 2008; entitled Shaft (Drift), it might better have been called “attack”. It is an etude in short, sharply articulated notes, whose epitome is the dry, loud sound produced by the ratchets in the percussion. The piece is for an unusual quartet of percussion, piano, electric guitar and saxophone (baritone and soprano, bracketing the extremes of the instrument family’s range). The attacks come fast in clusters of repeated notes, and then they spread out more, covering more ground. The saxophone often stutters rapidly; at times the sound of keys and pads slapping down swallows up any sense of pitch. The pianist mutes the strings so that almost all pitch is lost and the only remaining sound is of the felt hammer striking a non-resonant string. The electric guitar emits its own high-pitched, gossamer flutters. For contrast, there are occasional great sweeping gestures in the piano, or chords in the guitar that bend out and back in. At its best the piece generates a fitful sense of forward momentum, with correspondences and juxtapositions that catch the ear, without quite resolving into a whole

The brief solo piano work fardanceCLOSE from 2012 is an oddball in this company.  I heard it as a program music, enacting the destruction of Czernowin’s sound world by a brutal deployment of diatonic harmony. Most of the piece consists of fleet patters of notes which chase themselves around at the extremes of the keyboard. At the high extreme, it evokes a flock of swallows on the wing, darting, changing direction suddenly and simultaneously. When these figures descend into the bass, they swim around in the pedal, high skittering transformed into a rich blur with an underlying rumble of movement. Then at the end, a loud and violent gesture erupts—a short chord struck twice followed immediately by a sustained major chord. These two chords clash dissonantly with one another, but played together as unit, they simply take over the music and the sound world we first heard is extinguished entirely. I have never heard a major chord sound so sinister. It’s an effective theatrical moment, but it is brief and the drama doesn’t haunt for long. Stephen Drury played the piece with energy and vivid contrast, the final explosions frightening in their intensity.

Before the performance of Slow Summer Stay III, the Callithumpians performed Morton Feldman’s The Viola in My Life I from 1970, a miniature “concerto” for viola, piano, percussion, violin, cello and flute. This piece lives in the same stylistic world as Czernowin, but in a very different neighborhood. The surface is typically Feldman – almost exclusively pianissimo with alternations of lines of pitches, often by the solo viola, with clusters from several instruments at once.  The musical material is reduced to a handful of pitches, and the music unfolds at a glacial pace. It is easy to feel that nothing is happening in the piece, and it does aspire to a kind of crystalline static quality. But it repays concentration; the reduction in pitch material makes it possible to listen to the piece horizontally and vertically at once. One can trace the journey of a given pitch, from its quasi-melodic reiterations in the viola into the dense community of the clustered chords that follow it. Listened to with patience, the piece creates a self-defined and self-consistent world which both has a sensual, if somewhat tart, surface, and intellectual dimensions that can be explored in realtime as the sound rolls out before you. Violist Ashleigh Gordon found ways to make the austere lines of the solo expressive, without drawing any attention away from the ensemble. The few moments where the ensemble was not perfectly aligned underlined the fragility and challenge of music whose surface is the opposite of virtuosic, without disrupting the overall atmosphere.

Slow Summer Stay III was the most satisfying work of Czernowin’s. The piece is a knitting together of two octets (Slow Summer Stay I and II); the two earlier pieces were written knowing that the third piece was to be made of them. The program notes say that they were combined with “shifts of time and other changes”.  There are therefore sixteen players in III, forming two identical octets, each with its own conductor. The orchestration is again unusual: each octet consists of a bassoon, two clarinets, viola, percussion, piano, cello and amplified acoustic guitar. The notes further tell us the piece is a “loose palindromic canon”, which provided just enough scaffolding to structure our listening – and sure enough, musical ideas could be heard echoing between the two octets. The ideas were sometimes merely rhythmic cells, or repeated intervals; but they made their own logic, and created their own language. Perhaps my ears had been trained by the Feldman, but I found this piece both challenging and comprehensible, despite the fact that for most of the piece the two octets play in independent rhythms. The musical fragments put into play were strongly characterized, and some underwent fascinating timbral changes as they arrived in new instruments. The piece is dense, but the musical events could be parsed as they occurred, something I found almost impossible in Afatsim. The piece begins with a long stretch of music for the first octet alone; this meant that when the first octet stopped playing near the end, leaving the second octet to bring the piece to a conclusion, there was a sense of arrival, a dramatic event that felt organic and inevitable. All of the performances of the Czernowin were presumably definitive, given the composer’s presence; certainly Slow Summer Stay III was a showpiece of ensemble and balance thanks to the Consort’s conductors Drury and Jeffrey Means.

The other non-Czernowin piece was Scelsi’s Hyxos, from 1955, which was played early in the program and existed uneasily with the rest of the evening. Written for alto flute and gongs, the piece is in three movements. It is an ancient—sounding monologue for flute, although the gongs find their own melodic voice at the very end of the piece. The flute’s lines are broadly modal, and the modest genius of the piece is in the apparent narrative development that emerges; but as part of the “Avant Gardner” series, it seems ill fitting, being less innovative or challenging than say, Debussy’s 1913 Syrinx. Flutist Ashley Addington gave a performance both musical and conversational, with close attention to the dramatic structure of the work.
