\chapter{16 June 2013}

\textsc{Boxoff}

Aston Magna, America’s oldest summer early music festival, opened its 41st season on Friday at Slosberg Auditorium at Brandeis University. The early music movement has been around long enough now that much of what was once revelatory—gut strings, lowered concert pitch—is now commonplace. Nevertheless, Friday night’s program, “The Art of the Chalumeau,” proved the festival can still have offer surprise and revelation to audiences.

The chalumeau is a single-reed instrument, the precursor to the clarinet; its name comes from a French transformation of the Greek word for “reed.” Aston Magna’s clarinetist Eric Hoeprich’s chalumeau was a surprisingly little thing, about the size and appearance of a soprano recorder, with a small clarinet-style reed and mouthpiece. With a bore similar to that of the modern clarinet, it shares the property of “overblowing” at the twelfth rather than the octave—overblowing allows the player to double the range of the instrument by forcing the air column to vibrate more quickly. The chalumeau has two small keys to allow the player to reach the upper notes of that twelfth, but their location and the overall design of the instrument doesn’t actually permit overblowing, so that the range of the instrument is not much more than an octave. The chalumeau exists vestigially in the modern clarinet, where the lowest octave of the instrument is still referred to as the “chalumeau register.”

Ancient clarinets have become somewhat more common, thanks in large part to Mozart writing his great clarinet works for an odd transitional version of the instrument, but the chalumeau has always had an air of mystery to it. Almost no original chalumeaux exist today—Wikipedia says eight (Mr. Hoeprich’s book “The Clarinet” is a major source for that article). Once thought to be an obscure and rarely used instrument, research has uncovered a surprising amount of music written for chalumeaux in their heyday, from the late 1600s to early 1700s. Much of the music is for chalumeau in ensemble; there are also handfuls of pieces for groups of chalumeaux. “The Art of the Chalumeau” gave a charming and fascinating overview of pieces for solo chalumeau, as well as a performance of the Fasch concerto for chalumeau, the best known concerto for the piece.

As played by Hoeprich, the instrument sounded clearly ancestor of the clarinet that it is, but with a bit of a ghostly air to it. The tone in the middle of its range was mostly familiar, but at the extremes of its range it picked up a different timbre, perhaps a little reedier – a supple reediness on the low end, a more piercing quality up high. It lacked the punch and projection of the modern clarinet, but instead substituted a sweet, even seductive roundness of tone. What historical accounts we have of the sound of chalumeau players are divergent; the Grove quotes critics calling it “raucous,” “disagreeable,” “savage,” “like a man singing through his teeth.” However, it was also called “infinitely pleasant,” and the music on the program made it clear that the chalumeau was valued for its velvety tone, its singing quality, and as a companion to the human voice. Hoeprich’s tone was warm, liquid and personal, making a case that the dismissive critics might well have been listening to less accomplished players who produced sounds not unlike those of beginning clarinet students today.

The first half of the concert sandwiched four short pieces featuring the chalumeau between a string concerto by Vivaldi and the Marcello concerto for oboe and strings, simultaneously placing the instrument in context and giving a sense of surprise and wonder when we finally got to hear it play. The four chalumeau works were instrumental adagios by Johann Adolph Hasse and Francesco Conti, alternating with arias from Ziana and a young Handel, with soprano Kristen Watson.  The Hasse adagio, from a trio for oboe, chalumeau and continuo, allowed the listener to easily contrast the sound of the baroque oboe, a more established instrument, with its piquancy, firm attack, and extended range, with the plush tone and muted articulation of the still-evolving chalumeau. The Ziani aria “Tutti in pianto” from Chilonida had a virtuosic obbligato chalumeau part (written by Joseph I of Hapsburg). The extent of the virtuosity called for in several of these pieces (especially the Fasch concerto) was surprising, given the limited range of the instrument and the complexity of fingering required in the absence of durable and reliable keys. Handel’s contribution was “Io sperai trovar nel vero” from The Triumph of Time and Truth, his first cantata, and gave the chalumeau the chance to shine with lines of surprising chromaticism. In the second half, Johann Friedrich Fasch’s concerto for the chalumeau combined long, intense singing lines in the slow movements with almost frantically florid passagework in the fast movements. Fasch experiments with long held tones with slowly increasing or decreasing volume, showing that this favorite clarinet effect, which reaches its apotheosis in Messiaen’s “Abîme des oiseaux”, was already fascinating composers before the clarinet even came into being.

After the Fasch, a contrasting pair of Vivaldi arias compared the use and effect of the chalumeau and the oboe. First there was a glowing, subtly dancing “Domine Deus” from the Gloria featuring Ms. Watson and oboist Stephen Hammer. Then, Hoeprich and Watson combined for what I thought was the real revelation of the evening, “Veni, veni” from Juditha Triumphans devicta Holofernes barbarie (RV 644). This song of grief and consolation is Vivaldi at his finest—simple materials (a leaping dotted rhythm, a brief run, a pulsing harmonic cushion in strings) expertly juxtaposed to great emotional effect. The performance was hushed, intimate, and immensely affecting.

Watson’s voice is well-structured and warm, subtly colored, blending in well with the orchestra, always presence without overwhelming or drawing attention to itself. In the shorter arias I thought perhaps she was a bit restrained, but the final work of the evening, a cantata by Francesco Conti, turned the focus entirely on her, and she gave an exciting and varied performance. The work itself is well-made if not deeply inspired, with a libretto filled with romantic clichés. However, Watson’s performance redeemed any weakness in the piece, singing with a sense of drama and intensity perfectly that was scaled perfectly to the needs of the music, the size of the ensemble and the space. I am often impatient with recitative, but Watson brought a dramatic presence and liveliness that I found entirely engaging, enough so that I stopped looking at the libretto translations—the text was much less interesting than what she was doing with it. The Conti did include a chalumeau in the orchestra, but its unique qualities were harder to discern in this context.

An appropriately minimalist string section was provided for the evening – violinists Dan Stepner and Jane Starkman, violist Anne Black, Guy Fishman on cello, Anne Trout on violone and Catherine Liddell on theorbo. They provided sensitive and attentive accompaniment in various combinations throughout the evening, and shone on their own and individually. The first piece on the program, Vivaldi’s concerto in E minor for strings (RV 134), had an especially lively first movement “fugue” —the fugue subject is so short and so simple it is more of a kaleidoscope of stretti, and it was tossed around the ensemble with great energy and abandon, and Stepner brought a particularly incisive and biting tone to the somewhat less inspired third movement, goosing it with needed extra energy.  Oboist Hammer provided a satisfying and thoughtful performance of Alessandro Marcello’s well-known oboe concerto at the end of the first half. The performance was attractively enigmatic—full of feeling, but restrained and hidden, drawing you into the music rather than showing it off for you.

By the end of the evening, it was clear why the chalumeau caught the attention of composers with its distinct voice, entirely unlike that of the other winds. However, even in Hoeprich’s capable hands one can tell the instrument is unwieldy, and its range limiting—there was a great deal of finger-flying activity required to execute trills and ornamentation, and moments of the Fasch were thrilling for the amount complex movement required. Once the technical innovations in wind instruments favored the clarinet over the chalumeau, its brief time in the sun was over.