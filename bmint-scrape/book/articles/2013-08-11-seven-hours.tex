
\chapter{11 August 2013}

\textsc{Boxxx}

In 1987 composers Michael Gordon, David Lang and Julia Wolfe formed Bang on a Can, dedicating it to fostering the kind of music they wrote. To draw attention to the project they mounted the first “Bang on a Can Marathon,” 12 hours of music in New York City. It turns out they were on to something; the marathons have continued, year to year, and in 1999 the Bang on a Can Summer Festival (nicknamed “Banglewood”) came to Mass MoCA, in North Adams. At the end of each festival they mount another marathon—or technically a half-marathon, scheduled to run for a mere six hours. While the New York marathons function as showcases for the composers and ensembles under the Bang on a Can umbrella, the Mass MoCA marathons exist to show off the fellows and staff of the festival, for whom it is the culmination of three weeks of work. It therefore has that special energy which comes from a group of people who have come together with shared passion and spent weeks forging relationships and making art together. This summer’s festival marathon took place over more than seven hours on Saturday, August 3. Nineteen works were presented between 4 and 11:15 pm.

The style of music that Bang on a Can represents is not easily labeled, though it has several clear characteristics. Most importantly, they present concert music—notated, and to be played in public venues by trained musicians. At its best the music strives to fulfill Elliott Carter’s two criteria for good Western art music, that it be “interesting and durable.” The work of the three founding composers is heavily influenced by minimalism, but the influence is realized differently for each. Julia Wolfe’s writing is frequently very dense, especially for strings. Her Fuel for string ensemble was performed at the Marathon, three movements of almost continuous tremolos, shot through with sudden changes in dynamics and accent. Inspired by the docks at Hamburg, she found a way to evoke the sounds of an industrial waterfront—rumbling of engines and other vehicles, shriek of metal—without resorting to mere sound painting. This music is tough and uncompromising, and if the listener is not concentrating fully on its slow evolution, its surface can become grating, perhaps intentionally so.

David Lang’s minimalist techniques are less aggressive if no less intense. Lang has gained recent notoriety since being awarded the Pulitzer Prize for his Little Match Girl Passion. This year’s marathon featured these broken wings for “Pierrot plus percussion” (violin, cello, piano, flute/piccolo, clarinet/bass clarinet). The first movement consisted of small cells of notes over a rapid pulse, which shifted and interlocked, creating a surface of small waves, generating a pervasive melancholy and a lurking anxiety. The second movement was a tour de force of truly minimal means: it began with a single pitch on a flute intoned over notes struck on a motorless vibraphone. Lang subtly varies articulation and attack to create interest. After a sudden crash—a bundled metal chain had been dropped to the floor –a second pitch appeared. (It called to mind Ligeti’s Musica Ricercata, piano variations that add one pitch per variation, although Lang stopped after a handful of pitches had been presented.) The third movement begins with ostinatos and displaced melodic lines.

Lang’s music has a fragile beauty with a feeling of menace, built out of clearly audible processes that embrace small imperfection to create surprise. It is the sense of audible process that characterizes many of the pieces presented by Bang on a Can. This is coupled to a nondenominational approach to tonality: many pieces are unabashedly tonal, while some are more than willing to experiment with crashing dissonance and even noise. Still more important to the group are pulse and rhythm; in fact, over seven hours of this music, one starts hunting for the particular groove a piece is going to run in, and then listens for how it will use that groove: does it become a rut, does it lock in and allow the composer to run with it, does it become a base expectation that the music will frustrate or play against? More important, the willingness to engage tonality and pulse simultaneously gives many of the pieces a connection to non-concert music—that is to say, to popular music. A drum set appears in several of the pieces, an instrument whose presence and associations can swamp the music it is a part of. However, in Bill Ryan’s Drive for clarinet, saxophone, trumpet, violin, piano and drum set, it is within a dynamic texture and its distinctive sound provides contrasting color. In arrangements of four pieces by Aphex Twin, drum set material is transformed and distorted, stuttering and shifting, creating as much negative rhythmic space as it does driving beats.

Bang on a Can does not shy away from amplification, which permits composers to work with chamber ensembles that combine brass and strings while keeping the latter audible. The amplification during the marathon was never used merely to increase the volume of the music for impact, a weakness found in some minimalist ensembles—I’m thinking here of Philip Glass, who became notorious for using amplification to create visceral immersion. However, all of the works at this marathon were amplified, even though some did not seem to demand it. The venue at Mass MoCA does not appear to be designed for any sort of unamplified performance. It is the approximate size and configuration of a hotel ballroom—a large black box with a very high ceiling, and nothing to help the acoustics. One can reasonably object to amplification of instruments because of the effect it has on the sound: it removes depth, typically holding the sound in plane of the speakers, and it etiolates high overtones, flattening timbre. All this made the event sonically tiring over the course of the evening, and contributed to additional delays between pieces—not only did instruments, stands and chairs need to be moved, but microphones needed to be positioned and checked. It should be noted that the amplified sound was generally excellent, although the final Aphex Twin arrangements were not well-balanced.

Two pieces stood out as music one might want to return to in detail, each for soloist with ensemble. Bun-Ching Lam’s Piccolo Concertino succeeded in making music that was thoughtful and probing for an instrument often used only for brilliance. Constructed primarily from overlapping long lines, streams of notes passed from instrument to instrument, Lam made canny use of the lower register of the instrument and of unusual pairings—the “orchestra” was composed of oboe, clarinet, French horn, trumpet, violin, viola, cello, bass and percussion. Lina Andonovska played the piccolo part with a varied palette and with intensely focused calm. However, the high point of the evening had to be Annie Gosfield’s sort-of cello concerto entitled Almost Truths and Open Deceptions. Gosfield has had a long career creating powerful music based on the repetitive poundings and cyclings of industrial machinery (she is fond of two-part titles, and one her pieces in this genre is called XXX and XXX). Almost Truths and Open Deceptions combines melodic gestures of emotional power and Romantic intensity in a matrix of varied repetition that never ceased to surprise. Written for cello with a small amplified ensemble of violin, viola, piano, percussion and bass, it owed much of its impact to the impassioned and charismatic performance of Ashley Bathgate, who was able to take the lead when the music demanded it, then recede back into the ensemble seamlessly.

The level of performance throughout was impressive. Over the seven hours, players kept wwith more music, which often made quite different demands, and the energy and musicianship never flagged. The enthusiasm of the participants was particularly notable. Reading the program biographies, one is struck by how different many of these performers are. There are those who are clearly attempting to create a career in the usual way, whose biographies are filled with the names of the venues and ensembles and teachers with which they have been associated, along with eponymous websites. But many of the players restrict themselves to a brief statement of why they dedicate their lives to music, some because they are undergraduates, others because their musical lives have taken them on different paths. Some are just playful and strange. It is this sense of community, a shared vision that may be difficult to articulate—David Lang twice referred to “this music” bringing them together, as if we couldn’t quite say what it was exactly, but we all knew it when we heard it—that are enough to bring people together year after year. Above and beyond the durability of any single piece, the community of musicians for whom such concert music is a rallying point, a place to make a home, is what makes attending the Bang on a Can marathon so satisfying, and leaving one at the end of seven hours tired and overwhelmed, but still curious to hear what the next piece will sound like.