\chapter{2 March 2014}

The Quatuor Ebène played an unusual program at Friday’s Celebrity Series concert at Jordan Hall. The first half was Mozart’s Quartet in E-flat, K.428, and Bartók’s Third Quartet. The second half consisted improvisational arrangements by the quartet of jazz standards (“Misty”, “Nature Boy”, “All Blues”) and popular music (“Come Together”, “Miserlou Twist” from Pulp Fiction, a tune from M. Portal’s soundtrack to Max Mon Amour), as well as an arrangement of Piazzola’s “Libertango”. For an encore, they performed “Someday My Prince Will Come” from “Snow White and the Seven Dwarves.”

The quartet consists of four young Frenchmen, who cut striking figures on stage, violinists Pierre Colombet and Gabriel Le Magadure and cellist Raphaël Merlin looking sleek in black jackets and slacks. Violist Mathieu Herzog is built a bit more substantially, and has a vaguely glowering look about him which he used to good effect. There was quite a lot of last-minute audience noise (please, Boston, show up on time!) as the quartet prepared to start the Mozart. As they waited, with bows poised, Herzog very visibly raised his gaze from his music to the hall, and the sounds quieted instantly.

The Mozart quartet has more than a touch of weird. The opening begins with unison octave E-flats and then lands on an unexpected A-natural, and takes its time unwinding the consequences of this choice. The Ebènes had taken quite a long time audibly tuning off stage, which paid off in the utterly pure unisons they produced. The perfection of execution emphasized the strangeness in the music—one unfamiliar with the pieces could be forgiven for wondering if they had begun the Bartók by mistake. Their sound was compact, smooth, and balanced. The first violin led but never dominated. There was an ethereal quality to the sound, but it still had good bones to it. Once the first violin broke out into sixteenths in the last movement, you could feel that sixteenth pulse trembling just under the surface, even through silences.

For the Bartók, their sound opened up and there was a sense of air among the players. The counterpoint in the first part was crystal clear and the individual lines took up distinct locations in space. The interpretation emphasized the episodic nature of the piece—the contrasts in material, dynamics, and timbre that drive the piece along. In some performances this can give a sense of controlled chaos, which is thrilling if done well. The Ebènes eschewed chaos; each new episode came fully formed and thoroughly realized, if not always building on one another. However, the steady accretion of material resulted in an impressive edifice, and the fleet and thrilling passagework in the allegro sections adorned the structure with some bravado. By the end, the whole was exactly the sum of its well-wrought parts—no more, no less.

The Ebènes have a case of genre-anxiety that makes them a bit defensive about their programming. During the second half the pieces were introduced from the stage by cellist Raphaël Merlin, who began the set by congratulating the audience on remaining for the “strange” pieces that were to follow. The player’s press biography is more aggressive:

Rather unusual in today’s world of chamber music, Quatuor Ebène’s stylistic acrobatics may at first meet defiant ears. Defiant, perhaps, because of the general misuse of the term “crossover,” which so often serves to cover mediocrity and redundancy. And yet, with the Ebènes, whenever they create a new work, it is always with taste and integrity.

I’m not sure to whom these “defiant ears” belong. Perhaps it is my own mongrel upbringing, but the idea of accomplished musicians of any stripe who enjoy playing music in diverse genres is hardly challenging. But there is something to be said about there being different modes of listening and enjoying music, and the Ebènes may want to consider thinking more deeply about how that influences their concertizing. It will not do to pretend to ignore genre or style, and the fact that one loves a variety of musics does not mean that one can play any mixture of music and have it understood and signify in the same way.

The very first piece played in the second half pointed up the challenge in mixing “popular” music, no matter how realized, with music composed in the Western art tradition. The players began producing hazy, dissonant clouds of sound which grasped around for a tonal center; a melodic line wandered through the fog, as if it were hunting for a tonic obscured by layers of suspensions. Then the haze began to resolve itself, and suddenly “Misty” was being sung… and laughter of recognition spread through the audience. The song was expertly and passionately played, and the arrangement retained enough tang that it never threatened to turn into lounge music (the great danger of string quartet jazz). But once the tune appeared, the act of listening changed completely. That laughter signaled a change in how we received what we were hearing; and it showed how a change in style can discomfit us, which we cover with humor. Before the tune appeared, the sounds were connected to the pieces in the first half by our mode of attention; once it appeared, it would have been ludicrous to try to go back and listen in the way we had been listening before. Now our ears were attentive to the luxury of melody, to the elaboration of the familiar. The kind of pleasure the inventiveness of their arrangement of “Come Together” provided works differently on us than the inventiveness of Mozart.

Let me be clear: there is nothing more or less valuable in either mode of hearing. But they do exist, and if they are not outright incompatible, they are at least in conflict. The Ebènes tacitly acknowledge this by the segregation of the program. As entertaining, playful and smart as their take on “Miserlou Twist” is, it would have been entirely out of place in the first part of the program. Following “All Blues” with Bartók would have caused a kind of stylistic whiplash.

What we end up with is an evening that, for this listener, was enjoyable but unsatisfying. I would have loved to hear them play Ravel or Fauré after Mozart and Bartók, even if that meant losing their “jazz”. It might have cast a bit more light on the talents of the quartet, and made more connections between the pieces. Alternatively, I would pay good money to hear them play an entire evening of their own arrangements in an environment where I might sit a bit more comfortably and even have a drink or two.

The music on the second half was listed in the program under the rubric “Fiction”, which is the title of their recording of much of this repertoire. This part of the evening contained some spectacular musicianship. As might be expected there was improvisation, but it was secondary to the inventiveness of the arrangements, which emphasized harmonic density and color. “Nature Boy” emphasized the uncanny quality of the leaps in the melodies; the Max Mon Amour tune was pretty without being cloying. Colombet has a tendency to let his technique run away with his improvisation, a criticism one can reasonably level at later Coltrane, so he’s at least in good company. Merlin produced an astonishing pizzicato break in “All Blues.” The chaos that might have been lacking was in exhilarating abundance in the arrangement of “Libertango”, one of the few times I have seen classical players get Piazzola right. Merlin provided some entertaining patter between tunes, including some banter between the players (Herzog seems to be a bit of a troublemaker). The encore, “Someday My Prince Will Come”, began and ended with the quartet singing the song a capella in French, a moment that was charming and touching as well as musically satisfying. The gentlemen may have light voices, but they can harmonize excellently, and M. Le Magadure has an affecting high tenor/falsetto. The sight of string players singing struck much of the audience as funny much as the appearance of “Misty” did—the anxieties surrounding music’s presentation seemed particularly pronounced tonight. That intimacy, mixing with the nostalgia attached to the song and its own intrinsic yearning, made for a powerfully melancholy close to the evening. Make no mistake, these four musicians are hugely accomplished performers, and their work together was alert, sensitive, and intellectually interesting, in both halves of the program. But the only thing the two halves of the program really shared with one another was the quartet itself.