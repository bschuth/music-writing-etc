\chapter{30 June 2013}

\textsc{Boxxx}

The imprimatur of the Boston Symphony Orchestra gives the BSO Chamber Players some solid local cachet. Audiences trust them, and this may give them the freedom to program concerts like that heard at the Rockport Chamber Music Festival Thursday, which bookended intriguing lighter pieces by Martinů and Elliott Carter with Mozart and Brahms .

Mozart's Piano Quartet in E-flat Major, K. 493, is the second of his two. The first, K. 478, was meant to be the first of three, but it sold so badly that the publisher cancelled the remaining two.  The instrumental combination was unusual, the musical language challenging for the time, and the parts too difficult for the amateurs who were the primary audience for this work. Nevertheless, Mozart went ahead and wrote K. 493 anyway, and without seeming to have taken any of these criticisms to heart. The piano is asked to play brilliant passagework that would give pause to most amateurs, and there is no lack of harmonic invention and chromaticism. Haldan Martinson (violin), Steve Ansell (viola) and Jules Eskin (cello) of the Chamber Players invited David Deveau to take the piano for a beautiful if tame performance of the work, through which a subtle charm and sweetness shone. The third movement elicited the second laugh I have heard in performance at Rockport within a week; the first came in the final movement of the Bartok Fifth Quartet. Any concert community that can get the jokes in both Mozart and Bartok, and that is willing to laugh out loud at both of them, is OK in my book.

Bohuslav Martinů’s Nonet combines a string trio with bass (the string players from the Mozart with bassist Edwin Barker) with a wind quintet (Elizabeth Rowe, flute; John Ferillo, oboe; William Hudgins, clarinet; Richard Svoboda, bassoon; and James Sommerville, horn).  Martinů’s music hovers at the edges of the repertoire. A wildly prolific composer, he has a unique but derivative voice. His most typical music combines folk-influenced materials with neoclassical techniques, mixed with a creative sensibility that is restlessly, almost frantically, inventive. The Nonet is a relative rarity in that it does not include a piano, allowing Martinů to experiment with the sound possibilities offered by this miniature orchestra. The overall effect in the two fast outer movements is that of a mildly modernist Dvořák hopped up on Pulcinella; the constantly changing combinations of registers and instruments provide most of the interest as the ingratiating folk-like melodies churn and combine. The middle movement Andante provided a note of melancholy for contrast, without providing much additional profundity. The Players performed with precision and rhythmic drive, the various permutations of tone and timbre allowed to express their personality clearly no matter how briefly they were constituted.

Double-bassist Edwin Barker provided the one piece of on the program less than 50-years -old, the Elliott Carter’s brief Figment III for solo bass from 2007. As with much late Carter, the work is transparent and well-crafted, a compositional etude exploring the characteristics of the largest string instrument. It contrasts high, singing melody, not tonal but easily singeable, with low, growling motives of rapid notes, and with transitional material that hops from one extreme of the instrument to another. The piece plumbs no depths, but provides a miniature showpiece for an often neglected instrument. Overhead conversation at intermission revealed some anxiety about the Carter, but in the event this music was neither challenging nor forbidding; in fact, it functioned as an excellent palate cleanser before the final piece, the Brahms Clarinet Trio with Deveau, Hudgins and Eskin.

Brahms wrote the Trio the same summer he wrote the Clarinet Quintet. The Trio doesn't rise to the same level of accomplishment as the Quintet, and the clarinet writing is not as idiomatic (the viola can be successfully substituted in this piece as with the later clarinet sonatas, something unthinkable in the Quintet). Clarinetist William Hudgins joined Eskin and Deveau for a cool read of this piece, with all edges polished smooth and every note in place. There was something not quite connected in this performance, which might possibly be laid less at the feet of the performers than to the performance space. Having heard three concerts in the Shalin Liu Performing Arts Center, it clearly has an unusual acoustic. That the sound has unparalleled clarity seems to be agreed upon; however, that clarity comes with some sacrifice. Broadly speaking, one can speak of a deficit of resonance and brilliance, but the nature of the sounds in the hall is more varied and harder to pin down than that. The effect of the space is different depending on the instrument; the piano suffers least, perhaps because it provides so much resonance of its own. The piano sound has a round and luminiscent quality, but a blunted attack. Wind instruments retain their tone quality, but sound stark and isolated when soloing; in ensemble, the winds’ timbres are very “tight” together, there is little sense of them expanding into the hall. Strings suffer the most, lacking both brilliance and roundness, producing a sound that is both steely and weak. The hall does this without sounding underpowered—and one's ears do adjust over the course of an evening, much as one’s eyes accommodate in a dim but not dark room. But each time I return to hear the start of a new program, the acoustic obtrudes once again.

This has differing effects on different repertoire as well. On this evening, the Mozart was dominated by the piano, which is not necessarily inappropriate for the piece, although the strings seemed to recede further than one might have wished. The Martinů benefited from the clarity, as the relentless activity in the piece remained comprehensible and audible, though the strings struggled to match the presence of the wind quintet. The Brahms was downright odd-sounding: the clarinet and cello inhabited entirely different sound worlds, while the piano’s personality frequently overshadowed them (some of this can be blamed on Brahms who, in the trio as well as in the clarinet sonatas, frequently overburdens the piano). The room has a definite personality, and the effects it produces feel deliberate; it may take some time for performers to learn how best to make use of its peculiar qualities.
