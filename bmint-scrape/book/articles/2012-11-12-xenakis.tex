\chapter{12 November 2012}

\index{Xenakis, Iannis}The 20th-century saw an outpouring of classical music entirely unlike that which had been written in the previous three hundred years. The crumbling of tonality emboldened serious composers to experiment with alternative ways of organizing sound. Concert audiences have, in the main, failed to accept these pieces. Yet this music was the life’s work of deeply creative and intelligent people who were saying something about the world in which they found themselves. Finding a human connection is the main goal of Xenakis In First Person, a hybrid theater work and concert of the music of the Greek composer Iannis Xenakis, which is being presented by Alea III at the Tsai Performing Arts Center this Wednesday at 8:00 p.m. Admission is free.

The concept for the piece originated with Alexandros Mouzas, a Greek composer, and the acting texts were assembled by musicologist, composer and Xenakis scholar Antonios Antonopoulos. Mouzas and Antinopoulos are in Boston to prepare the piece, which was first presented in Athens in 2011.  It alternates scenes of Xenakis (played by actor Jake Murphy) in his study with performances of his works. The texts are taken from Xenakis’s own words, as found in his writing and in recorded interviews. Mouzas says he asked Antonopoulos to look for “texts that showed who Xenakis really is as a person. What he felt about life, about art; about his wife, about Greece, about politics.” By presenting Xenakis in his own words, the hope is that the audience will be willing to take a step closer to Xenakis’s music.

Xenakis was a man of passion and intellect, who came of age amid the violence of the twentieth century. He was, in many ways, an epitome of the twentieth century intellectual and artist. He was the child of Greek parents in Romania; he lost his mother at the age of five; when the Greek civil war came, he fought the English. He was severely injured in the war, losing his left eye. He was an architect of some accomplishment, working for a time in Le Corbusier’s studio, as well as a mathematician and later, a computer programmer.

The range of Xenakis’s music is immense, which makes it difficult to give a simple description of his style. He is closer in spirit to Varèse than to Schoenberg. Schoenbergian twelve-tone music, for all of its still-provocative radicalism, was a relatively conservative reaction to the crisis of tonality. A serialist composer is still thinking of classical music as the arrangement of specific pitches at specific times.  The order of pitches is critical to a serialist— but most listeners struggle to hear or comprehend those pitches without the context provided by harmony. The units of work in Xenakis include pitches and rhythms, but are also frequently textures or gestures -- clusters of notes, glissandi, and patterns of density. Xenakis devised his own set of principles for developing music given this change of emphasis, and he famously did so using mathematical processes, frequently involving statistical methods involving chance. This was not Cagean chance -- Xenakis once said that “all chance must be calculated” -- but it was certainly different from the authoritarian control to which serialism often aspired.

Mouzas and Antonopoulos are quick to point out that Xenakis’s mathematics were a starting point and not an end in themselves; Antonopoulos: “Yes, he used the mathematics, and the book he wrote (Formalized Music) is quite complex, but that book came out of his early work, and he almost always made adjustments to match the artistic idea he was trying to achieve.” Sam Solomon, who will be conducting Persephassa and Okho, two of Xenakis' percussion works, says that “mathematicians often find the music not rigorous enough. If you try to trace back the notes to the math, it frequently doesn't match up. He was always actively composing within what the mathematics provided.”

Xenakis wrote a wide variety of works in many media. A varied cross-section of his instrumental chamber music will be presented Wednesday night: Dhipli Zyia, is an early folk-influence work for violin and cello. Evryali, by contrast, is a monumental examination of rhythm and virtuosity for solo piano. The evening will end with Charisma, a brief but intense requiem for French composer Jean-Pierre Guézec for clarinet and cello. These pieces will be played by Yukiko Shimazaki, piano; Sasha Callahan, violin; Leo Eguchi, cello; and Diane Heffner, clarinet.

Xenakis’s methods, which did not need to privilege pitch over other elements of sound, were well suited to percussion, and the evening will also offer two very different pieces for percussion: Okho for three djembés, and Persephassa, a tour de force for six percussionists. Okho is very approachable; Solomon points out that the instrumentation, three performers playing the same kind of African drum, affords a homogeneity of sound similar to that of a string quartet. Xenakis was never afraid of pulse in his music—again, in contrast to much other music written at the same time; the effacement of pulse was a common theme in the music of mid-century. Much of Okho sticks to a pulse, finding its variety in the combination of lines among players, and in the tonal possibilities of the djembe. Persephassa could not be more different. Written for six percussionists playing a wide variety of instruments, from tympani to mouth sirens, it is a dramatic showpiece, as opposed to Okho’s gentler chamber music. It requires that the six players be placed around the performing space, making intense demands on the players when the music requires close synchronization. Persephassa is an encyclopedia of percussion writing techniques, experimenting with pulse as well as with independent tempi; with extremes of dynamic; and with the architecture of the concert hall.

In addition to the spoken texts and live performances, the evening will contain excerpts of Xenakis’s site-specific electronic pieces; he was a pioneer in this field, writing music to fill architectural spaces in Montreal, Osaka, Paris, and the ruins of Persepolis in Iran. He also designed light shows, coordinating the firing of flashbulbs and reflection of lasers with the music using paper tape and photoelectric cells. The piece includes be two original videos by Vicky Betsou, assembled from pictures of Xenakis, of his site-specific performances, and of his graphical sketches for his scores, which are frequently astonishingly beautiful.

Xenakis’s music is the product of the composer's intensely personal search for a new mode of expression. The results he produced are often unlike anything the casual concertgoer is prepared to hear. Iannis Xenakis In First Person aims to bring the unfamiliar listener to closer to Xenakis. Alexandros Mouzas says, “In Athens, I overheard a man talking to his friends at intermission. ‘Finally, I can understand this music!’ he said.”  There may be no better way to begin to comprehend this music, and similar music, by giving a composer a chance to disclose himself while surrounded by realizations of his work.

Xenakis in First Person will be presented this Wednesday at 8:00 p.m. at the Tsai Performance Center, 685 Commonwealth Avenue, Boston. Admission is free. Persephassa will be presented again (along with works by Jo Kondo, John Luther Adams, Joseph Celli and Lei Liang) on this Friday at a free concert by the BU Percussion Ensemble at the BU CFA Concert Hall, 855 Commonwealth Avenue.
