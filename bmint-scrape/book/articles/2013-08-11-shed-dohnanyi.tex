\chapter{11 August 2013}

\textsc{Boxxx}

Since the ongoing Festival of Contemporary Music has been featuring the music of Elliott Carter in Ozawa Hall, it was initially encouraging to see that a piece of his would also make it out to the larger venue and to a broader audience. Sound Fields, is hardly typical Carter, though. His only piece for a traditional string orchestra, it is an experiment in sound density, and as such lacked any of the usual sense of intellectual swiftness or lapidary variety that characterizes his best work. The composer likened the desired effect to the color field paintings of Helen Frankenthaler—interesting work, but not exactly the image I have of Carter. The dynamic remains almost constantly at a low mezzo-piano to piano, and most of the instruments play only a single pitch per instrument, mostly in the same register throughout. The musical interest of the piece is the density of the texture at any time, a density that consists both of dynamic, louder as more play, but also of harmonic complexity—as more players simultaneously play, more pitches are sounded, creating a variety of consonances and dissonances, apparently unmoored from any clear tonal center. The effect is that of a slow, atmospheric unfolding. It is the experiment of an expert craftsman, but an experiment nevertheless. The BSO played this relatively simple piece with a cool, bracing sonority, although for a piece exclusively focused on the addition and subtraction of voices there was unwelcome uncertainty in several of the entrances. Having heard so much Carter in the Festival of Contemporary Music, where the audiences have been closely attentive and where one might be forgiven for finding Carter even a little old-fashioned, it was jarring to hear how even these few minutes of his music made the Shed audience restive. A nasty upper-respiratory bug apparently afflicted Lenox in the wake of Friday’s rain.

Yefim Bronfman and von Dohnanyi seemed to have very different ideas about Beethoven’s C-minor Piano Concerto. The extended orchestral introduction set out the themes uncontroversially, with ceremony and public display. Bronfman’s entrance, though, was understated; he made the music smaller, more personal. The orchestra eventually matched this sensibility, the soloist and ensemble achieving a chamber-music ambiance together only to have Bronfman kick out the blocks and barnstorm through the cadenza, finishing big and loud. The second movement was consistently beautiful, with Bronfman remaining understated and the orchestra providing supple support to the increasingly rich adornments Beethoven adds to the melody. But in the third movement, Bronfman played the rondo theme with a subtlety of inflection that might be taken for detachment; von Dohnanyi coaxed the orchestra to life with subtly driving accents and a bit of tempo pushing, but the music proceeded in two different tracks until Bronfman again roused himself to bring the concerto to a bombastic conclusion.

After intermission, von Dohnanyi and the orchestra delivered a Brahms Fourth Symphony that was lithe and athletic, with moments of glowing beauty and emotional commitment. The sound in the Shed is infuriatingly inconsistent—at times I found the low strings undervoiced, only to hear them ring out minutes later. The sound in the winds was coarse at times, but there were moments of surpassing sweetness as well. The variability seemed too great to be laid entirely at the feet of the conductor and players, and it made for puzzling moments. Nevertheless, there was plenty to savor here—the subtly muscular rubato of the opening statements of the first movement, the long line and shifting color of the second movement. The third movement was giocoso enough to provoke a guttural “yeah!” from somewhere behind me out on the lawn, provoking laughter and a smattering of applause, and the orchestra received an extended ovation at the symphony’s conclusion even as the less hardy made their rapid way out of what had become a beautiful but surprisingly cool night.