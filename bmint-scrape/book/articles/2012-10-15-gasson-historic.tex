\chapter{15 October 2012}

Renowned violinist Jaap Schröder \index{Schröder, Jaap} and fellow members of the \index{Skálholt Quartet} Skálholt Quartet performed in a free concert at Gasson Hall at Boston College on Saturday night. After offering quartets by Haydn and Boccherini, they were joined by Owen Watkins, a Boston-area performer on historical wind instruments, for the Mozart Clarinet Quintet.

At 86, Schröder is something of a grand old man of the early music movement, holding the position of concertmaster with the Academy of Ancient Music in the early 1980s, and amassing a significant discography of historically informed performance. He may also be remembered locally as an early member of Aston Magna.

The Skálholt Quartet: Schröder and Rut Ingólfdóttir, violins; Svava Berhnaðdóttir, viola; and Sigurður Halldórsson, cello; is the by-product of a 1996 request for the members to play Haydn’s \textit{Seven Last Words of Christ} at Skálholt cathedral in the south of Iceland. They have been playing and recording together ever since.

The program opened with the \index{Boccherini, Luigi!Quartet in A, op. 32, no. 6}Boccherini Quartet in A, op. 32, no. 6, and the \index{Haydn, Franz Joseph!Quartet in C, op. 54, no. 2}Haydn Quartet in C, op. 54, no. 2. The collective sound from the players’ gut strings more than filled the voluminous 350-seat hall. The pieces presented real contrasts. The Boccherini is a genial but strongly colored collection of movements. The Haydn has a surface of seriousness but is full of moments of characteristic humor. On this evening, Schröder’s tone lacked focus and projection, at times disappearing in the more full-throated accompaniment of his peers. But there were moments of great ensemble playing, and Schröder employed end-of-phrase rubato with strong effect. The Skálholt didn’t seem to find the same humor in the Haydn that I do; this was high church Haydn from start to finish.

The focus of the concert was on the Mozart, for which \index{Watkins, Owen}Owen Watkins played on a historically accurate basset clarinet of his own making. Clarinetists both bless and curse Anton Stadler, Mozart’s inspiration for both the Quintet and the Clarinet Concerto. Stadler was one of the first great clarinetists, renowned for his singing tone, and without him these cornerstones of the clarinet repertoire would not exist. However, he was an experimenter, and the instrument for which Mozart’s great works are written are not the “standard” clarinet of his date, but for his preferred instrument at the time, what we now call a “basset clarinet.” Watkins’ basset clarinet was longer than a regular clarinet, a full major third lower than the standard clarinet, and it contained a cylindrical resonator at the end rather than the usual bell. As he walked onstage it resembled an ancient and ornate golf club.

The other reason to curse Stadler is that he was given the autograph manuscripts of both the Quintet and Concerto, and managed to lose both of them. The earliest versions we have of the pieces come from printings in the early 1800s, some 10 or more years after its composition, and they were arranged for the standard clarinet. Once you know the pieces were written for basset clarinet, you can guess at passages that would benefit musically from having an instrument with an extended lower register — places where the melodic line suddenly jumps as it hits the end of the instrument, or arpeggios that “bounce back” when they go too low. It is an irony that the arguably two finest compositions for the “clarinet” use an instrument that is not the clarinet we know; that we can only speculate about how Mozart used the extended range of the instrument, since the autographs were lost; and finally, that these two pieces are the only pieces for that unusual instrument.

Owen Watkins is a virtuoso performer on historical clarinets, oboes and recorders. He is also a member of Boston’s legendary Von Huene early music workshop and a colleague of Schröder’s. Six months before Schröder came to Boston, he called Watkins and told him they must play the Mozart together. There was one problem, though — Watkins did not have a basset clarinet! Although you can purchase a modern basset clarinet, it is simply a modern instrument that has been lengthened. True historical basset clarinets are not easy to come by, but he is a maker of instruments as well as a player, so he set to work. He took the top half an historical clarinet he had on hand, and proceeded to make his own lower joint. In six months, he had the basset clarinet we heard on Saturday.

So how did it sound? My first reaction on hearing the instrument was surprise at how much it sounded “just like a clarinet.” The apparent rusticity of the instrument had prepared me for something a bit rawer. The clarinet initially got its name from the clarion, or “trumpet”, quality in its upper register, and I half-expected some trumpeting forth, I suppose. In fact, the tone of the instrument was rather softer and less incisive than a modern instrument; it was “sweeter” than the usual modern clarinet. In the higher registers it sounded more plaintive than steely, more English horn than modern clarinet. This enabled it to both blend well with the gut strings while remaining in the front of the sound.

The modern clarinet is a triumph of metal engineering, bristling with chrome keys that could not have been made in Stadler’s time. These enable the playing of a full chromatic scale without excessively awkward fingering, while keeping most notes generally in tune. The lack of elaborate keying makes fingering the older instrument more complex, and means some notes have perceptibly different tone color. Mostly these differences were heard in only passing, though at some points these differences were more profound. In the more harmonically wayward eight-notes in the first movement development, or in a repeated half-step figure in the minuet, they gave a unique personality to the sound.

The lack of the autograph manuscript means we have no definitive interpretive notations — dynamics and articulations. The ensemble did not provide any surprises interpretively, but the clarinet’s articulations were in several places unexpected. Hearing them (and seeing what I could of the player’s right hand), it appeared that at least some of those choices were made in order to make the passage fit the demands of the instrument. It would be fascinating to know if these choices were the ones Mozart would have heard, and if it is possible that behind the somewhat fossilized and “obvious” choices one hears nowadays, there is the possibility for a much more varied range of interpretation based on the demands of the instrument itself.

\index{Mozart, Wolfgang!Clarinet Quintet, K. XXX}The performance was graceful and understated, the first movement especially successful and heartfelt. Watkins played with confidence and virtuosity despite the demands of the older instrument. The piece gives all the players interesting work, and they clearly enjoyed the give and take. The capacity audience appreciated the effort as well, granting a standing ovation at the end of the evening. Perhaps some orchestra may make it possible for us to hear this unique instrument in the only other piece written for it?
