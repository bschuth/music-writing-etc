\chapter{11 December 2013}

\textsc{Boxxx}

Sunday night MIT threw a family party for John Harbison’s 75th birthday, in the form of a celebratory concert at Kresge Auditorium. Entitled “Definitely Not Retiring,” assembled by Dr. Frederick Harris, MIT Director of Wind and Jazz Ensembles, it made manifest two aspects of Harbison’s life that are not always visible from a concert stage: the depth of his abiding love for jazz, and the centrality of his role as educator.

As it turned out, there were really only two pieces on the program that fall under our usual reviewing purview here: one piece of Harbison’s concert music, and a brief, lesser-known work of Bach. Harbison’s Cortège, for six players, is the third piece in a series of imagined music being played for composers upon their arrival in the “other world,” after their death. Coming after imaginings on the heavenly arrivals of Schubert and Stravinsky; Cortège contemplated the passing of Donald Sur, a close colleague of Harbison’s who died in 1999. The work is three movements of incantatory, hieratic music that lives in the overlap between the modern and the ancient. The first movement rings rhythmic changes on the sounding of a single interval, alternating with episodes of contrasting, fluid sound; the second pits pitched and unpitched sounds against each other; the third begins with the tolling of handbells, which is undermined by noise and scratching.  Often monothematic, it never loses interest thanks to the range of instruments and their theatrical opposition to one another—in addition to a conventional tuned and un-tuned percussion battery, the piece calls for a conch shell, a guiro, and a “lion’s roar,” as well as those handbells, for which the players donned cloth gloves. The MIT Wind Ensemble Percussion Section gave a performance both precise and full-throated under Fred Harris’s direction.

Harbison’s love for and indebtedness to Bach is well attested, and it felt right that he took the stage before intermission to conduct O Jesus Christ, mein Lebens Licht, BWV 118. The piece is a single choral movement: Harbison informed us it was properly a motet, mis-classified as a cantata. It was written for an unusual combination of brass instruments, including two mysterious ‘litui’. The wind writing suggests outdoor performance, perhaps a funerary ceremony, and text is brief and dour; the hopeful opening line is offset by the closing sentiment “on earth here am I but a guest/And by sin's burden sore oppressed.” On this occasion it was played in a weirdly successful arrangement for six players from the MIT Wind Ensemble—four saxophones and two clarinets (which took the place of the litui). The result was modestly pleasing; the saxophones filled the room with a cushiony take on baroque figuration while the clarinets warbled disconsolately beneath the somewhat underpowered voices.

Beyond these two pieces, the first half of the evening was filled with gifts for John; the second half, with John playing his own jazz compositions with faculty friends and with students. The surprises included a poem, “A Rhumba for John Harbison,” read on tape by MIT President Rafael Reif. Written in response to the piece “A Rhumba for Rafael Reif” which Harbison wrote for Reif’s inauguration, it was charming and funny, managing to pleasingly rhyme “dandy” and “moustache.” Composer Peter Child then appeared and offered his one-minute “75th Birthday Canon” for piano-four-hands with Peter Godart, a MIT student and pupil of Harbison’s. Godart then played a piece of his own, a “Bach Intervention”, which began with a quasi-baroque stream of notes that gradually thickened and darkened, creating a harmonic world obliquely jazz-like, but far from any recognizable tune. After comments from MIT Associate Provost Phillip S. Khoury, the lights were lowered and a fragile Ran Blake appeared to play a hallucinatory interpretation of Billy Strayhorn’s Lush Life.

At this point I think the audience might have been forgiven for thinking the evening a bit downbeat for a 75th birthday party—the opening and closing works openly referencing death, the surprise musical pieces all with a bit of darkness to them. But we then spent the second half of the evening with John Harbison, jazz man and songwriter, and the room livened noticeably. He is the pianist with “SIN”, “Strength In Numbers”, the MIT Faculty Jazz Ensemble that “performs rarely, if at all” according to the program. Fred Harris, the concert organizer, was on drums; Mark Harvey on trumpet; Dylan Sherry on saxophone; and Keala Kaumeheiwa on bass. As a special guest, Harbison’s wife Rose Mary on violin for two numbers. The crowd favorite among the four numbers in their short set was Sweet Pretty Baby, which Harbison told us was a more-or-less straight crib of Douce dame jolie by Guillaume Machaut (or “Gil Macho”, a better name for an American songsmith). Harbison, it should be pointed out, has a hushed but very effective style at the microphone. Since the concert programs had mysteriously disappeared earlier that day, he was often called upon to give introductions of his pieces, and he did so with a sense of humor that was sly and self-deprecating. BMInt is pleased to disseminate that missing document here.

After SIN’s set, members of the MIT Vocal Jazz Ensemble joined them to perform eight more Harbison songs. Unfairly labeled “rejects” by the composer, they were tunes that lacked lyrics, so Harbison set the Ensemble students to writing them. All eight songs had student-composed lyrics, frequently sung by lyricist. In all of this music, Harbison and his students achieved an uncanny historical ventriloquism – the music is never way “out there”, and the lyrics written by the young singers could have been easily have been written 75 years ago (according to the composer they were fed a diet of Ira Gershwin, John Mercer, Oscar Hammerstein, and Dorothy Fields). You could imagine any of these songs emerging from a scratchy AM wireless on the set of an RKO Radio Picture. They don’t break ground in American popular song, instead, they renovate the existing edifice, adding modestly to an aging monument. Harbison’s work in this genre isn’t as consistently interesting as his concert music, but I’d be more than happy to pay a cover and two-drink minimum to hear a few sets.

The evening ended with two versions of “Happy Birthday” wrapped around “Aunt Hagar’s Blues”. The first “Happy Birthday” was comedic and bubbly, alluding to the composer’s love for Italian food; the second was a bit grandiose, written by Peter Godard and involving some weighty introductory effort on the Kresge’s Holtkamp organ. The blues was set up as a showpiece for the Vocal Jazz Ensemble, which collectively produced beautiful close harmony, and individually, scatted their ways over the changes before bringing the evening to a brilliant end.

During all of this, Harbison sitting in the back at piano provided the image that summarized the evening for me: the celebrated composer unobtrusively providing accompaniment, and support, as his students sang his songs, to which they have added their words. He was given a closing ovation that was warm, but understated; by the end it felt like an intimate kitchen session among friends, and the crowd took its time dispersing into the night.
