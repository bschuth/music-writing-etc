\chapter{20 October 2012}

``They just played Pictures at an Exhibition. It was amazing. This sentence may be all you need to know about the \index{Marine Band}Marine Band concert at Symphony Hall Friday night. That, and the fact that it was a text sent by the adolescent boy sitting in front of me. The Marine Band, “The President’s Own”, is the nation’s oldest professional musical organization, and it knows its business well. Sixty well-drilled players with fantastic technique consistently deliver a popular, uplifting experience calculated to amaze. This free concert drew a large audience that was younger, more varied and more enthusiastic than I am used to seeing Symphony Hall. The presentation was that of a Pops concert – the first “piece” was a conductorless fanfare that would not have been out of place at an awards show, which ushered on and then underscored GySgt (The rank titles presented here are taken verbatim from the program. I trust others with more knowledge of the military will know what they mean.) Kevin Bennear, credited as the “concert moderator”.

\index{Sousa, John Philip}What the Band it does well, it does extremely well. The Marine Band playing Sousa is a tautology – the one is the sound of the other, like the Vienna Philharmonic playing waltzes.  They offered two inevitabilities, “Semper Fidelis” and “Stars and Stripes Forever”, which were thrilling. They played one rarity, “The Aviators,” which was instantly forgettable, despite being played with the same skill and intensity.

\index{Mackey, John}The Band performed two “modern” works, modern only in the sense of having been written recently. John Mackey’s Asphalt Cocktail (2009), transcribed by MSgt Donald Patterson, was not a success. GySgt Bennear, using the composer’s words, described it as a tonal depiction of “the scariest NYC taxi ride you can imagine, with the cab skidding around turns as trucks bear down from all sides.” He left out another phrase the composer uses in his notes to the score: “That title screams Napoleonic Testosterone Music. I was born to write that!” Yes, that was about right. It was simply too loud and congested to make a coherent impression. When the end of your piece features a percussionist tossing around a large aluminum garbage can and you can’t hear it, something has gone awry.

\index{Gandolfi, Michael!Flourishes and Meditations}Michael Gandolfi is the Chair of the Department of Composition at the New England Conservatory, and was in attendance for the Band’s performance of his piece Flourishes and Meditations on a Renaissance Theme. Commissioned by the band, this was an attractive set of neoromantic, tonal variations on a Spanish theme from Rodrigo. After the frontal tutti assault of the Mackey, Gandolfi’s much lighter touch was a relief. Much of the piece was either antiphonal, or constructed of ostinatos under the chorale-like theme. Here and there were occasional surprises – a small eruption of dissonance here and there, a moment of brief structural disintegration near the end which made me hope for more exciting transformations, but which quickly subsided. He had the courage to end his piece quietly, alone of all the works on the program. It was pleasant, professional and beautifully crafted, but slight, and the audience became audibly restive as the performance stretched past ten minutes.

\index{Mussorgsky, Modest!Pictures at an Exhibition}The major piece on the program was Mussorgsky’s Pictures at an Exhibition, orchestrated by Ravel, transcribed by Paul Lavender. That is correct – Ravel’s orchestration merely “transcribed” for band. I love the sounds of bands and wind ensembles, and I have little patience with those who belittle them. So I am frustrated when bands present European classical music in slavish arrangements meant to reproduce some familiar piece rather than exploit the unique possibilities of the instruments provided them. The Mussorgsky was exactly this, a respectful, inoffensive and uninspiring rote reproduction of an overexposed piece. Colonel Michael J. Colburn’s direction was metronomic, and his tempos were far too rapid for the slow movements, as if to avoid boring a restless audience.

\index{Mendelssohn, Felix!Concert Piece No. 2, op. 114}The most successful work of the night, apart from the incomparable Sousa, was Mendelssohn’s Concertpiece No. 2 in D minor, Op. 114, a “duo concerto” for clarinets featuring MGySgt Lisa Kadala and MGySgt Jeffrey Strouf as soloists. This is not great Mendelssohn – which means it is tuneful, well-constructed and ingratiating, if not touching or thought-provoking. It is in three quite brief movements, fast-slow-fast, each of which ends at just the right time as the material exhausts itself. The transcription, by Thomas Fox, reduced the 60-person ensemble to 20, half of whom played clarinet. The result was charming; the sound of massed clarinets properly deployed is a beautiful thing too rarely heard. The soloists were note-perfect and pitch-perfect, and were content to simply place the notes in front of us, rapidly and in tune and barely touched with interpretation. It was not moving or thought-provoking. However, it was amazing.
