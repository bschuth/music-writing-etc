
\chapter{16 November 2012}

Conrad Susa’s opera Transformations is being presented at the Boston Conservatory Theater this weekend.  Written for eight singers and eight instrumentalists, Transformations sets most of Anne Sexton’s book of poetry by the same name, retelling tales from the Brothers Grimm. The poems retain the outlines of the familiar stories — Snow White, Rapunzel, Rumpelstiltskin, etc. — but contain unexpected cruelties, sexual confusions and disturbances, and an obsession with death.

Having encountered the poems first, they seem unlikely texts for an opera. The stories are told in the third person. The language is complex, demanding close reading. The imagery is startling yet the style cool. Susa has set the text as Sexton wrote it, and mostly succeeds in writing vocal lines and orchestral parts that preserve the words, at the cost, though, of making much of the music merely pleasing rather than memorable. A pastiche of American popular styles combined with operatic voice production, the piece occupies an unusual place between traditional opera and musical theater. There are few significant set pieces, notably a monologue for Sexton herself. Otherwise the stories are told by the ensemble, in an unbroken line of spoken recitative. This is not at all grand opera; but likewise, it poses challenges that make Sondheim’s Into the Woods seem juvenile.

Conductor Andrew Altenbach guided the singers and players effectively through the varied demands. The un-amplified sound of the players in the pit as well as the singers on stage projected well in the dry but clear confines of the Boston Conservatory Theater. Yet, the amplified monologue of Sexton, performed in a more crooning pop style by Christina Pecce had more immediacy in the space.

The cast I saw Thursday night tackled the challenge of the piece with aplomb (there are two casts; this cast will perform again on Saturday). Catherine Malfitano, who made her breakthrough in role of No. 2 (Anne Sexton) in the premiere production at the Minnesota Opera 1972, acted as artistic advisor.  Christina Pecce, in that role, gave perhaps the most polished and engaged performance of the evening.

The set by Julia Noulin-Mérat and lighting by Carl Weimann were simple but ingenious. Most of the opera took place in a wood, the stage covered with leaves, with only a picnic table and some institutional chairs as props.  Sonotubes plastered with newspaper gave the impression of birches, but they were themselves transformed to make Rapunzel’s castle, or to make the diminutive scale of Rumplestiltskin concrete. At the end of the acts there was an indication that the play was taking place in a mental institution (a common setting for productions of Transformations), but this was perfunctory. The direction by Nathan Troup emphasized the physicality of the actors, whose bodies were responsible for shaping the space, and whose movements and postures illuminated the texts that were being sung. The singers attacked the challenges of the piece with bravery and confidence, and the acting and the singing were both excellent.

Danielle Lozano and the ensemble (Max Wagenblass photo)

While the texts were surprisingly clear, they were not universally so, and several moments in the opera were blunted when the texts could not be made out. The stage pictures during the “White Snake” were striking, but their purpose obscure; the lesbian embraces that began the “Rapunzel” section are accompanied by wordless singing that obliterates the crucial texts that explain the nature of the relationship. Sexton’s language favors unusual imagery and language that is sarcastic or lacerating, replete with anachronistic, ironic similes; in this production, too often these moments were played for laughs, which risked turning the poetry into clowning. The audience seemed to enjoy the amusement, but the comedy undercut the devastating twists that arrive at the end of each story, and blunted the revelation that ends the piece.
