
\chapter{2 December 2012}

The Longwood Symphony Orchestra presented a surprising and satisfying program at Jordan Hall Saturday night. Celebrating its 30th anniversary, this unique ensemble comprises primarily musicians who are also medical professionals. Nevertheless, their artistry is ambitious. The evening began with a pair of unfamiliar 20th century American works, and closed with the Beethoven Violin Concerto featuring Benjamin Beilman as soloist.

In pre-concert remarks, Music Director Ronald Feldman emphasized the value he places on American music. The two American pieces he presented were “melting pot” Americanism– Ellen Taaffe Zwilich’s Concerto Grosso is rooted in the German Baroque, and the Two Sonnets by Michelangelo that followed were by the naturalized Czech, Karel Husa, inspired by the famous Italian. The pieces contrasted sharply: the Zwilich aspires to the spirit of Handel as well as to his style, while the Husa is somewhat more forbidding.

Concerto Grosso incorporates and transforms material from the first movement of Handel’s D major violin sonata, pitting it against figures of Zwilich’s own in five short movements. The music, for reduced orchestra plus harpsichord, is broadly tonal. The first movement was reminiscent of Britten, the second of Shostakovich, and the evolution of the material was easy to follow. It is in the “tradition”, if you will, of Lukas Foss’s Baroque Variations or Alfred Schnittke’s own Concerto Grosso \#1, while avoiding the gleeful destruction of the first or the aggression of the second. Feldman brought out the muscle in this piece without sacrificing balance — the harpsichord is critical to the sense of texture, and it was never lost in the mix. Before the piece began, concertmaster Sherman Jia and harpsichordist Leslie Kwan (on loan from the L’Academie Baroque Orchestra) performed the sonata movement of Handel’s with grace and emotion, giving those unfamiliar with the piece a chance to hear the inspiration for it.

Karel Husa’s best well-known piece is his Music for Prague, 1968, and the Two Sonnets by Michelangelo, were written just a few years later. They share a similar language: extended clusters of tones provide a background for fragments of melodic material, frequently in the high winds or in pitched percussion, and he is not afraid of brass. The two sonnets in question are “La Notte” (The Night) a night-music meditation for large orchestra featuring sensuous fragments of melody from the saxophone (beautifully played by Karen Cubides); and “A Dio” (To God), a rather raucous essay for the entire orchestra that builds to an impressive and frightening climax before subsiding.  As with the Zwilich, we were offered a chance to hear the inspiration for these pieces: Bill Barclay, director of music at Shakespeare's Globe Theatre was on hand to read the sonnets in translation (he is also a composer, and his score for Hamlet was recently in Boston at Arts Emerson in October).

It was wonderful to hear a non-professional orchestra tackle unfamiliar works, and to do so with such energy and commitment (not to mention large forces — the stage was filled to the apron for the Husa). Feldman brought conviction and confidence to the performance. I found the Zwilich pleasing, if slight — at a short distance, I cannot recall much of the piece beyond the Handel phrases from which it was constructed — and the Husa uneven, although “La Notte” was quite haunting, and deserving of a second listening.

The second half offered a totally different and unrelated experience:  the presentation of a warhorse with a promising young soloist. Benjamin Beilman graduated from the Curtis Institute of Music just this past May, and he came to Jordan Hall and the Longwood having already had his Wigmore Hall solo debut as well as having released a recording of the complete Prokofiev sonatas. His performance was technically brilliant, with a round and focused tone that had no trouble filling the hall and sailing over the orchestra. This was an exciting and athletic performance, as well as crowd-pleasing, drawing a standing ovation from an audience that was sitting on its hands during the first half. Mr. Beilman has skill to burn, and is often quite anxious to get on with burning it. The first and third movements had an attacking edge, and he seemed most comfortable when he could really lay into the virtuoso passages. He did not lack for sensitivity — every phrase was shaped with intention and clarity — but it was the more aggressive moments that seemed to capture his attention. If this is the way he plays Beethoven, I have no doubt that Mr. Beilman’s Prokofiev is electrifying. This approach served the second movement less well, where the overall feeling was one of impatience.

Mr. Beilman presented his own cadenza in the first movement, an unannounced surprise that had me scrambling for my notebook to gather impressions. It was idiosyncratic, spinning off Paganini-esque passagework that I imagine would have made Beethoven start. Halfway through, the timpani joined in, repeating the famous five-note pattern that starts the concerto — this caused the violinist to play something approximating a folk tune over the ostinato, before freeing itself for a few more arabesques.  I found this delightful, if a touch bizarre. He also interpolated cadenza material from a rarely performed arrangement Beethoven made of the concerto for violin and piano, moments which similarly surprising if not quite so unusual. Coming into this concert, the last thing I expected was to be surprised by the Beethoven, and I thank Mr. Beilman for that gift.
