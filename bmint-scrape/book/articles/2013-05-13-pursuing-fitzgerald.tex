\chapter{13 May 2013}

\textsc{Boxx}

The Great Gatsby is the great elusive target of American literary adaptation, constantly pursued but never caught. Iconic and beloved, with a cool shell of distance wrapping a gooey center of pure romantic idolatry, it is so tempting, but I’m in the camp that believes any attempt to turn the novel into performing art is doomed from the start. This weekend has not disabused me of that opinion. However, both Baz Luhrmann’s art-direction orgasm of a film and John Harbison’s rather more sophisticated opera succeed in producing something of value as a result of their pursuits, even if what they created is not quite a complete success.

Harbison’s opera has taken far too long to get to here—Emmanuel Music's Orchestra and Chorus presented its Boston premiere Sunday at Jordan Hall under the excellent Ryan Turner. Staring life in 1985 as a concert overture, it was expanded into full opera in response to a commission the Metropolitan Opera. Harbison’s Gatsby first saw the light of day in December, 1999, and was last performed in New York in 2002. In 2008 Ensemble Parallele in San Francisco requested a reduced scoring, and over the years Harbison has revised the original, mostly making cuts suggested during the first run. The version we heard Sunday as the “provisional final shape for the opera” was the revision of the full scoring.

The novel fits a little awkwardly into traditional operatic conventions. At the center is a love triangle: Jay Gatsby and Tom Buchanan both love Daisy Buchanan—Tom is her husband—Jay the love of her life from the past. Jay has recreated himself as a fabulously rich man in order to win Daisy. A soprano, Daisy is not the tragic figure opera often requires—it’s Jay, who suffers the consequences of his hope and desire. There is a tragic female character, but she’s Tom’s mistress, Myrtle Wilson. Jay’s end comes not at the hands of Daisy’s jealous husband, but shot by Myrtle’s husband George who thinks him responsible for Myrtle’s hit-and-run death. It’s something of a muddle, but there’s operatic potential there. However, the novel’s greatness lies not in melodrama, but in the tension between those histrionics and the character of the language, the way the first-person narration of Nick Carraway both involves us and leaves us critical distance. The great moments of the book are journalistic, not operatic. The second most violent moment of the book is described almost monosyllabically: “Making a short, deft movement, Tom Buchanan broke her nose with his open hand.”

In the opera, the emotional extremity is embodied in the orchestral writing, which is kaleidoscopic, almost wild, expressionistic, with romantic swooning, dissonant climaxes, shot through with masterful ventriloquisms of 1920s jazz songs. The songs emerge from radios or from the band at Gatsby’s parties, and Harbison does such an excellent job imitating the period that you’d be forgiven for thinking them quotations. The lyrics, by Murray Horwitz, are also lovely, more literary than authentic twenties songs, but affecting and thoughtful. The overture offers in miniature the musical tactics of the opera as a whole: the songs are juxtaposed against the “operatic” writing, and then often the two intertwine. There’s an element of pastiche here, but the songs are more than a period distraction; as the opera progresses you can hear their melodies’ ghosts haunting the other music. That “other” music is a glory of its own—even in this reduced scoring the stage was full of wind, brass and percussion—the variety of colors that emerged was astonishing. The composer is willing to be programmatic, evoking wind, rain, trains, car travel; there’s canny use of pitched percussion. Harbison’s musical voice emerges in coloristic sensibility and in the handling of rhythms. There’s also some quotation, or stylistic tribute: I thought I heard the Gershwin of “Summertime” in an early duet; Britten in some of the exchanges between the men.

Devon Guthrie as Daisy Buchanan, as Gordon Gietz as Jay Gatsby (Julian Bullit photo)

The vocal writing, in contrast, is prosaic to the point of frustration. Several times the opera prepares you for a big set piece, an aria or ensemble, only to leave you waiting for the voice to open up the way the orchestra does. Perhaps this contrast was meant to achieve the narrational distance of the novel, but it felt like a tease, and it ends up courting boredom. When the parts do expand melodically, they can be oddly aggressive—when Daisy sings about missing the “old warm world” she ascends almost to coloratura heights, which is at odds with the longing of the text.

Harbison wrote his own libretto, and despite the revisions made since 1999, it still feels in need of pruning. After a brisk opening, the piece slowly begins to lose energy; by the second act your realize just how much exposition there is in the novel, how much information needs to be communicated to understand what has brought the characters to their crisis. By the end of the piece, there was a sense of exhilaration—but also of exhaustion. This slim novel conjures up a rather heavy opera.

The opera was presented in concert. The singers were appropriately attired: the rich and idle in summer suits and glamorous dresses, George Wilson in work clothes, Myrtle in a tawdry flapper dress. Entrances, exits and occasional movement on stage were minimal, but serving to clarify exactly who was talking to whom, and how intimate were their conversations. I found this remarkably effective, to the point where I wondered if a full production might be more distracting than illuminating. The texts were presented on screens on either side of the stage, and were necessary to follow the action.

Vocally, it was the men whose parts stood out. There is a gender imbalance in the novel. The obsessions and desires of men drive the action. Women are primarily objects of desire, to be wanted or possessed. In the opera, men generally hold the stage and control the action. In performance, the women all struggled to be heard over the orchestra; this was particularly the case with Devon Guthrie as Daisy, whose part would have been quite difficult to follow without supertitles. Daisy is a difficult nut to crack in any case, and despite some fine singing, she remained inscrutable. Gatsby and Tom Buchanan are dueling tenors, brought to life by Gordon Gietz and Alex Richardson, respectively. Their casting was excellent, Gietz’s athletic and lean dramatic tenor contrasting with the muscular, dark voice of Richardson’s threatening Buchanan. Nick Carraway’s baritone is an interesting compositional choice; in the recording of Gatsby from the Met, Dwayne Croft’s voice threatens to dominate the proceedings. David Kravitz perhaps has a surer dramatic sense, for while his voice carries the necessary weight and stability required for Carraway, his unprepossessing characterization ensured that Carraway never overtook the stage. Of all the performers, Kravitz consistently best found what opportunities there were in the vocal writing, and also conveyed a dry wit that ensured every humorous moment landed with just the right amount of emphasis. Kravitz and Krista River as Jordan had excellent sense of timing and connection, animating what is something of a superfluous additional romantic relationship. The vocal discovery of the evening was David Cushing as George Wilson. His rich bass filled the hall and brought dignity and possession to this car mechanic, despite the dropped g’s of his working man dialect. I don’t know if this is Harbison’s intention, but as embodied by Cushing, Wilson becomes a heroic figure, his sense of loss at Myrtle’s death becoming the one authentic, direct emotion, his solemn calling to God for Gatsby seeming oddly moral. There are a handful of smaller character players—James Maddalena’s Wolfsheim stood out for his direct and simple portrayal that nevertheless had an air of evil about it. Charles Blandy was the Radio Singer, projecting the words to those 1920-style melodies through a megaphone, which gave his tone just enough “radio” without obscuring the text.

Harbison could not have asked for more passionate players. The expanded orchestra of Emmanuel Music under Ryan Turner reveled in the possibilities the instrumental writing afforded. Special kudos are due to the players in the stage band, who invested Harbison’s polished and sophisticated impersonations with joy and humor. The Emmanuel Chorus, decked out in 20s costumes, brought brio and high spirits to the party scenes, both in their music and in their vivid interactions with one another.

Harbison’s Gatsby is an imperfect adaptation—it remains faithful to the intentions of the novel, but the mere fact of its being an opera distorts its material. In return, the attention to the original source makes the opera less effective in purely operatic terms. However, there are immense pleasures in the experience—it may not be heard in an opera house very often, but will be worth seeking out if it does. Happily for those who wish to hear it for themselves, this production will be performed one more time, at Tanglewood on July 11.

Ed. Note: This review has been edited in response to readers' comments.