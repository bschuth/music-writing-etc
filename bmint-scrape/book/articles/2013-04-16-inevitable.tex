\chapter{16 April 2013}

\textsc{Boxof}

As part of the Cambridge Science Festival, Dinosaur Annex gave five world premieres at an unusual venue, the MIT Science Museum in Cambridge, in “Hi-Fi-Sci”, a “conversation between science and contemporary composition.” The conversation on Saturday night was an awkward one, with science and art not exactly knowing what to say to each other.

The composers were given a doubly difficult task. They were asked to review videos in the MIT archives that showed various scientific experiments, processes and demonstrations. They were then to prepare music that would be performed while the videos were projected. They were invited to edit the videos in whatever manner they desired. The first difficulty was simply to understand what the videos represented. Four of the five confessed either in their program notes or in comments before the audience that they did not really understand the science behind the material they had chosen. The second difficulty was the challenge inherent in any music/video combination: should the two media share interest, or one or the other predominates? Should the music react directly to the events on screen? The works in the first half of the concert chose to take the videos as raw inspiration, engaging the science glancingly, reacting aesthetically to the images. The compositions in the second half presented extended videos that could stand on their own merit, with music that complemented and underscored them.

Kurt Rohde’s This Is How It Works for viola and electronics opened the evening. Rohde was the most diffident about the project: faced with 18 pieces of video of widely varying content and image quality, he said he “edited the videos into something that would be appropriate for the kind of music I write.” The videos, produced by Dr. Alfred Goldberg, Dr. Tom Kirchhausen, Dr. Janet Iwasa and Dr. Irving Epstein, ran a gamut from scintillating red and green pixels to gray interference waves and elaborate videos of clathrin proteins forming spheres and then dispersing. Rohde’s music was a dense and dark texture of electronics with the viola alternately muttering and interjecting into the sound space. The overall effect was to make the video sinister: as the clathrins suddenly aggregated, I found myself repelled by their alien character (they look a bit like the three-legged aliens in the old War of the Worlds movie), although what they were doing was benign: creating shelter for the transport of material through a cell’s cytoplasm. John Mallia’s Dangerous Passing had a similar approach, taking ultra-slow-motion films from Harold Eugene “Doc” Edgerton and editing them together, producing short snippets of images, sometimes very heavily posterized or otherwise processed.  Neither the music (pointillist with occasional romantic outbursts), nor the video was compelling enough to hold interest on its own; the two media seemed to be talking past each other. Flutist Sue-Ellen Hershman-Tcherepnin, violinist Daniel Stepner, cellist David Russell, percussionist Jonathan Hess performed, with their sounds electronically processed by the composer.

Kate Soper’s Top used different techniques to foreground her music—literally so, in that she performed as the soprano soloist, and alone among the performers stood in front of the visual example that she had selected. The video by Dr. Zvonimir Dogic, showed the slow aggregration of proteins in brain tissues, a pretty but very slow developing image that was wallpaper to Soper’s performance. The video repeated three times, and each time the very short story “The Top” by Kafka was told on tape, as Soper sang and vocalized over it while a flute and cello accompanied the texts. This fable, about a philosopher who stops children’s top mid-spin to glimpse the truth of the universe, is first edited to ribbons. As the electronically altered fragments fly by, Soper echoes bits and pieces, a word here, a repeated phoneme there, occasionally resting on a pure sustained tone. The second time the story on tape is more audible, and longer tones predominate, and by the third time one can (mostly) hear the complete story, the voice now sings nothing but pure tones. The coalescence of the music materials very broadly mirrored the spontaneous assembly of the proteins shown on the video. In performance, the video was easily ignored, but as a piece of music, this was a pleasant minature. Hershman-Tcherepnin on flute and David Russell on cello joined Soper, with John Mallia running the electronics.

Both Peter Child and Tamar Diesendruck chose to use videos in more or less unedited form. Diesendruck chose an interaction from Terry Winograd’s early artificial intelligence experiment SHRDLU as her subject for her piece SHRDLU dialog. SHRDLU was a computer program that could parse a subset of English, and that knew how to describe and manipulate a simple world made up of geometric shapes: blocks, boxes, pyramids. A video monitor would display the simple virtual world it inhabited, and as it responded to statements and questions from the operator it “moved” objects as requested. Diesendruck’s music alternated between “moving music” that appeared when objects were being manipulated, sounding a bit like a hydraulic crane; and “text music”, which took the texts that were typed into SHRDLU and its responses and translated them into an intervallic language—the number of pitch events in a given stretch of this music was equal to the number of vowel sounds in the texts. This made for an appealing, if not intrinsically interesting, soundtrack to the video. The music was distinctly subsidiary to the video—one had to read and follow the conversations on screen, and close attention to the music was not possible. The overall effect was mildly humorous, as the text music had a kind of playfulness and the arch answers from SHRDLU were often funny in themselves. The music was played by Stepner, Russel, Hess and Katherine V. Matasy on clarinet and Mallia on electronics.

The final piece was Peter Child’s Seeing the Unseen. It is a soundtrack to a somewhat famous 1936 film by Doc Edgerton, showing the same high-speed stroboscopic photography used by Mallia, but unlike Mallia, Child lets the film play straight through, without editing or processing. If you’ve ever seen slow motion pictures of milk drops splashing up in circular crowns of droplets, you’ve seen frames from this film. Child has added a quasi-minimalist background, which much like Soper’s work moves from rapid change to longer and slower development, although in this case the composer is reacting to the time-stretching present in the ultra-slow-motion images. While the music is not ambitious, it suited its material well, developing slowly enough that one can pay attention to its transformations while simultaneously attending to the images on the screen. It is excellent silent movie accompaniment, maintaining musical interest making the familiar images of dripping milk, falling cups, and hummingbirds look new, restoring a sense of wonder to the very idea of slow-motion. It was the piece most in tune with its source material. The music was played by Herschman-Tcherepnin, Matasy, Russell and violinist Gabriela Diaz. This was the one piece that was not significantly altered with electronics, and was the one where the skill and artistry of the Annex players was most evident.

It was an interesting evening despite the sense of disconnect between the composers and their required material. It was frustrating to hear composers, some of whom write music that requires significant knowledge of contemporary music history and a broadly educated ear, so blithely confess ignorance of the science underlying their source material. Perhaps this experiment might be worth trying again, but emphasizing the content of the science, rather than its visual representations.
