\chapter{30 January 2013}

\textsc{Underfull vbox?}

The sight of musicians carrying sheaves of taped-together photocopy and manuscript onto the stage holds out the exciting promise that you’ll hear something that is about to speak for the first time. That promise was fulfilled Sunday night at Harvard's Paine Hall, when Dinosaur Annex presented a recital of recent works, including one world premiere.

That premiere was the high point of a remarkably strong program. In Thin Air, for violin, piano and percussion was composed by Yu-Hui Chang, Associate Professor at Brandeis and Co-Artistic Director (with Sue-Ellen Hershman-Tcherepnin) of Dinosaur Annex. Chang tried to give some sense of the import of the work, both in her program notes and in a brief address to the audience before the performance. Her opaque capsule summary was that the piece was “about the sense of unfulfillment regarding things intangible.” A more programmatic description was also offered: that the first movement expressed anxiety in the face of incompletion; the second, the “anchor” of the piece, was about yearning, reaching out without attaining; the third, an introspective “deep breath,” relaxing, accepting that life “may be beautiful.” The first movement was built over a constant fast pulse, established by percussionist Robert Schulz on “earth plates”, flat metal pieces that provided a light, hard, high-pitched rhythmic structure, over which violinist Gabriela Diaz played brief groups and clusters of notes moving on and off the pulse. Anxious this music may have been, but its rhythmic and motivic structure was intense and purposeful; this was a clear-headed, focused anxiety. The second movement featured the violin playing long lines that were not conventionally yearning; they would move half-step-wise before allowing themselves greater compass; but to my ear they never unburdened themselves, always turning their yearning inward. The piano and percussion were dark background to the already dark lines of the violin. At times the bass drum emitted an eerie and disturbing wail, produced by dragging rubber mallet-heads across the surface of the drum. With all respect to the composer, I found most of the third movement not at all relaxing, but even harrowing – the violin, having tried to speak its mind and failed, spends much of the first half of the third movement muttering repeated figures to itself, as if it had withdrawn from the fight and was struggling just to keep itself together; the other players assume the work of keeping the piece going. Eventually material from the earlier movements return, and by the end the violin is able to speak again – but its return felt provisional, and a sense of uncertainty remained as the music ended. Diaz played with personality and aplomb; Donald Berman on piano was an ideal accompanist, allowing his instrument to comment and support the violin without intruding upon it. Schulz and the composer created a fascinatingly varied sound world with a fairly small battery of cymbals and drums.

Before In Thin Air concluded the concert’s first half, we had heard two other works. Annie Gosfield’s (b. 1960) The Harmony of the Body Machine was written in 2003 for Jean Jeanrenaud, the cellist with the Kronos Quartet until 1999. The piece is written for solo cello and processed recordings of industrial sound. In addition, the recordings included electronic manipulations of Jeanrenaud’s playing – her actual performance was eliminated, but the ghost of her manipulated sounds remained in the mix. Gosfield has written quite a few pieces inspired by sounds which she arranges and modifies. The Harmony of the Body Machine is an odd kind of bagatelle compared to some of those other pieces (seek out, for example, Flying Sparks and Heavy Machinery). The cello plays lines of varying length, frequently using extremely high harmonics that skitter rapidly. The mechanical sounds are rhythmical and dampened, providing a pulsing cushion of pitched and non-pitched sound to underpin the cello. It was interesting to hear how the technical sounds altered the received experience of the cello—the cello’s lines were not distinctive enough to have borne listening to unaccompanied, but provided passing interest against the churning electronic background. However, despite the expert and assured playing of Rafael Popper-Keizer, it felt like a sketch rather than a completed piece. Also, there was something alienating about watching this piece: at several moments Popper-Keizer just stopped playing. The sight of a human musician waiting for the machine to give the sign to move on while we all watched him was at odds with the warmth and connection that was visible when the members of the Annex were playing together.

Ricardo Zohn-Muldoon is a Mexican-born composer whose principal teacher was George Crumb. This gave his Jácaras from 2006 a particularly piquant tension. Written for violin, cello and piano, and played by Diaz, Popper-Keizer and Berman, it followed the Gosfield. I must confess that the opening of the piece did not encourage me. Hearing a fragmented statement of jagged melodies, soon accompanied by the pianist reaching into the instrument, I was afraid we might be in for a rather forbidding stretch, despite the program’s informing me that “jácaras” were popular ballads. The piece is structured as a set of “variations around ideas extracted from a song cycle” that the composer wrote in 2003, each variation titled with the Mexican popular form it was meant to evoke (alas, the program did not offer us these titles). After struggling at first to find some purchase with the variations, I found I was hearing rhythms and harmonies as if through a “haze” of the more challenging material, as if those ballads were distant memories that barely were there, even as those memories provoked the sounds that were in the foreground. There was one quite astonishingly beautiful slow, high, quiet, atmospheric variation that took me by surprise. One of the drawbacks of attending a concert of entirely new music is that you have to hold on to dear life to try to hear what the composer is saying, while struggling to find your own reaction to it. I would very much have liked a second chance at Jácaras.

The music that followed intermission was a bit more conservative in flavor. Daniel S. Godfrey’s (b. 1949) Luna Rugosa from 2003 was a fresh breath of air from about 1915. According to the composer, the lush and contrapuntal piece evokes, “the fluid, ever-changing outlines of the mirrored moon, the erratic beauty of wild roses on a rugged shore, the moody solitude one feels by the sea at night”; it does so ravishingly. Keizer and Berman were joined by clarinetist Diane Heffner and Sue-Ellen Hershman-Tcherepnin on flute in a hushed performance that was both rigorous and dream-like.

For the final piece, Stephen Stucky’s (b. 1949) Boston Fancies, all of the previous players were joined by Anne Black on viola, and were conducted by Jeffrey Means. Though the oldest piece on the program despite having being written in 1985,  there is nothing especially Bostonian about it , save that it was written for Boston Musica Viva. The “fancies” are three slow episodes in this seven movement piece, placed between four “ritornellos”. The music is busy and full of invention, and the composer’s use of instrumentation and tone color helps keep its structure and argument clear. Just to give the most obvious example, the opening 12-note theme might be hard to track if played on a piano, but stated multiple times using the same combination of timbre and dynamic, it settles into your mind comfortably. Shifting through different moods and textures, there are: rapid interlocking dialogue, rich long-line string ensemble, an episode of obsessive repetition for piano and marimba, and even some moments of apparent free play, as the conductor dropped his hands to his sides while the music continued. Almost as soon as you notice material reappearing from the opening, the piece ends abruptly, as if Stucky suddenly tired of his own skill and decided to end the game early. Compared to the other pieces, Boston Fancies might have felt a little slick, a little too much taken with its own inventiveness, but it was made a fine showcase for the players of Dinosaur Annex. They infused the piece (and the entire evening) with the joy of playing together, of tossing ideas back and forth, of deep, mutual listening; of interconnection.
