\chapter{22 February 2013}

\textsc{overfull vbox}

Henry James’s short story Owen Wingrave is, at first glance, an unpromising subject for opera. The story is simple: Owen is the scion of a family which has, for generations, found its identity in military service. While studying at a military “cramming” establishment run by Spencer Coyle, he decides that he “despises” military service and decides to abandon his training. This brings him in direct conflict with the Wingrave family, a formidable lot indeed. The story is told elliptically, mostly through the eyes of Coyle, who begrudgingly begins to admire Owen’s principled stand. By the end it has a taste of Gothic horror—the Wingraves (note that surname) have more than military catastrophes in their past, Owen might be reclaiming as well as renouncing a principle, and a mysterious tragedy brings the piece to an abrupt close.

In 1970, Benjamin Britten nevertheless composed an opera based on Owen Wingrave. He clearly was not interested in James’s storytelling craft or in the story’s strange passages through multiple genres. Instead, in it he found a platform to express his own pacifism in the face of the Vietnam War, a vehicle for his unique musical and dramatic talents. Britten’s Owen Wingrave is a strange beast—composed initially for television (the original 1971 production is available on DVD), it is a coarser and blunter achievement that James’s story (which, to be honest, can be reasonably classified as a fascinating but flawed work). The opera speaks in Britten’s instantly recognizable voice, both in its music and in the concerns to which he and librettist Myfawny Piper bend James’s text. The Boston University College of Fine Arts School of Music Opera Institute and School of Theater are making it possible for Boston audiences to see this opera produced as well as one might hope, opening Owen Wingrave Thursday night at the Boston University Theater.

Musically, the piece has the distinctive sound of 1960s and 1970s Britten. While it is not properly tonal, it is not aggressively dissonant. Its ambivalent way with harmony can end up sounding a bit monochromatic, but only a little additional attention reveals that constant change and evolution of musical material. The opera has few set pieces—or rather, it has a plethora of them, as the ensembles group and regroup constantly and as individuals step forward for episodes to sing solos that are perhaps too short to be recognized as arias. The music was rich and varied, even though these performances use David Matthews’s reduced orchestration.

Making up these varying ensembles are a small handful of characters—in addition to Owen and Coyle, there is Lechmere, a fellow-student of Owen’s; Coyle’s wife; his aunt who raised him, Miss Wingrave; Mrs. Julian and her daughter, Kate, with whom Owen has some youthful history; and his terrifying grandfather, Sir Philip Wingrave. These last five all reside at the Wingrave ancestral estate, Paramore, where Owen is summoned once his aunt learns of his decision.

Most of the opera consists of Owen being attacked and berated by the Paramore clan, with Owen confirming his decision in the face of all attacks. All Wingraves hold their beliefs with an iron grip, and Owen is as steadfast as any of them. Piper and Britten confront the difficulty of James’s technique by essentially ignoring the original text, and providing lots of extra writing. Much of the middle of the first act is an interpolation based on Coyle’s statement in the story that when he sees Owen at paramour it appears as if he has aged five years—in the opera, we witness the attacks and insults that presumably induced that premature aging. If there is one major flaw in the opera, it is this prolixity, this need to make the conflict between those who revere the military and those, like Owen, who reject it, manifest. The terms of the argument are never in question; the positions take are unsurprising and uncontroversial. It is only the insistence of the music and the passion of the expression that keep it from becoming banal.

The staging was extremely simple and extremely effective (Jim Petosa was the stage director, and JiYoung Han the scenic designer). Most of the stage was empty, flanked by two huge Victorian doorways. Long strips of black material hang on the back of the stage, occasionally re-configuring to permit entrances and exits; the impression is of crepe at a funeral. The palette throughout is black and white, predominantly black and oppressive. The opera makes use of some effects that came from its origin as a television play—cross-cuts between distinct scenes, jump cuts, a dinner scene where the characters each step away from the scene and directly address the audience/camera. These were mostly handled effectively, though two cuts intended to show the passage of time in Act Two elicited unintentional laughter. Some of this blame can be handed to the composer; he also put together a small set piece on the word “scruples” which it is hard to hear in any production without laughing.

Owen Wingrave is something of a “problem opera,” but a problem opera from the best writer of opera in English in the 20th-century. (By the way, there are supertitles provided during the performance, but I rarely needed them). It is gratifying to see new opera performers of the caliber present on this evening, and it is even better to get to see them in unfamiliar repertoire. The Boston University School of Music Opera Institute is clearly on to something. This performance was a pleasure throughout, and the vocal and theatrical performances were convincing and aesthetically pleasing. The biographies of the participants are all impressive, many with international experience, and the performance was of the highest quality. The orchestra, led by William Lumpkin, started out a bit tentative but quickly found its voice, and the piece moved quickly and turned on dimes as necessary. The quality of the orchestra is critical for the piece—very few of the vocal lines are especially memorable on their own, but the orchestral textures and color around the voices provides the greatest interest. This is typical for Britten: for example, “Now the great bear” in Peter Grimes begins as a reciting tone which gains its emotional impact from the orchestral collapse around it. As such, I did occasionally find myself wishing we had the full orchestration, as much of the interest in the score comes from the way Britten uses percussion and brass, which are much cut down in this version. Vocally, I was particularly impressed by Zack Rabin as Spenser Coyle, for his ability to bring both gravity and levity to the same part; and by Brendan Daly as Lechmere in the opening scene, the one place where Britten allows us a sense of fun and joy—one could almost understand why one would want to be a soldier by seeing the pure joy Lechmere brings to brandishing a sword. Celeste Fraser as imposing Miss Wingrave was a force of nature, intimidating both through her stage presence and the force of her voice. Nickoli Strommer as Owen has a difficult job. Owen becomes a little tedious to listen to, and the vocal line given to him is for the most part sober and reasonable, not the stuff of great notice. However, the one great set piece in the opera is Owen’s “peace aria” in the second act, and there, among the suddenly released cascades of pitched percussion and the text oddly reminiscent of Corinthians, Strommer took the stage and commanded the hall. His command only made clearer the irony that the peace he is advocating has many of the same qualities as does military service.

The Boston University School of Music Opera Institute will present three more performances of Owen Wingrave Friday through Sunday. As this is a student performance of the Opera Institute, the cast on Friday and Sunday will be different from the one here reviewed.
