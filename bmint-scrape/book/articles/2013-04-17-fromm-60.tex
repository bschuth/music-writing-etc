\chapter{17 April 2013}

\textsc{Outbos}

On my way to Saturday night’s Fromm Foundation anniversary concert, I was accosted by a gentleman pushing a cart on the Red Line. He looked over my shoulder as I read Paul Griffith’s Modern Music and After. After some gruff preliminaries he asked “What do you think of Schoenberg?” I said I liked some of his music quite a lot, some not so much, which led to a lengthy, if one-sided and disjoint discussion more or less about musical modernism, which ended suddenly as he departed at Central Square. The concert ended up being something of an answer, if not to the specific question about Schoenberg, then to the more general question: “Why would you listen to that music?”

We all know what that music is—the music produced from the cultural ruins after World War II, simultaneously adventurous and experimental, and often perceived as hostile by audiences. The Fromm Foundation has, since 1952, been at the center of the discussion of where contemporary concert music is going. There were 4 pieces performed at Paine Hall Saturday night by the astonishing Sound Icon ensemble, each of which addressed the question differently. The concert was the second of two concerts commemorating the 60th anniversary of the Foundation (and the 40th anniversary of it taking up residence at Harvard), and there was a sense of celebration—there was a large and appreciative audience, and the celebratory occasion gave a reason to indulge in the luxury of a large ensemble.

Gunther Schuller’s Tre Inventione was written in 1972 but starts out sounding rather older. The opening movement was made from short phrases whose melodic content sounded strictly serial; my ear cannot keep track of a 12-note row played at speed, but the sound was certainly familiar. This movement would not have provided much of an answer to my subway interlocutor, but the next two movements showed why Schuller remains an important figure. The second was a study in extended notes and timbres, including a quite alarmingly shrill moment I will not forget, while the third had some jazziness to it, even something that sounded like a stride piano bass coming in and out of focus. Conductor Jeffrey Means directed the five quintets of five instruments that made up the ensemble, keeping the music clear while making the most of the surprises it had to offer.

One of the blessing and curses of modern concert music is that there is no one way to organize sound. With traditional harmony optional for nearly 70 years, each piece must both teach the listener how to listen to it while also being the listened-to object. Lee Hyla’s Pre-pulse Suspended for ensemble and piano was written in 1984 and uses rhythm and dramatic solos to draw in the listener. Violent quasi-improvisational lines are traded between the bass clarinet and strings at the beginning, with rhythmic irruptions that never quite settle into anything regular, but which engage the ensemble in intermittent spasms. This dissipates into a quieter, lyrical passage, before the opening gestures return transformed to close the piece. This music remains challenging. One may not always be able follow the argument that organizes the pitches, but Hyla provides enough threads into his labyrinth to keep the intellect engaged, and enough expressionistic zeal to keep emotions high. Gabriela Diaz on violin and Michael Norsworthy on bass clarinet jousted with passion and brutality, playing with an enjoyable savagery of tone and attack.

It would not be a Fromm concert without a new piece; unfortunately, the premiere, Karola Obermuller’s elusive corridors failed to hold my attention. Written for bass clarinet, piano and electronics, it felt formless, filled with tremolo gestures, multiphonics and cascading runs that may have had some interesting color, but which did not engage this listener. The reviewer of a brand new piece has some responsibility to go as far as possible to meet a piece halfway; the notes provided by the composer are often a help in this respect. Here is the first part of what was offered to help the listeners find their way into elusive corridors:

an incredibly heavy thing. so heavy that it is not fathomable. absolutely immobile. looking into a corridor that loses itself into infinity. no end in sight. the corridor seems strangely familiar, almost like a subway tunnel. from somewhere behind, objects keep being ripped into the passage, so extremely fast that none of them can be recognized.

Michael Norsworthy handled the technical complexities of the clarinet part with aplomb, and Yoko Hagino’s piano was supportive and sensitive.

Barbara White’s Third Rule of Thumb for percussion quartet also came with a lengthy note that explained this piece for nominally unpitched percussion in terms of development by Isolation, Opposition, Participation with a final Assimilation. Happily, this piece was readable without these categories—the various sounds of claves, triangles, cymbals, drums were introduced singly and then in ensemble, once coming together in a cloud of various noises trading off, later coming together in a more rhythmically unified section that the composer marks as “grooving.” The piece is crowd-pleasing and ingratiating, an excellent curtain raiser to the second half of the concert.

The rhythm and relative simplicity of Third Rule of Thumb is one way of reaching an audience; Elliot Carter’s Double Concerto for harpsichord, piano and two chamber ensembles is almost the polar opposite, and yet the piece has continued to appear on concert programs (Fromm once called it “Elliot’s Firebird” because of its popularity). The surface of Carter’s music may be initially as forbidding as that of any strict serialist—I can remember hearing the 1962 Epic recording of this as an undergraduate and quickly putting it away as incomprehensible. But now, with some years of critical listening behind me, I found the piece fascinating and constantly interesting, even if I lack the ability to describe exactly what is going on at any given moment. One can hear correspondences and echoes, can track the evolution of groups of pitches, and can tell when we move from one region of music to another. The piece is famous for exploring what Carter called “tempo modulation,” where changes in rhythm become perceived as changes in tempo. In the Concerto often these modulations are happening in several parts at several different rates of speed. It is was a luxury to hear the piece performed live by musicians who were up to the challenges—being able to place the various lines of music in space made the complexity of the piece easier to track, and the emotional commitment of the players of Sound Icon gave the various independent lines distinct personalities. Carter’s sound world is utterly personal and internally coherent, and his work changes how musical time and expression are perceived. Modernist music offers insights that are available nowhere else. I’m not sure that would satisfy the man on the train, but for one evening, at least, it was possible to sit in a hall full of people who felt that way, hearing musicians who made convincing cases for this art.
