\chapter{15 November 2012}

\textsc{Xenakis concert}

Iannis Xenakis In First Person, a combination one-man monologue and concert of the work of the 20th-century Greek composer, aims to bring the audience to a closer understanding of the music by hearing the words of the composer spoken straight from the man himself, as it were. In the incarnation presented by ALEA III at the Tsai Performance Center on Wednesday night, the words were not particularly successful; but the music was, sometimes overwhelmingly so.

The focal point of the evening was Persephassa, performed by the Boston University Percussion Ensemble under the direction of Samuel Z. Solomon. It will be performed again this Friday, November 16th at 8 p.m. at the Boston University CFA Concert Hall, and it is an experience you should not miss. Written for six players, playing a battery of struck instruments, positioned around the edges of the room, it provides devastating impact as well intellectual stimulation. Despite the fact that it does away with traditional notions of melody or harmony, it is closely argued; despite the absence of traditional notions of form, it bears close attentive listening and its developmental methods are audible and inevitable. A professor of mine once said that great works “teach you how to read them”, and Persephassa does it exactly this. Clocking in at a bit under a half an hour, it uses that time with great care, introducing new elements and developing previous motives. The range of instruments is dazzling -- drums, symbols, gongs, tuned pieces of metal and of wood, maracas, mouth sirens – and Xenakis combines them slowly at  first, and then more rapidly as the piece progresses. As a piece of spatial music, it is kin more to Gabrieli than to Henry Brant. Xenakis isn’t merely playing with sound in space, he’s using physical separation to create dialogue between players, or to heighten the sense of an enveloping mass. When the players play in synchronization, the effect is powerful and overwhelming. When they play in independent meters, one listens to it much as one would listen to a six-voice Bach fugue – switching between either tracking individual parts, or allowing the combination of sounds to come together in a texture. One of the most striking moments comes after the “Acme mouth sirens” (that’s what they are called in the score!) have entered. They sound comical and cartoonish when they first appear; at one moment, Solomon made an upward gesture, and they all swooped in unison with a goofy kazoo-like feeling, causing the audience to laugh out loud. But there was a point where, after some gentle noodling, there was a sudden, terrifying sound of hard sticks on tam-tams, and in that moment the sirens suddenly sounded like they were warning of an air-raid. For sheer dramatic contrast it called to mind Shostakovich; as a study in how context influences perception, it was pure Eisenstein. Embedded in a “theater piece”, Persephassa provided the most theatrical experience of the night. Mr. Solomon stood on stage in a spotlight, his six players arrayed around him, appearing as a compelling focal point. His movements (or lack of them in the more independent passages) provided a visual cue to help structure one’s hearing of the piece. The playing was vigorous, exciting, precise and impressive.

This was followed by Okho, a relatively late piece (1989) for three djembés played joyfully by members of the Boston Conservatory Percussion Ensemble. In contrast to Persephassa, during most of Okho the players share a common pulse, and in place of the battery of instruments in the larger piece, these players have available the not inconsiderable ranges of pitch, attack and timbre of these versatile African drums. Think of the world’s most skillful and intellectual drum circle, and you get an idea of the ambience of Okho. Full of rhythmic surprises and interplay, trading of phrases and dense ensemble playing, Okho was invigorating, urgent and… charming? Why not, charming!

The evening opened and closed with duos. The early work Dhipli Ziya (1951) for violin and cello was the first piece performed. Informed by Eastern European folk music, with moments of virtuosity, this piece duo was unlike later Xenakis, being broadly tonal and structured around traditional melody and harmony (although an aggressive Bartokian harmony). It was played with passion and vitality by Sasha Callahan, violin, and Leo Eguchi, cello. One can admire the workmanship of the piece without wishing there were more like it.  The final piece was Charisma, a duo for cello and clarinet. Unlike the relatively ingratiating Dhipli Ziya, Charisma is a unflinching and unsentimental commemoration of the untimely death of a friend of Xenakis of a heart attack at 37; its epigraph is from the Iliad, “then the soul like smoke moved into the earth, grinding”. This evening, it stood as an expression of the pain and despair suffered by Xenakis as he slowly lost his faculties, becoming unable to work and ultimately unable to recognize his home or family. It is program music, a four-minute depiction of the sundering of soul and body, and of the soul’s lost wandering. The instruments produce extreme sounds — grinding the bow on the bridge of the cello, key noises and multiphonics from the clarinet — to depict a harrowing outcry in the face of death. Think of Anton Webern by way of Francis Bacon. Clarinetist Diane Heffner joined Mr. Eguchi for a moving and convincing performance.

The theatrical concept, generated by Alexandros Mouzas and realized by Antonios Ant\-on\-o\-pou\-los, alternated the music with a series of monologues taken from the writings and interviews of Xenakis. This would seem to have required some work. The second half concluded with a lengthy section of a Greek television documentary which consisted of Xenakis holding forth among galleries of Greek antiquities. To judge by this, Xenakis was drawn to sweeping generalizations about music, philosophy, history and politics. In this film, he comes dangerously close to seeming like crank, so caught up in his own ideas as to be careless about expressing them clearly (to be fair, this may be an effect of subtitle translations, which seemed extremely awkward). The assembled texts of In First Person were edited carefully and were less digressive than the documentary, but nevertheless the monologues had a patchwork quality. Important biographical information flies by: the death of his mother when he was “five or six,” his conversion and then aversion to religion are all dispatched within the space of 30 seconds. Some of the speeches give the impression of a fantastically arrogant man — it is unclear whether this is the intended effect, or just the effect of having to compress his thoughts and achievements into a brief spoken moment. I only began to feel a sense of real human connection near the end of the piece, when we heard heartbreaking fragments of letters written when he was losing his memory to ill health. Ironically, these are words written or spoken after he had stopped composing. The young actor Jake Murphy, playing Xenakis, was overmatched. I don’t know how much time he was given to prepare, but it was not enough to enable him to master the text; he appeared to be working from the script. He was amplified, which drained color from his voice and was unflattering to his often dry delivery. The set, which in the original Athens production was a small but evocative realization of Xenakis’ study, was here a simple table with a black fabric skirt and a water pitcher. It evoked a sales meeting in a hotel conference room rather than the workplace of a groundbreaking composer. The cumulative effect of these choices was to distance Xenakis, to turn these statements (whether intellectual or emotional) into a rather dry lecture rather than an appeal to a human audience.

The overtly theatrical gestures were mixed in their effect. Pianist Yukiko Shimazaki was required to play the fearsome Evralyi while the score was projected above her. The score emphasizes the vicious demands Xenakis is making — there are stretches where Xenakis notates this relentless piece, made of mostly of masses of sixteenth notes, spread across five staves, as if this was the only way to make it look as hard as possible. Shimazaki seemed to play those notes in place and more or less in rhythm, but the performance lacked spark. Perhaps the fact that we could all see what she was supposed to be playing inhibited the sense of risk taking that the cruelly complex score would seem to encourage.

Videos by Vicky Betsou that accompanied excerpts of Xenakis’ electroacoustic works were very attractive, affording the audience a chance to see the graphical work that preceded the rendering of some of Xenakis’s orchestral pieces into conventional notation. But the brief excerpting reduced the pieces themselves to abstract chunks of sound without impact. The video’s images of young people huddled in sleeping bags under laser lights at the baths of Cluny gave one only a tourist’s idea of what a mind-blowing experience the Polytopes de Cluny might have been.

Despite the show’s weaknesses I stepped out into the cool night air invigorated and stimulated. Congratulations to ALEA III, and I can only ask for more challenges of this sort in the future.
