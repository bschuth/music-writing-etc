\chapter{22 November 2013}

\textsc{Boxxx}

Lizzie Borden is a shadowy cultural memory, a Fall River woman who murdered her parents with a axe, “forty whacks” and all that, but when one looks more closely at her story, there’s not much more to hold on to. She was acquitted of the murders, and no culprit was ever found. There were dark murmurings of familial repression and authoritarianism. Lizzie herself lived the rest of her life in Fall River. She’s an almost blank slate of ambiguity upon which any of number of stories can be written. The Boston Lyric Opera’s production of Jack Beeson’s Lizzie Borden, which began Tuesday at the Park Plaza Castle, dispenses with some of that ambiguity right up front; the opera begins with Lizzie stalking silently on stage carrying that infamous axe, and she buries its head in the family kitchen table. With that we leave behind any question of culpability, and instead plunge into an 80 minute slow motion nervous breakdown, a dramatic justification for an act of horror and violence.

There is little plot to drive the evening; instead there is a web of relationships, each of which contributes its toxicity to the Borden family environment. Lizzie (mezzo-soprano Heather Johnson) mourns her late mother and resents her stepmother Abigail (soprano Caroline Worra), who monopolizes her banker husband Andrew’s affection (bass-baritone Daniel Mobbs). She helps her sister Margret (soprano Chelsea Basler) to escape the house with Jason MacFarlane, a ship’s captain (baritone David McFerrin). Lizzie appears to have only one friend, the Reverend Harrington (tenor Omar Najmi), to whom she seems fruitlessly attracted. The opera proceeds through seven scenes: opening scenes which establish the characters lead to a climactic scene of confrontation, leading to the inevitable catastrophe and brief denouement. The production takes place on an enormous, sharply rectangular stage designed by Andrew Holland. Its linoleum floor rises up against a similarly enormous historical picture of the Borden’s house, which is severely tilted, as if the plane of the stage had upended the house. The setting is spare and threadbare: the linoleum downstage is worn away, exposing bare wood; atop it is that kitchen table, a metal, mid-20th-century middle-class table, with cushioned chairs.

Beeson and librettist Kenward Elmslie consciously structured the piece on the Greek story of Electra, and the piece feels structurally like a Greek tragedy, being a series of personal confrontations that lead to a fateful decision. This occurs when Andrew rejects MacFarlane’s proposal to Margret, instead insisting he must marry Lizzie instead. This act of malice (which is a wholesale invention of the authors) plunges Lizzie into erotic confusion, in which she begins to desire MacFarlane for her own. Into this blunders Abigail, who torments Lizzie, dressing her in her dead mother’s wedding gown, and precipitating her own death.

A theatre piece that moves episodically towards a character’s mental collapse, Lizzie Borden bears more resemblance to Wozzeck than to, say, Lucia di Lammermoor.  Neither piece bears a musical resemblance to Lizzie, which speaks in an American mid-century tonal idiom, Barber shorn of his more pungent aromas; symphonic Bernstein without the jazz. Both Lizzie Borden and Wozzeck are willing to indulge in cartoonish stereotyping of the protagonist’s tormentors; in both operas this can lead to dramatic problems, as the morally “bad” characters can become more theatrically interesting than the good. This certainly is a risk this production does not avoid – Worra’s “evil stepmother” is played with a blowsy vivacity, smoking constantly and parading around in leopard-skin fabric. She is the only person in the opera who seems authentically alive, and her energy animates the other characters when they come into contact with it: for Lizzie, it provokes her to confrontation; for Andrew, it reduces him to a kind of erotic jelly.  Even her cruelty has a liveliness to it—she’s a cat playing with a mouse when she humiliates Lizzie, taking a full throated cackling enjoyment in the ludicrousness of it all. Worra plays the part with great humor – some of which is written in (the ridiculous stratospheric notes interpolated as she sits singing at the harmonium), some of which comes from her carriage and swagger. Mobbs’s Andrew begins the evening staring catatonically, but once he comes to life, he finds a kind of grim animation as he proclaims the dogmas of his tight-fisted New England thriftiness. When he begins dictating to his difficult children, his music becomes martial and Andrew almost perky. Mobbs succeeds in bringing a range of colors to a character that might almost have been intended to be colorless. By contrast, Johnson’s Lizzie is sentimentally stoic for most of the evening, though as her character’s dissolution accelerates at the end she becomes thrilling to watch. At the peak of her unraveling she sheds her glasses and lets her hair down, and Johnson’s singing and acting suddenly acquire a terrifying animal quality. Johnson has a powerful instrument, well-focused, rich and well-defined. Her Lizzie is a creature of anger and pent-up energy, one whose madness owes more to Medea than to Lucia, and so she has little in the way of florid passagework. Instead, her music is heroic, even in extremity, and her Lizzie never fades or falters.

For an opera whose action is driven by psycho-sexual confusions and competitions, the piece is strangely cold and distant, though with the already noted exception of Abigail’s sexual spell over Andrew. What passes for young love here, the relationship of Margret and Jason, is perfunctory and awkward. Basler is made to portray Margret as an asylum inmate, shoulders slumped and head tilted forward, costumed unflatteringly. When Margret and Jason sing to one another, they sing of escape and comfort, of islands and gardens, of a longing to merely be elsewhere, and safe.  Lizzie may or may not be enamored of the Reverend, but he barely exists, and could conceivably have been dispensed with entirely.

This might be due to the fact that what is being presented is a new chamber version of this opera, commissioned by the BLO. The original opera was presented it 1965 and acquired some measure of success at that time; I am not familiar with the original, and so cannot assess the nature and quality of the work performed by orchestrator Todd Bashore and dramaturg John Conklin. The stage direction (by Christopher Alden) emphasizes this distance between the characters, who frequently converse without facing one another. Therese Wadden’s costumes and Jason Allen’s wig and makeup design subtly emphasize the ritual stereotyping of the production, signaling messages about role and position through typical clothing of the 1950s: Andrew’s conservative suit, Jason’s windbreaker and hat, Abigail’s cocktail lounge outfit and impressive hair.

The opera is being presented in a cavernous former drill room of what was, originally The First Corps Cadets’ Armory. The inclined plane of the set is surrounded by chairs on three sides; I was sitting facing the stage, and it appeared that much of the action is focused in that direction. The sightlines are not ideal; when Lizzie hides under the kitchen table near the end of her collapse, I found myself forced to admire the very expensive haircut of the gentleman in front of me, whose head completely obscured the action. For a large space with minimal acoustic support, everything was quite audible, if not always clearly defined; the projected supertitles are absolutely required to make sense of the libretto. The chamber ensemble of 19 players led by David Angus made some wonderful sounds; there are moments of glassy harmony in Beeson’s score that positively glowed. The orchestra is located to the right of the stage, and perhaps this and the tubby sound of the space is responsible for the cluttered quality of the sound when there was more going on in the orchestra. A small chorus of children, drawn from the Handel and Haydn Youth and Women’s Chours and the Pals Children’s Chorus, appear at the opening and closing of the piece: first they sing charmingly as a church chorus, and at the end tauntingly, singing the old “forty whacks” playground chant, as Lizzie, having recomposed herself, steels herself for the rest of her life.

Lizzie Borden runs through the rest of the weekend and will also make an appearance at Tanglewood in July.
