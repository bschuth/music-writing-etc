\chapter{21 October 2013}

\textsc{Boxxx}

Boston Baroque’s first “New Directions” concert of the season offered an engaging and personable evening of music this past Saturday at the Pickman Concert Hall at Longy. Inexplicably entitled “Time Curves”, the program put together works by Corelli, de Falla, Rameau and Boston Baroque’s Music Director Martin Pearlman. None of the works was terribly demanding, and as a group they did not have a whole lot to say to each other. However, they were all ingratiating, and none was so long that it overstayed its welcome. It was charming enough that one might have hoped for a larger audience to appreciate it, but the Red Sox were clinching the pennant that night, and I expect audiences all over the city were showing the effects.

The opening four-movement Trio Sonata, Op. 3, No. 1 by Corelli showed the composer in his most professional, audience pleasing form. The movements are all brief, constructed with clarity and charm, conducting their business effortlessly. The solo lines were played by Dan Stepner and Julia McKenzie on violin, with Pearlman and viola da gamba player Laura Jeppesen providing the continuo. The two violinists presented a sharp contrast in approach. Stepner’s playing was extroverted, pointed, sharply articulated, while McKenzie’s lines were less rhetorical, more nuanced, with a smoother profile. As is often true in Corelli, much of the music is imitative, and hearing the lines played with such different inflections gave the impression of a conversation between two divergent personalities, who perhaps were not always exactly listening to each other. Even though at times the differences were so great as to suggest issues with ensemble, it was ultimately successful, roughing up a musical surface just enough to engage the attention in music that could easily have flowed right on past without registering.

These four players were joined by soprano Teresa Wakim for the “cantate” Orphée by Louis-Nicholas Cl­érambault (1676-1749). This was a set of four couplings of recitative and air that sketched just the middle of the Orpheus legend, starting after Eurydice’s death and before Orpheus looks back. As such it hardly plumbs the depths of the myth -- the piece ends by celebrating Orpheus’ “resounding victory,” which is rather jarring if you know what is supposed to come next. Wakim has an attractive and supple voice that she uses expertly, with French diction clear enough for me to follow a text only provided in translation, despite my terrible French. She is asked to present the words of all of the characters: the omniscient narrator, Orpheus, and Pluto; she did so without attempting to differentiate between them, perhaps missing some opportunity to invest the piece with a bit more drama. However, the cantate exists primarily to allow Clérambault the chance to show off his inventiveness in the crucial aria where Orpheus seeks to convince Pluto to release Eurydice. It opens with just the violins and harpsichord playing quite high, imagining Orpheus’ lyre as a rather delicate, even fragile instrument. This is followed by a dark and halting plea for Eurydice, and then by a perky, dancing interlude where Orpheus asks Pluto to think of his own desire and love for his wife, Proserpina, before closing with a recapitulation of the plea. It is effective if a bit odd, and the appeal to Pluto’s remembered desire allowed Wakim to demonstrate her lithe scale work. The piece moves briskly and leaves a pleasant impression that does not linger.

The main work on the program was Manuel de Falla’s Concerto for Harpsichord, Flute, Oboe, Clarinet, Violin and Cello from 1923-26. Pearlman stated that this was the first harpsichord concerto of the 20th century, and that it represents the initiation of de Falla’s neoclassical period. It certainly feels transitional and experimental.  The harpsichord part is frequently thick with notes, perhaps in an attempt to keep this instrument audible, an attempt which was only fitfully successful. Often the harpsichord would sound as if it were in a stereo recording being played back on a system with a broken speaker—it would receive focus, but would seem recessed, unequal. In the first movement, the other instruments were asked to shape melodic lines made out of very brief note values widely separated by rests. This was clearly designed to keep them from swamping the keyboard, but the effect was often awkward and uncomfortable. The ensemble played in front of the harpsichord, which meant that not only was the keyboard sonically in the rear, but the audience did not have the benefit of seeing it played (at least not in the seats on the floor).  The final movement was the most successful, with a sharp humor and lightness that recalls minor French neoclassicists—Jean Francaix, perhaps, with a bit more gravity and a bit less humor. The audience gave out a light chuckle as the piece came to an end, and gave it a warm reception. Pearlman and McKenzie were joined by Sarah Brady (flute), Jennifer Slowik (oboe), William Kirkley (clarinet) and Rafael Popper-Keizer (cello).

Donald Berman came out after intermission to play Martin Pearlman’s Variations on WoO 77 – Fantasy on a Theme of Beethoven, written in 2001 and revised in 2011.  WoO 77 is also a set of variations, often described as “six easy variations” in G major. The theme shared by both pieces is gracious and technically uncomplicated. Beethoven keeps his variations equally simple, while Pearlman allows himself to impose a variety of contrasting styles on the overall shape of the theme. There’s a fragmented rag, a tribute to Schumann, and a spectral waltz among others – all of them having a strong personality and occasionally eliciting amused reactions from the audience. It is a commonplace to talk about a theme coming back “transformed” at the end of a set of variations—Pearlman makes the transformation literal, as the theme starts out verbatim and then slowly stumbles to its end as we discover the injuries it has suffered. Berman played with his usual accomplishment, able fully to realize the full schizophrenia of these bagatelles without any unwanted excess.

The concert closed with Rameau’s Troisìeme Concert from the Pìeces de clavecin en concert, a set of pieces intended to be performed on keyboard alone, or with a variety of possible instruments. Much of the music has the instruments doubling voices in the keyboard, though at times the keyboard recedes to provide accompaniment. In this performance Perlman, Stepner and Jeppesen returned to provide a beautifully inflected performance, with the balances between the harpsichord, violin and viola da gamba precisely calculated. It was a particular pleasure to get to hear Jeppessen play figures in the foreground rather than serving as continuo; her tone was simultaneously gentle and insistent when she was carrying the lead. The final tambourin was a rustic fiddle piece, and the players happily dug in and played with wild abandon, sending us out into the night before any runs had scored at Fenway.
