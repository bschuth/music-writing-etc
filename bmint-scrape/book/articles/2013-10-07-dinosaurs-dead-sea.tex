\chapter{7 October 2013}

\textsc{Boxx}

Dinosaur Annex opened its 2013-2014 season at Brandeis’s Slosberg Music Center with  “Dead Sea Scrolls, an evening with Tony Arnold”. Presented in conjunction with the Dead Sea Scrolls: Life in Ancient Times, an exhibition hosted by the Museum of Science for which Brandeis is the educational partner, the concert was connected to the exhibition by one brand-new work: Where it Finds Nothing But the Wind, a piece for ensemble and soprano based on ten texts from the scrolls, composed by Eric Chasalow, the Irving G. Fine professor of music at Brandeis.

As fascinating as the Dead Sea Scrolls are, from a musical point of view, the main focus of the evening had to be Tony Arnold. Over the last decade she has amassed a formidable reputation in the performance of contemporary music, from Webern to the over 250 works she has premiered. She performed two Chasalow pieces this evening: in addition to the premiere, she also sang an earlier piece of Chasalow’s, The Furies.

Chasalow is new to me: on the evidence of these two compositions, he has a remarkable talent for writing vocal music that is simultaneously challenging and idiomatic. The two pieces share a harmonic language that can vary from moment to moment, especially in the newer piece, but most of the vocal phrases have shapes and intervals that hew to a fractured tonality. In Wind, the instrumentation (flute, percussion, guitar, electronics) hinted at the textures of Boulez’ Le marteau sans maître, but where the vocal lines in that piece sound strained and even tortured, Chasalow’s writing is never forbidding, even when calling for huge leaps or rapid passagework. In both pieces the texts are presented without repetition, or at least with very little (there were moments in the Hebrew and Aramaic of Wind where it was unclear); the settings tend to be syllabic, with melisma used with discretion.

The use of electronics is also a distinctive feature of these works, and here the distance in time between the two works is noticeable. The Furies was composed in 1984 poems of Anne Sexton from her book of the same name; the piece is for soprano alone with electronics. These are not traditional sung texts—lengthy, impassioned, syntactically complex and disturbing and mysterious in their imagery. Chasalow makes of them a heightened declamation, where the text is primary and the musical setting casts light and shadows around the words, placing them in relief while avoiding italicizing them. The act of placing the syllables of the poetry over notes gives the piece a sense of controlled hysteria. A sentence like “I wonder, Mr. Bone Man, what you’re thinking/of your fury now, gone sour as a sinking whale,/crawling up the alphabet on her own bones,” which seems cryptic on the page now also carries emotional and dramatic weight. Sexton’s poems require concentration to pick up off the page, so the printed text was still necessary to follow the piece, but Chasalow’s writing kept the words clear and comprehensible, without making them prosaic. Arnold was the ideal interpreter of this work; her voice is a rich, strong and centered soprano that in these pieces is exquisitely controlled and flexible. Sudden leaps to the stratosphere happened with just the right amount of effort, depending on the musical situation—at one moment they seem almost tossed off, like a leaf on wind; when more drama is required, there is more force and more struggle. She has an innate theatricality: her face is responsive, her body language tuned to the expressive requirements of the text. However, she never draws attention to herself or to her expressions. The events of the music express themselves through her body without underlining. As the she sang Sexton’s poem, the electronic accompaniment constantly burbled and commented around her; this was something of a mixed bag. A soloist interacting with a bodiless electronic score always strikes me as alienating; if that was Chasalow’s intent, Arnold’s ability to connect with the audience certainly undermined it. The electronic accompaniment formed a kind of cloud around Arnold, making the experience of declamation less stark and naked, and occasionally an evocative or witty echo or comment might be heard between the parts. The sounds and effects certainly sounded like 1984; some of the timbres caused uncomfortably vivid memories of video games, and some of the effects (sudden crescendo as if the sound were played backward, certain ringing resonances) sounded clichéd. Shorn of these kinds of sonic experimentation, it sounded at least plausible that the accompaniment might be arranged for conventional instruments (perhaps with some extended techniques) which might rescue the piece from the 1980s, and would soften the woman vs. machine quality of the work. The last decade has seen some interesting work in arranging music that was originally arranged for mechanical devices: Nancarrow’s player piano music most notably, but also in Alarm Will Sound’s acoustic arrangements of Aphex Twin.

The electronics in Where it Finds Nothing But the Wind are less pervasive, and their sounds now echo those of electronica, with deep resonating sounds and bell-like moments. The ten movements of the piece are set with great variety: one movement was entirely electronic, a couple made generous use of reverberation and echo. The conventional instrumentalists were frequently called upon to use extended techniques: the flute speaking into her instrument, the guitar playing above the nut. There was a touch of imitative exoticism in the percussion. However, all this variety left me a little cold, the music written within a narrow emotional range. Although the texts range from as apocalyptic pronouncement (“And the shaft of a spear; And they must be burned up on the spot”) to a wild story from Enoch where the “sons of the sky” impregnate human woman so that they give birth to 3000 cubit giants, without the printed texts one might be hard put to match the arrangement with the excerpt. The one exception is the sunny and light penultimate movement of Benediction, where the flowering terebinths and delightful palm leaves are evoked through skitterings of notes shared by the soprano and flute. The story from Enoch seemed in particular a lost opportunity – the opening fragmented the texts so thoroughly that some audience members could be seen searching through the printed text to be sure they had not lost their way. The impression, perhaps intended, was of embarrassment at the sexuality of the story, or at least of its surface zaniness. But with so many of the texts sounding so typically “scriptural” and non-narrative (“You who seek faithfulness, listen to my words, all that comes out of my lips”), it seems a shame to undercut the one outlandish story. The same vocal profiles heard in The Furies are on display here, though they are more likely to dart off in unexpected directions. The texts are sung in their original languages, and although translations were provided I was unable to gain a sense of how well the musical material tracked the sense of the words. Arnold was if anything more impressive, completely inhabiting the music, giving a sense of mastery of the material that was never overtly controlling. Sue-Ellen Hershman-Tcherepnin, flute; Jonathan Hess, percussion; and Daniel Lippel, guitar, made up the ensemble of accompanying instruments, with Chasalow credited for performing the electronics.

In addition to the Chasalow/Arnold pieces, the Dinosaurs performed two instrumental works that demonstrated the breadth of musical expression available from the same compositional strategy. Both Cendres (1998) by Kaija Saariaho (flute, cello, piano) and The Riot (1993) by Jonathan Harvey (clarinet, flute, piano) make a point of juxtaposing strongly characterized material – sometimes overlapping, sometimes in extreme opposition. Saariaho’s materials are spectral and blurry. The piece develops out of a single pitch sounded in the piano, which then expands into octaves; the flute and cello frequently move in and out of phase with each other, bending and winding pitches around each other. When the instruments depart from one another, the music becomes dense and busy, as if the players are all talking past one another – the music threatens to fall apart. By contrast, Harvey’s materials are “chunkier”, some of them evoking boogie woogie, and one descending harmonic gesture in the piano which sounded as if it must be a quote from some nineteenth-century piece, but whose provenance escaped me. Where Saariaho’s music intertwines, Harvey’s bangs together; Cendres is by turns mysterious and sharply chaotic; The Riot is comic and buffoonishly violent. Donald Berman’s piano and Sue-Ellen Hershman-Tcherepnin’s flute appeared in both pieces; Joshua Gordon played cello in Cendres and Diane Heffner, clarinet in The Riot.
